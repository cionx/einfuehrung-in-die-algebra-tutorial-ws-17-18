\section{}

Es sei $\pi \colon  K[t] \to K[t]/(p(t))$, $q(t) \mapsto \class{q(t)}$ die kanonische Projektion.
Da $p(t)$ das Minimalpolynom von $a$ ist, faktorisiert der Auswertungshomomorphismus
\[
          \ev_{a}
  \colon  K[t]
  \to     L \,,
  \quad   q(t)
  \mapsto q(a')
\]
durch $\pi$ zu einem Körperisomorphismus
\[
          \induced{\ev_{a}}
  \colon  K[t]/(p(t))
  \to     K(a)  \,,
  \quad   \class{q(t)}
  \mapsto q(a)  \,,
\]
der das folgende Diagramm zum kommutieren bringt:
\[
  \begin{tikzcd}[column sep = large]
      K(a)
    & K[t]/(p(t))
      \arrow{l}[above]{\induced{\ev_a}}
    \\
      {}
    & K[t]
      \arrow{u}[right]{\pi}
      \arrow{ul}[left below]{\ev_a}
  \end{tikzcd}
\]
Wir schreiben im Folgenden abkürzend $p'(t) \coloneqq f_*(p(t))$.





\subsection{}


Es sei $\pi' \colon  K'[t] \to K'[t]/(p'(t))$, $q'(t) \mapsto \class{q'(t)}$ die kanonische Projektion.
Da $a'$ eine Nullstelle von $p'(t)$ ist, faktorisiert der Auswertungshomomorphismus
\[
          \ev_{a'}
  \colon  K'[t]
  \to     L'  \,,
  \quad   q'(t)
  \mapsto q'(a)
\]
durch $\pi'$ über einem Ringhomomorphismus
\[
          \induced{\ev_{a'}}
  \colon  K'[t]/(p'(t))
  \to     K'(a')  \,,
  \quad   \class{q'(t)}
  \mapsto q'(a)   \,,
\]
der das folgende Diagramm zum kommutieren bringt:
\[
  \begin{tikzcd}[column sep = large]
      K'[t](p'(t))
      \arrow{r}[above]{\induced{\ev_{a'}}}
    & L'
    \\
      K'[t]
      \arrow{u}[left]{\pi'}
      \arrow{ur}[below right]{\ev_{a'}}
    & {}
  \end{tikzcd}
\]
Es gilt $\ker \pi' = (p'(t))$ und somit $p(t) \in \ker(\pi' \circ f_*)$, weshalb der Ringhomomorphismus $\pi' \circ f_* \colon K[t] \to K'[t]/(p'(t))$ nach dem Homomorphisatz einen Homomorphismus
\[
          \tilde{\varphi}
  \colon  K[t]/(p(t))
  \to     K'[t]/(p'(t))
  \quad   \class{q(t)}
  \mapsto \class{f_*(q(t))} \,,
\]
faktorisiert, der das folgende Diagramm zum kommutieren bringt:
\[
  \begin{tikzcd}
      K[t]/(p(t))
      \arrow{r}[above]{\tilde{\varphi}}
    & K'[t]/(q'(t))
    \\
      K[t]
      \arrow{u}[left]{\pi}
      \arrow{r}[above]{f_*}
    & K'[t]
      \arrow{u}[right]{\pi'}
  \end{tikzcd}
\]
Ingesamt haben wir damit das folgende kommutierende Diagramm:
\[
  \begin{tikzcd}[column sep = large]
      K(a)
    & K[t]/(p(t))
      \arrow{l}[above]{\induced{\ev_a}}
      \arrow{r}[above]{\tilde{\varphi}}
    & K'[t]/(p'(t))
      \arrow{r}[above]{\induced{\ev_{a'}}}
    & L'
    \\
      {}
    & K[t]
      \arrow{u}[left]{\pi}
      \arrow{ul}[left below]{\ev_a}
      \arrow{r}[above]{f_*}
    & K'[t]
      \arrow{u}[right]{\pi'}
      \arrow{ur}[below right]{\ev_{a'}}
    & {}
  \end{tikzcd}
\]
Der gesuchte Homomorphismus $\varphi$ ist nun $\varphi \coloneqq \induced{\ev_{a'}} \circ \tilde{\varphi} \circ \induced{\ev_a}^{-1}$.
\begin{equation}
  \label{equation: big diagram}
  \begin{tikzcd}[column sep = large]
      K(a)
      \arrow[bend left = 25]{rrr}[above]{\varphi}
    & K[t]/(p(t))
      \arrow{l}[above, near start]{\induced{\ev_a}}
      \arrow{r}[above]{\tilde{\varphi}}
    & K'[t]/(p'(t))
      \arrow{r}[above, near start]{\induced{\ev_{a'}}}
    & L'
    \\
      {}
    & K[t]
      \arrow{u}[left]{\pi}
      \arrow{ul}[left below]{\ev_a}
      \arrow{r}[above]{f_*}
    & K'[t]
      \arrow{u}[right]{\pi'}
      \arrow{ur}[below right]{\ev_{a'}}
    & {}
  \end{tikzcd}
\end{equation}
Betrachtet man, wie das Element $t \in K[t]$ in dem obigen Diagramm herumgeschoben wird, so erhalten wir das folgende Diagramm:
\[
  \begin{tikzcd}[column sep = large]
      a
      \arrow[|->, bend left = 35]{rrr}[above]{\varphi}
    & \class{t}
      \arrow[|->]{l}[above, near start]{\induced{\ev_a}}
      \arrow[|->]{r}[above]{\tilde{\varphi}}
    & \class{t}
      \arrow[|->]{r}[above, near start]{\induced{\ev_{a'}}}
    & a'
    \\
      {}
    & t
      \arrow[|->]{u}[left]{\pi}
      \arrow[|->]{ul}[left below]{\ev_a}
      \arrow[|->]{r}[above]{f_*}
    & t
      \arrow[|->]{u}[right]{\pi'}
      \arrow[|->]{ur}[below right]{\ev_{a'}}
    & {}
  \end{tikzcd}
\]
Insbesondere gilt somit $\varphi(a) = a'$.
Für jedes $x \in K$ erhalten wir das folgende Diagramm:
\[
  \begin{tikzcd}[column sep = large]
      x
      \arrow[|->, bend left=30]{rrr}[above]{\varphi}
    & \class{x}
      \arrow[|->]{l}[above, near start]{\induced{\ev_a}}
      \arrow[|->]{r}[above]{\tilde{\varphi}}
    & \class{f(x)}
      \arrow[|->]{r}[above, near start]{\induced{\ev_{a'}}}
    & f(x)
    \\
      {}
    & x
      \arrow[|->]{u}[left]{\pi}
      \arrow[|->]{ul}[left below]{\ev_a}
      \arrow[|->]{r}[above]{f_*}
    & f(x)
      \arrow[|->]{u}[right]{\pi'}
      \arrow[|->]{ur}[below right]{\ev_{a'}}
    & {}
  \end{tikzcd}
\]
Also gilt $\restrict{\varphi}{K} = f$.

\begin{remark}
  Die obigen beiden Diagramme lassen sich auch in Gleichungsketten übersetzen:
  Es gilt
  \[
      \varphi(a)
    = \varphi( \ev_a(t) )
    = \varphi( \induced{\ev_a}( \class{t} ) )
    = \induced{\ev_{a'}}( \tilde{\varphi}( \class{t} ) )
    = \induced{\ev_{a'}}( \class{f_*(t)} )
    = \induced{\ev_{a'}}( \class{t} )
    = \ev_{a'}(t)
    = a' \,,
  \]
  und für jedes $x \in K$ gilt
  \begin{align*}
        \varphi(x)
    &=  \varphi( \ev_a(x) )
     =  \varphi( \induced{\ev_a}( \class{x} ) )
     =  \induced{\ev_{a'}}( \tilde{\varphi}( \class{x} ) )
    \\
    &=  \induced{\ev_{a'}}( \class{f_*(x)} )
     =  \induced{\ev_{a'}}( \class{f(x)} )
     =  \ev_{a'}(f(x))
     =  f(x) \,,
  \end{align*}
  und somit $\restrict{\varphi}{K} = f$.
\end{remark}





\subsection{}

Ist $\varphi \colon K(a) \to L'$ ein Körperhomomorphismus mit $\restrict{\varphi}{K} = f$, so gilt für das Element $a' \coloneqq \varphi(a) \in L$ mit $p(t) = \sum_i p_i t^i$, dass
\begin{align*}
      p'(a')
  &=  f_*( p(t) )( a' )
   =  \left( \sum_i f(p_i) t^i \right)( a' )
   =  \sum_i f(p_i) (a')^i
  \\
  &=  \sum_i \varphi(p_i) \varphi(a)^i
   =  \varphi\left( \sum_i p_i a^i \right)
   =  \varphi( p(a) )
   =  \varphi(0)
   =   0 \,.
\end{align*}
Hieraus folgt, dass die angegebene Abbildung wohldefiniert ist.
Nach dem vorherigen Aufgabenteil ist sie surjektiv.
Sie ist auch injektiv, denn $K(a)$ wird als Ring von $K \cup \{a\}$ erzeugt, so dass jeder Ringhomomorphismus $\varphi \colon K(a) \to L$ durch die Einschränkung $\restrict{\varphi}{K}$ und das Bildelement $\varphi(a)$ bereits eindeutig bestimmt ist.





\subsection{}

Da $\varphi$ ein Körperhomomorphismus ist, und somit inbesondere injektiv, genügt es zu zeigen, dass $\im \varphi = f(K)(\varphi(a))$ gilt.
Indem wir $K'$ durch $f(K)$ ersetzen, können wir o.B.d.A.\ davon ausgehen, dass der Körperhomomorphismus $f$ surjektiv ist.
Dann ist auch $f_*$ surjektiv.

Es gilt $K'(a') = \im \ev_{a'}$, denn $a'$ ist algebraisch über $K'$, da $a'$ eine Nullstelle des Polynoms $p'(t) \in K'[t]$ ist (es gilt $p'(t) = f_*(p(t)) \neq 0$, da $p(t) \neq 0$ gilt, und $f_*$ wegen der Injektvität von $f$ ebenfalls injektiv ist).
Aus der Kommutativität des Diagramm~\eqref{equation: big diagram} und der Surjektvitäten von $\ev_a$ und $f_*$ folgt somit, dass
\[
    \im \varphi
  = \im (\varphi \circ \ev_a)
  = \im (\ev_{a'}' \circ f_*)
  = \im \ev_{a'}
  = K'(a')
  = \varphi(K)(\varphi(a)) \,.
\]





