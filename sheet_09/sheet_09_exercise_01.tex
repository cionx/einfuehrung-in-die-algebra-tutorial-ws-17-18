\section{}





\subsection{}

Die Aussage ist \emph{wahr}:

Da $M/K$ algebraisch ist, gibt es für jedes $a \in M$ ein Polynom $p(t) \in K[t]$ mit $p(t) \neq 0$ und $p(a) = 0$.
Dann gilt auch $p(t) \in L[t]$, weshalb $a$ algebrasich über $M$ ist.
Das zeigt, dass auch $M/L$ algebraisch ist.

Jedes Element $a \in M$ ist algebraisch über $K$, da $M/K$ algebraisch ist.
Insbesondere ist jedes $a \in L$ algebraisch über $K$, und somit $L/K$ algebraisch.





\subsection{}

Die Aussage ist \emph{wahr}, denn nach der Gradformel gilt
\[
    [M : K]
  = [M : L][L : K] \,,
\]
und nach Annahme gilt $[M : L], [L : K] < \infty$





\subsection{}

Die Aussage ist \emph{wahr}:
Per Aufgabenstellung ist $L$ ein algebraischer Abschluss von $\Real$.
Außerdem ist $\Complex$ ein algebraischer Abschluss von $\Real$.
Es gibt deshalb einen $\Real$-Isomorphismus $L \to \Complex$.
Insbesondere gilt
\[
    [L : \Real]
  = \dim_\Real L
  = \dim_\Real \Complex
  = 2 \,.
\]





\subsection{}

Die Aussage ist \emph{falsch}:
Es seien $\alpha \define e^{2 \pi i / 5}$ und $\beta \define \alpha + \alpha^{-1}$.

\begin{claim*}
  Die Zahl $\beta$ erfüllt das Polynom $p(t) \define t^2 + t - 1$.
\end{claim*}

\begin{proof}
  Es gilt $\Phi_5(\alpha) = 0$, da $\alpha$ eine primitive $5$-te Einheitswurzel ist.
  Also gilt
  \begin{align*}
        0
     =  \alpha^4 + \alpha^3 + \alpha^2 + \alpha + 1
    &=  \alpha^{-1} + \alpha^{-2} + \alpha^2 + \alpha + 1
    \\
    &=  \alpha^{-1} + (\alpha^{-1} + \alpha)^2 - 2 + \alpha + 1
     =  \beta^2 + \beta - 1 \,.
  \qedhere
  \end{align*}
\end{proof}

Es gibt mehrere Möglichkeiten, einzusehen, dass $\Rational \subsetneq \Rational(\beta)$ gilt:

\begin{itemize}
  \item
    Das Polynom $p(t)$ hat keine rationale Nullstelle, denn die beiden komplexen Nullstellen sind $(-1 \pm \sqrt{5})/2$.
    Somit gilt inbesondere $\beta \notin \Rational$.
    (Man kann hier bereits erkennen, dass $\beta = (-1 + \sqrt{5})/2$ gilt.)
    (Da $p(t)$ qaudatrisch ist, ergibt sich hieraus auch, dass $p(t)$ irreduzibel ist.)
  \item
    Das Polynom $p(t) = t^2 + t -1 \in \Integer[t]$ ist normiert und somit primitiv.
    Das Polynom $\induced{p}(t) = t^2 + t + 1 \in (\Integer/2)[t]$ ist irreduzibel, da es quadratisch ist und keine Nullstellen besitzt (da $\induced{p}(0) = 1 = \induced{p}(1)$ gilt).
    Nach dem Reduktionskriterium ist $p(t)$ somit irreduzibel.
    Somit ist $p(t)$ das Minimalpolynom von $\beta$ über $\Rational$, weshalb $[\Rational(\beta) : \Rational] = \deg p = 2$ gilt.
    Inbesondere gilt $\beta \notin \Rational$.
  \item
    Das Minimalpolynom von $\alpha$ über $\Rational$ ist $\Phi_5(t)$ (die Irreduziblität ist aus der Vorlesung bekannt), weshalb $[\Rational(\alpha) : \Rational] = \deg \Phi_5 = 4$ gilt.
    Deshalb ist die Familie $(1, \alpha, \alpha^2, \alpha^3)$ eine $\Rational$-Basis von $\Rational(\alpha)$.
    In dieser Basis gilt
    \[
        \beta
      = \alpha + \alpha^{-1}
      = \alpha + \alpha^4
      = \alpha -\alpha^3 - \alpha^2 - \alpha - 1
      = -\alpha^3 - \alpha^2 - 1 \,.
    \]
    Inbesondere gilt somit $\beta \notin \generated{ 1 }_\Rational = \Rational$.
\end{itemize}

Es gilt $\Rational(\beta) \subseteq \Rational(\alpha)$, da $\beta = \alpha + \alpha^{-1} \in \Rational(\alpha)$ gilt.
Dass bereits $\Rational(\beta) \subsetneq \Rational(\alpha)$ gilt, ergibt sich ebenfalls auf verschiedene Weisen:
\begin{itemize}
  \item
    Nach den ersten beiden der obigen Argumentationen ist $p(t)$ irreduzibel über $\Rational$, und somit das Minimalpolynom von $\beta$ über $\Rational$.
    Also gilt $[\Rational(\beta) : \Rational] = \deg p(t) = 2$.
    Nach der letzten der obigen Argumentation gilt $[\Rational(\alpha) : \Rational] = 4$.
    Es gilt somit
    \[
        [\Rational(\alpha) : \Rational]
      = 4
      > 2
      = [\Rational(\beta) : \Rational]
    \]
    und deshalb $\Rational(\alpha) \supsetneq \Rational(\beta)$.
  \item
    Es gilt $\Rational(\beta) \subseteq \Real$, da $\beta = \alpha + \alpha^{-1} = \alpha + \conjugate{\alpha} \in \Real$ gilt (sowie $\Rational \subseteq \Real$).
    Es gilt aber auch $\alpha \notin \Real$, und somit $\alpha \notin \Rational(\beta)$.
    Also gilt $\Rational(\beta) \subsetneq \Rational(\alpha)$.
\end{itemize}

Insgesamt ergibt sich, dass $\Rational(\beta)$ ein echtere Zwischenkörper $\Rational \subsetneq \Rational(\beta) \subsetneq \Rational(\alpha)$ ist.

\begin{remark}
  Wir werden im weiteren Verlauf der Vorlesung sehen, dass die Körpererweiterung $\Rational(\alpha)/\Rational$ \emph{galoissch} ist, und deshalb die Zwischenkörper $\Rational \subseteq K \subseteq \Rational(\alpha)$ auf bijektive Weise den Untergruppen der \emph{Galoisgruppe} $\Aut{\Rational(\alpha)/\Rational}$ entsprechen.
  Dabei gilt $\Aut{\Rational(\alpha)/\Rational} \cong \Integer/4$.
  Da $\Integer/4$ genau drei Untergruppen hat, nämlich $0$, $2\Integer/4$ und $\Integer/4$, hat die Erweiterung $\Rational(\alpha)/\Rational$ genau drei Zwischenkörper, nämlich $\Rational(\alpha)$, $\Rational(\beta)$ und $\Rational$.
\end{remark}





\subsection{}

Die Aussage ist \emph{wahr}:
Das Minimalpolynom von $\alpha \define \sqrt[p]{q}$ über $\Rational$ ist $p(t) \define t^p - q$, wobei sich die Irreduziblität aus dem Eisenstein-Kriterium ergibt.
Folglich ist der Grad $[\Rational(\alpha) : \Rational] = p$ prim.
Für jeden Zwischenkörper $\Rational \subseteq K \subseteq \Rational(\alpha)$ gilt nun
\begin{gather*}
    p
  = [\Rational(\alpha) : \Rational]
  = [\Rational(\alpha) : K] [K : \Rational]
\shortintertext{und somit}
  [\Rational(\alpha) : K] = 1
  \quad\text{oder}\quad
  [K : \Rational] = 1 \,,
\shortintertext{und somit}
  K = \Rational(\alpha)
  \quad\text{oder}\quad
  K = \Rational \,.
\end{gather*}



