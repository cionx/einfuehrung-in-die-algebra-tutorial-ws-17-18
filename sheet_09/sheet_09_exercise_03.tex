\section{}





\addtocounter{subsection}{1}





\subsection{}

Im Tutorium haben wir genutzt, dass
\[
    K^\alg
  = \{ 
      x \in K
      \suchthat
      \text{$x$ ist algebraisch über $K$}
    \}
\]
ein Unterkörper von $K$ ist, und wegen $L_1, L_2 \subseteq K$ damit auch $L_1 L_2 \subseteq K$ gilt.
Es gibt auch noch alternative Argumentationsmöglichkeiten:

\begin{itemize}
  \item
    Jedes $x \in L_2$ ist nach Annahme algebraisch über $K$, und somit auch algebraisch über $L_1$.
    Also ist die Körpererweiterung $L_1(L_2)/L_1$ algebraisch, also $L_1 L_2 / L_1$ algebraisch.
    Nach Annahme ist auch $L_1/K$ algebraisch.
    Wegen der Transitivität von Algebraizität ist damit auch $L_1 L_2 / K$ algebraisch.
  \item
    Es seien $L_1 = K(\alpha_i \suchthat i \in I)$ und $L_2 = K(\beta_j \suchthat j \in J)$.
    Alle $\alpha_i$ und $\beta_j$ sind algebraisch über $K$, da $L_1/K$ und $L_2/K$ algebraisch sind.
    Dann gilt
    \[
        L_1 L_2
      = K(\{\alpha_i \suchthat i \in I\} \cup \{\beta_j \suchthat j \in J\}) \,,
    \]
    weshalb $L_1 L_2$ von Elementen erzeugt wird, die algebraisch über $K$ sind.
    Also ist auch $L_1 L_2 / K$ algebraisch.
  \item
    Da $L_1/K$ und $L_2/K$ algebraisch sind, lässt sich der Körper $L_1 L_2$ auch explizit beschreiben:
    Es sei
    \[
              L
      \define \left\{
                \sum_{i=1}^n x_i y_i
              \suchthat*
                \begin{array}{c}
                  n \geq 0,
                \\
                  x_i \in L_1,
                  y_i \in L_2
                \end{array}
              \right\} \,.
    \]
    Dann ist $L$ der von $L_1$ und $L_2$ erzeugte Unterring von $L$:
    Es gilt $1 = 1 \cdot 1 \in L$.
    Für alle $z_1, z_2 \in L$ mit $z_1 = \sum_{i=1}^n x_i y_i$ und $z_2 = \sum_{i=n+1}^m x_i y_i$ gilt dann auch $z_1 + z_2 = \sum_{i=1}^m x_i y_i \in L$.
    Für alle $z_1, z_2 \in L$ mit $z_1 = \sum_{i=1}^n x_i y_i$ und $z_2 = \sum_{j=1}^m x'_j y'_j$ gilt auch
    \[
          z_1 z_2
      =   \left(
            \sum_{i=1}^n x_i y_i
          \right)
          \left(
            \sum_{j=1}^m x'_j y'_j
          \right)
      =   \sum_{i=1}^n \sum_{j=1}^m \underbrace{(x_i x'_j)}_{\in L_1} \underbrace{(y_i y'_j)}_{\in L_2}
      \in L \,.
    \]
    Nach Annahme sind alle $x \in L_1$ und $y \in L_2$ algebraisch über $K$, weshalb auch $L$ algebraisch über $K$ ist.
    Außerdem ist $L$ als Unterring von $M$ ein Integritätsbereich.
    Nach Aufgabe 2 (c) von Zettel 8 ist $L$ somit bereits ein Körper.
    Also ist $L$ bereits der von $L_1$ und $L_2$ erzeugte Unterkörper, also $L = L_1 L_2$.
    Insbesondere sind alle Elemente von $L_1 L_2$ algebraisch über $K$.
    
    \begin{remark}
      Für beliebige, nicht notwendigerweise algebraische Körpererweiterungen $L_1/K$ und $L_2/K$ gilt
      \begin{align*}
            L_1 L_2
        &=  \left\{
              \frac{x}{x'}
            \suchthat*
              x, x' \in L,
              x' \neq 0
            \right\}
        \\
        &=  \left\{
              \frac{\sum_{i=1}^n x_i y_i}{\sum_{j=1}^m x'_j y'_j}
            \suchthat*
              \begin{array}{c}
                n, m \geq 0,
              \\
                x_i, x'_i \in L_1,
                y_j, y'_j \in L_2,
              \\
                \sum_{j=1}^m x'_j y'_j \neq 0
              \end{array}
            \right\} \,.
      \end{align*}
      Dies entspricht dem Quotientenkörper $\Quot{L}$ sofern man diesen in $M$ einbettet.
    \end{remark}
    \begin{example}
      Es sei $K(X,Y)$ der Funktionenkörper in zwei Variablen $X$ und $Y$, und es seien $K(X), K(Y) \subseteq K(X,Y)$ die Funktionenkörper in jeweils einer Variable, aufgefasst als Unterkörper von $K(X,Y)$.
      Dann gilt $K(X) K(Y) = K(X,Y)$.
      Aber
      \begin{align*}
                    \generated{ K(X) \cup K(Y) }_{\text{Ring}}
        =           \left\{
                      \frac{f(X,Y)}{g(X)h(Y)}
                    \suchthat*
                      \begin{array}{c}
                      f(X,Y) \in K[X,Y],
                      \\
                      g(X) \in K[X],
                      h(Y) \in K[Y]
                      \end{array}
                    \right\}
        \subsetneq  K(X,Y) \,.
      \end{align*}
      So gilt etwa $1/(1 + XY) \notin \generated{ K(X) \cup K(Y) }_{\text{Ring}}$.
    \end{example}
\end{itemize}





\subsection{}

Wir haben im Tutorium bereits einen Beweis gesehen, und geben hier noch einen weiteren, indem wir konkret ein $K$-Erzeugendensystem von $L_1 L_2$ aus $K$-Basen von $L_1$ und $L_2$ konstruieren.
Hierfür seien $x_1, \dotsc, x_n \in L_1$ und $y_1, \dotsc, y_m \in L_2$ jeweils endliche $K$-Basen;
da $L_1/K$ und $L_2/K$ endlich sind, gibt es diese.

\begin{claim*}
  Die Produkte $x_i y_j \in L_1 L_2$ bilden ein $K$-Erzeugendensystem von $L_1 L_2$.
\end{claim*}

Aus dieser Behauptung erhalten wir dann direkt, dass
\[
        [L_1 L_2 : K]
  =     \dim_K ( L_1 L_2 )
  \leq  n m
  =     (\dim_K L_1)(\dim_K L_2)
  =     [L_1 : K][L_2 : K] \,.
\]

\begin{proof}[Beweis der Behauptung]
  Wir geben zwei Beweise für die Behauptung an:
  \begin{itemize}
    \item
      Die Erweiterungen $L_1/K$ und $L_2/K$ sind algebraisch, da sie endlich sind.
      Wie bereits oben gesehen, gilt deshalb
      \[
          L_1 L_2
        = \left\{
            \sum_i \tilde{x}_i \tilde{y}_i
          \suchthat*
            \begin{array}{c}
              n \geq 0,
            \\
              \tilde{x}_i \in L_1,
              \tilde{y}_i \in L_2
            \end{array}
          \right\} \,.
      \]
      Dabei lässt sich jedes $\tilde{x}_i$ als $K$-Linearkombination der $x_j$ schreiben, und jedes $\tilde{y}_i$ als Linearkombination der $y_j$.
      Damit ist dann $\sum_i \tilde{x}_i \tilde{y}_i$ eine $K$-Linearkombination der $x_{j_1} y_{j_2}$.
      
    \item
      Da $x_1, \dotsc, x_n \in L_1$ und $y_1, \dotsc, y_m \in L_2$ jeweils $K$-Erzeugendensysteme sind, gelten insbesondere
      \[
          L_1
        = K(x_1, \dotsc, x_n)
        \quad\text{und}\quad
          L_2
        = K(y_1, \dotsc, y_m) \,.
      \]
      Damit gilt dann auch
      \[
          L_1 L_2
        = K(x_1, \dotsc, x_n, y_1, \dotsc, y_m) \,.
      \]
      Da die $x_i$ und $y_j$ algebraisch über $K$ sind (da $L_1/K$ und $L_2/K$ als endliche Körpererweiterungen inbesondere algebraisch sind), gilt dabei bereits
      \[
          L_1 L_2
        = K(x_1, \dotsc, x_n, y_1, \dotsc, y_m)
        = K[x_1, \dotsc, x_n, y_1, \dotsc, y_m] \,.
      \]
      Also wird $L_1 L_2$ als $K$-Vektorraum von den Monomen
      \[
        x_1^{\alpha_1} \dotsm x_n^{\alpha_n} y_1^{\beta_1} \dotsm y_m^{\beta_m}
      \]
      erzeugt.
      Dabei gilt $x_1^{\alpha_1} \dotsm x_n^{\alpha_n} \in L_1$, weshalb $x_1^{\alpha_1} \dotsm x_n^{\alpha_n}$ eine $K$-Linear\-kombi\-nation der $x_i$ ist;
      analog ergibt sich auch, dass $y_1^{\beta_1} \dotsm y_m^{\beta_m}$ eine $K$-Linear\-kombi\-nation der $y_j$ ist.
      Damit ist das Monom $x_1^{\alpha_1} \dotsm x_n^{\alpha_n} y_1^{\beta_1} \dotsm y_m^{\beta_m}$ insgesamt eine $K$-Linear\-kombi\-nation der $x_i y_j$.
      Da dies für jedes der Monome gilt, und $L_1 L_2$ diese Monome als $K$-Erzeugendensystem hat, sind die $x_i y_j$ bereits ein $K$-Erzeugendensystem von $L_1 L_2$.
  \end{itemize}
\end{proof}












