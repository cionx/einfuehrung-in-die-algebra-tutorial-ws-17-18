\section{}

Für ein Elemente $x \in R$ bezeichnen wir im Folgenden eine Zerlegung $x = \varepsilon p_1 \dotsm p_n$ in eine Einheit $\varepsilon \in R^\times$ und irreduzible Elemente $p_1, \dotsc, p_n \in R$ als eine \emph{Primfaktorzerlegung} von $x$.
Man beachte, dass a priori nicht gefordert wird, dass die $p_i$ prim sind.




\subsection{}

Wir formulieren zunächst einige (intuitive) Aussagen über Primfaktorzerlegungen in faktoriellen Ringen:

\begin{lemma}
  \label{lemma: extension of prime factorizations}
  Es seien $x, y \in R$ mit $x, y \neq 0$, so dass $x$ ein Teiler von $y$ ist.
  Dann lässt sich jede Primfaktorzerlegung $x = \varepsilon p_1 \dotsm p_n$ von $x$ zu einer Primfaktorzerlegung $y = \varepsilon' p_1 \dotsm p_n p_{n+1} \dotsm p_m$ von $y$ ergänzen.
\end{lemma}

\begin{proof}
  Es gibt $z \in R$ mit $xz = y$, und es gilt $z \neq 0$, da $y \neq 0$ gilt.
  Also besitzt $z$ eine Primfaktorzerlegung $z = \delta p_{n+1} \dotsm p_m$.
  Dann gilt
  \[
      y
    = xz
    = \varepsilon \delta p_1 \dotsm p_n p_{n+1} \dotsm p_m \,,
  \]
  und die Aussage ergibt sich mit $\varepsilon' \coloneqq \varepsilon \delta$.
\end{proof}

Für $x \in R$, $x \neq 0$ mit Primfaktorzerlegung $x = \varepsilon p_1 \dotsm p_n$ bezeichnen wir mit $\nu(x) \coloneqq n$ die Anzahl der insgesamt vorkommenden Primfaktoren (inklusive Vielfachheit).
Die Zahl $\nu(x)$ ist wohldefiniert, da die Primfaktorzerlegung von $x$ bis auf Permutation und Einheiten eindeutig ist.

\begin{lemma}
  \label{lemma: number of prime factors}
  Es seien $x, y \in R$ mit $x, y \neq 0$.
  \begin{enumerate}
    \item
      Es gilt genau dann $\nu(x) = 0$, wenn $x$ eine Einheit ist.
    \item
      Es gilt $\nu(x y) = \nu(x) + \nu(y)$.
    \item
      Ist $x$ ein Teiler von $y$, so gilt $\nu(x) \leq \nu(y)$.
    \item
      Ist $x$ ein echter Teiler von $y$, also $(y) \subsetneq (x)$, so gilt $\nu(x) < \nu(y)$.
  \end{enumerate}
\end{lemma}

\begin{proof}\leavevmode
  \begin{enumerate}
    \item
      In der Primfaktorzerlegung $x = \varepsilon p_1 \dotsm p_n$ gilt $n = 0$ und somit $x = \varepsilon \in R^\times$.
      Falls $x$ eine Einheit ist, so ist für die Einheit $\varepsilon \coloneqq x$ die Zerlegung $x = \varepsilon$ bereits eine Primfaktorzerlegung.
    \item
      Da $R$ ein Integritätsbereich ist, gilt auch $xy \neq 0$, weshalb $\nu(xy)$ definiert ist.
      Es seien $x = \varepsilon p_1 \dotsm p_n$ und $y = \delta q_1 \dotsm q_m$ Primfaktorzerlegungen.
      Dann ist
      \[
          xy
        = (\varepsilon \delta) p_1 \dotsm p_n q_1 \dotsm q_m
      \]
      eine Primfaktorzerlegung von $xy$ und somit
      \[
          \nu(xy)
        = n + m
        = \nu(x) + \nu(y) \,.
      \]
    \item
     Es gibt $z \in R$ mit $y = xz$.
     Es gilt $z \neq 0$, da $y \neq 0$ gilt, weshalb $\nu(z)$ definiert ist.
     Somit gilt
     \[
            \nu(y)
       =    \nu(xz)
       =    \nu(x) + \nu(z)
       \geq \nu(x) \,.
     \]
    \item
      In der obigen Situation gilt andernfalls $\nu(z) = 0$, weshalb $z$ dann eine Einheit ist.
      Deshalb gilt dann
      \[
          (y)
        = (xz)
        = (x) \,.
        \qedhere
      \]
  \end{enumerate}
\end{proof}



\subsubsection{}

Es sei $p \in R$ irreduzibel, und es seien $x, y \in R$ mit $p \divides xy$.
Gilt $x = 0$ oder $y = 0$, so gilt $p \divides x$ oder $p \divides y$.

Andernfalls gibt es Primfaktorzerlegungen $x = \delta q_1 \dotsm q_n$ und $y = \delta' q'_1 \dotsm q'_m$.
Dann ist
\begin{equation}
  \label{equation: first factorization}
    xy
  = (\delta \delta') q_1 \dotsm q_n q'_1 \dotsm q'_m
\end{equation}
eine Primfaktorzerlegung von $xy$.
Da $p$ irreduzibel ist und $p \divides xy$ gilt, lässt sich $p$ nach Lemma~\ref{lemma: extension of prime factorizations} zu einer Primfaktorzerlegung
\begin{equation}
  \label{equation: second factorization}
    xy
  = \varepsilon p p_2 \dotsm p_r
\end{equation}
ergänzen.
Da $R$ faktoriell ist, sind die beiden Primfaktorzerlegungen \eqref{equation: first factorization} und \eqref{equation: second factorization} eindeutig bis auf Einheiten und Permutation.
Es gilt deshalb $p \divides q_i$ oder $p \divides q'_i$ für passendes $i$, und somit $p \divides x$ oder $p \divides y$.



\subsubsection{}

Wir nehmen an, dass nicht jede aufsteigende Kette von Hauptidealen stabilisieren würde.
Dann gibt es eine unendliche, echt aufsteigende Kette von Hauptidealen
\[
              (a_0)
  \subsetneq  (a_1)
  \subsetneq  (a_2)
  \subsetneq  (a_3)
  \subsetneq  (a_4)
  \subsetneq  \dotsb
\]
Dann gilt $a_i \neq 0$ für alle $i \geq 1$ (denn sonst wäre $(a_i) = 0$ für ein solches $i$, und dann würde $(a_i) = \dotsb = (a_0) = 0$ gelten).
Nach Lemma~\ref{lemma: number of prime factors} erhalten wir eine unendliche absteigende Kette
\[
    \nu(a_1)
  > \nu(a_2)
  > \nu(a_3)
  > \nu(a_4)
  > \dotsb
\]
Dies ist aber nicht möglich.





\subsection{}

Wir müssen zeigen, dass es für jedes Element $x \in R$ mit $x \neq 0$ eine Primfaktorzerlegung
\[
  x = \varepsilon p_1 \dotsm p_n
\]
gibt, und dass diese eindeutig bis auf Einheiten und Permutation ist.

\subsubsection*{Existenz}

\begin{lemma}
  \label{lemma: proper divisors give bigger ideals}
  Es sei $x \in R$, und es sei $x = yz$ ein Zerlegung mit $z \notin R^\times$.
  Dann gilt $(x) \subsetneq (y)$.
\end{lemma}

\begin{proof}
  Es gilt $y \divides x$ und somit $(x) \subseteq (y)$.
  Wäre $(x) = (y)$, so gebe es ein $z' \in R$ mit $y = xz'$.
  Dann wäre $x = yz = xzz'$ und somit $1 = zz'$, da $R$ ein Integritätsbereich ist.
  Dann wäre $z$ eine Einheit mit $z^{-1} = z'$, im Widerspruch zu $z \notin R^\times$.
\end{proof}

Wir nehmen an, dass es ein Element $x \in R$ mit $x \neq 0$ gibt, dass keine Primfaktorzerlegung besitzt.
Dann ist $x$ inbesondere keine Einheit und auch nicht irreduzibel.
Es gibt deshalb nicht-Einheiten $y, z \in R$ mit $x = y z$;
dabei gelten $y, z \neq 0$ da $x \neq 0$ gilt.
Würden $x$ und $z$ beide eine Primfaktorzerlegung besitzten, so würden sich diese zu einer Primfaktorzerlegung von $x$ kombinieren lassen.
Also hat $x$ oder $y$ keine Primfaktorzerlegung;
wir können o.B.d.A. davon ausgehen, dass $y$ keine hat.
Da $z$ keine Einheit ist, gilt $(x) \subsetneq (y)$ nach Lemma~\ref{lemma: proper divisors give bigger ideals}.

Wir setzen $a_0 \coloneqq x$ und $a_1 \coloneqq y$.
Durch induktives Wiederholen der obigen Argumentation erhalten wir eine unendliche aufsteigende Kette von Hauptidealen
\[
              (a_0)
  \subsetneq  (a_1)
  \subsetneq  (a_2)
  \subsetneq  (a_3)
  \subsetneq  \dotsb
\]
Dies steht im Widerspruch zur Annahme (ii).



\subsubsection*{Eindeutigkeit}

Für zwei Primfaktorzerlegungen
\[
    x
  = \varepsilon p_1 \dotsm p_n
  = \delta q_1 \dotsm q_m
\]
zeigen wir die gewünschte Eindeutigkeit per Induktion über $n$:

Gilt $n = 0$, so ist $x = \varepsilon \in R^\times$ eine Einheit.
Dann gilt $q_j \divides x \divides 1$ für alle $j$, weshalb jedes $q_j$ eine Einheit ist.
Irreduzible Elemente sind aber per Definition keine Einheiten, weshalb $m = 0$ gelten muss.
Dann ist also $x = \varepsilon = \delta$, und die beiden Zerlegungen stimmen überein.

Es sei nun $n > 0$.
Nach Annahme (i) ist $p_1$ prim.
Aus
\[
            p_1
  \divides  x
  =         \delta q_1 \dotsm q_m
\]
folgt damit, dass $p_1 \divides \delta$ gilt, oder dass $p_1 \divides q_j$ für ein $j$ gilt.
Würde $p_1 \divides \delta$ gelten, so wäre $p_1$ eine Einheit, im Widerspruch zur Irreduziblität von $p_1$.
Also gilt $p_1 \divides q_j$ für ein $j$;
wir können o.B.d.A.\ davon ausgehen, dass $p_1 \divides q_1$ gilt.
Es gibt also $\delta' \in R$ mit $q_1 = p_1 \delta'$.
Da $q_1$ irreduzibel ist, folgt dabei, dass bereits $p_1$ oder $\delta'$ eine Einheit ist;
$p_1$ ist wegen Irreduziblität keine Einheit, so dass $\delta'$ eine Einheit ist.
Also sind $p_1$ und $q_1$ bis auf die Einheit $\delta'$ gleich.

Es gilt nun
\begin{equation}
  \label{equation: before induction}
    x
  = \varepsilon p_1 \dotsm p_n
  = \delta q_1 \dotsm q_m
  = \delta \delta' p_1 q_2 \dotsm q_m \,.
\end{equation}
Da $R$ ein Integritätsbereich ist, können wir die obige Gleichung durch $p_1 \neq 0$ teilen, und erhalten, dass bereits
\begin{equation}
  \label{equation: induction}
    \varepsilon p_2 \dotsm p_n
  = (\delta \delta') q_2 \dotsm m
\end{equation}
gilt.
Nach Induktionsvoraussetzung sind beide Seiten von \eqref{equation: induction} bis auf Einheiten und Permutation gleich.
Damit sind in \eqref{equation: before induction} bereits beide Zerlegungen bis auf Einheiten und Permutation gleich, da auch $p_1$ und $q_1$ bis auf Einheit gleich ist.

% \begin{remark}
%   Ein kommutativer Ring $R$ heißt \emph{noethersch}, wenn jede aufsteigende Kette
%   \[
%             I_0
%     \ideal  I_1
%     \ideal  I_2
%     \ideal  I_3
%     \ideal \dotsb
%   \]
%   von Idealen $I_j \ideal R$ stabilisiert.
%   Wir haben soeben insbesondere gezeigt, dass in einem noetherschen Ring $R$ jedes Element $x \in R$ eine Primfaktorzerlegung besitzt.
%   Diese erfüllt im Allgemeinen allerdings
% \end{remark}




