\section{}





\subsection{}

Wir bemerken zunächst einige (intuitive) Aussagen über Primfaktorzerlegungen in faktoriellen Ringen:

\begin{lemma}
  \label{lemma: extension of prime factorizations}
  Es seien $x, y \in R$ mit $x, y \neq 0$, so dass $x$ ein Teiler von $y$ ist.
  Dann lässt sich jede Primfaktorzerlegung $x = \varepsilon p_1 \dotsm p_n$ von $x$ zu einer Primfaktorzerlegung $y = \varepsilon' p_1 \dotsm p_n p_{n+1} \dotsm p_m$ von $y$ ergänzen.
\end{lemma}

\begin{proof}
  Es gibt $z \in R$ mit $xz = y$, und es gilt $z \neq 0$, da $y \neq 0$ gilt.
  Also besitzt $z$ eine Primfaktorzerlegung $z = \delta p_{n+1} \dotsm p_m$.
  Dann gilt
  \[
      y
    = xz
    = \varepsilon \delta p_1 \dotsm p_n p_{n+1} \dotsm p_m \,,
  \]
  und die Aussage ergibt sich mit $\varepsilon' \coloneqq \varepsilon \delta$.
\end{proof}

Für $x \in R$, $x \neq 0$ mit Primfaktorzerlegung $x = \varepsilon p_1 \dotsm p_n$ bezeichnen wir mit $\nu(x) \coloneqq n$ die Anzahl der vorkommenden Primfaktoren (inklusive Vielfachheit).
Die Zahl $\nu(x)$ ist wohldefiniert, da die Primfaktorzerlegung bis Einheiten und Permutation der Faktoren eindeutig ist.

\begin{lemma}
  \label{lemma: number of prime factors}
  Es seien $x, y \in R$ mit $x, y \neq 0$.
  \begin{enumerate}
    \item
      Es gilt genau dann $\nu(x) = 0$, wenn $x$ eine Einheit ist.
    \item
      Es gilt $\nu(x y) = \nu(x) + \nu(y)$.
    \item
      Ist $x$ ein Teiler von $y$, so gilt $\nu(x) \leq \nu(y)$.
    \item
      Ist $x$ ein echter Teiler von $y$, also $(y) \subsetneq (x)$, so gilt $\nu(x) < \nu(y)$.
  \end{enumerate}
\end{lemma}

\begin{proof}\leavevmode
  \begin{enumerate}
    \item
      In der Primfaktorzerlegung $x = \varepsilon p_1 \dotsm p_n$ gilt $n = 0$ und somit $x = \varepsilon \in R^\times$.
    \item
      Da $R$ ein Integritätsbereich ist, gilt auch $xy \neq 0$.
      Es seien $x = \varepsilon p_1 \dotsm p_n$ und $y = \delta q_1 \dotsm q_m$ Primfaktorzerlegungen.
      Dann 
      \[
          xy
        = (\varepsilon \delta) p_1 \dotsm p_n q_1 \dotsm q_m
      \]
      eine Primfaktorzerlegung von $xy$ und somit
      \[
          \nu(xy)
        = n + m
        = \nu(x) + \nu(y) \,.
      \]
    \item
     Es gibt $z \in R$ mit $y = xz$.
     Es gilt $z \neq 0$, da $y \neq 0$ gilt, weshalb $\nu(z)$ definiert ist.
     Somit gilt
     \[
            \nu(y)
       =    \nu(xz)
       =    \nu(x) + \nu(z)
       \leq \nu(x) \,.
     \]
    \item
      Ansonsten gilt in der obigen Situation $\nu(z) = 0$, weshalb $z$ dann eine Einheit ist.
      Deshalb gilt dann
      \[
          (y)
        = (xz)
        = (x) \,.
        \qedhere
      \]
  \end{enumerate}
\end{proof}



\subsubsection{}

Es sei $p \in R$ irreduzibel, und es seien $x, y \in R$ mit $p \divides xy$.
Gilt $x = 0$ oder $y = 0$, so gilt $p \divides x$ oder $p \divides y$.

Ansonsten gibt es Primfaktorzerlegungen $x = \delta q_1 \dotsm q_n$ und $y = \delta' q'_1 \dotsm q'_m$ Primfaktorzerlegungen.
Dann ist
\begin{equation}
  \label{equation: first factorization}
    xy
  = (\delta \delta') q_1 \dotsm q_n q'_1 \dotsm q'_m
\end{equation}
eine Primfaktorzerlegung von $xy$.
Da $p$ irreduzibel ist und $p \divides xy$ gilt, lässt sich $p$ nach Lemma~\ref{lemma: extension of prime factorizations} zu einer Primfaktorzerlegung
\begin{equation}
  \label{equation: second factorization}
    xy
  = \varepsilon p p_2 \dotsm p_r
\end{equation}
ergänzen.
Da $R$ faktoriell ist, sind die beiden Primfaktorzerlegungen \eqref{equation: first factorization} und \eqref{equation: second factorization} eindeutig bis auf Einheiten und Permutation.
Es gilt deshalb $p \divides q_i$ oder $p \divides q'_i$ für passendes $i$, und somit $p \divides x$ oder $p \divides y$.



\subsubsection{}

Wir nehmen an, dass nicht jede aufsteigende Kette von Hauptidealen stabilisieren würde.
Dann gibt es eine unendliche echt aufsteigende Kette von Hauptidealen
\[
              (a_0)
  \subsetneq  (a_1)
  \subsetneq  (a_2)
  \subsetneq  (a_3)
  \subsetneq  (a_4)
  \subsetneq  \dotsb
\]
Dann gilt $a_i \neq 0$ für alle $i \geq 1$ (denn sonst wäre $(a_i) = 0$ für ein solches $i$, und damit bereits $(a_i) = \dotsb = (a_0) = 0$), und für jedes $i \geq 1$ ist $a_{i+1}$ ein echter Teiler von $a_i$.
Nach Lemma~\ref{lemma: number of prime factors} erhalten wir eine unendliche absteigende Kette
\[
    \nu(a_1)
  > \nu(a_2)
  > \nu(a_3)
  > \nu(a_4)
  > \dotsb
\]
Dies ist aber nicht möglich.








