\section{}

Es sei $R$ ein euklidischer Ring und es seien $a_1, \dotsc, a_n \in R$ paarweise teilerfremd.
Wir wollen erklären, wie man für $b_1, \dotsc, b_n \in R$ das System simultaner Kongruenzen
\[
  \left\{
    \begin{array}{rcll}
      x &\equiv& b_1      & \pmod{a_1}      \,, \\
      x &\equiv& b_2      & \pmod{a_2}      \,, \\
        &\vdots&          &                     \\
      x &\equiv& b_{n-1}  & \pmod{a_{n-1}}  \,, \\
      x &\equiv& b_n      & \pmod{a_n}
    \end{array}
  \right.
\]
mithilfe des euklidischen Algorithmus systematisch lösen kann:
Nach dem chinesischen Restsatz ist die Lösungsmenge von der Form
\[
  b + (a_1 \dotsm a_n) R,
\]
wobei $b$ ein konkrete Lösung ist.
Es gilt also, eine entsprechende Lösung zu finden.
Wir geben zwei Möglichkeiten an, wie dies mithilfe des euklidischen Algorithmus getan werden kann:
\begin{itemize}
  \item
    Die erste Möglichkeit besteht darin, je zwei Kongruenzen durch eine äquivalente einzelne Kongruenz zu ersetzen:
    \begin{itemize}
      \item
        Wir betrachten zunächst den Fall $n = 2$, also das System von Kongruenzen
        \begin{equation}
          \label{equation: two equations}
          \left\{
            \begin{array}{rcll}
              x &\equiv& b_1 & \pmod{a_1} \,, \\
              x &\equiv& b_2 & \pmod{a_2} \,.
            \end{array}
          \right.
        \end{equation}
        Da $a_1$ und $a_2$ teilerfremd sind, gibt es $c_1, c_2 \in R$ mit $1 = c_1 a_1 + c_2 a_2$.
        Dann gilt
        \begin{gather*}
          \left\{
            \begin{array}{rcll}
              c_2 a_2 &\equiv& 1  & \pmod{a_1}  \,, \\
              c_1 a_1 &\equiv& 1  & \pmod{a_2}  \,,
            \end{array}
          \right.
        \shortintertext{und somit}
          \left\{
            \begin{array}{rcll}
              b_1 c_2 a_2 &\equiv& b_1  & \pmod{a_1}  \,, \\
              b_2 c_1 a_1 &\equiv& b_2  & \pmod{a_2}  \,.
            \end{array}
          \right.
        \end{gather*}
        Also ist $b \coloneqq b_1 c_2 a_2 + b_2 c_1 a_1$ eine der gesuchten Lösungen.
        Nach dem chinesischen Restsatz ist \eqref{equation: two equations} somit äquivalent zu der einzelnen Kongruenz
        \[
          x \equiv b \pmod (a_1 a_2) \,.
        \]
      \item
        Im allgemeinen Fall
        \begin{equation}
          \label{equation: multiple equations}
          \left\{
            \begin{array}{rcll}
              x &\equiv&  b_1     & \pmod{a_1}      \,, \\
              x &\equiv&  b_2     & \pmod{a_2}      \,, \\
              x &\equiv&  b_3     & \pmod{a_3}      \,, \\
                &\vdots&          &                   \\
              x &\equiv&  b_{n-1} & \pmod{a_{n-1}}  \,, \\
              x &\equiv&  b_n     & \pmod{a_n}
            \end{array}
          \right.
        \end{equation}
        können wir nach mit bereits betrachteten Fall $n = 2$ eine Lösung $b$ des Systems
        \begin{equation}
          \label{equation: first two equations}
          \left\{
            \begin{array}{rclc}
              x &\equiv& b_1 & \pmod{a_1} \,, \\
              x &\equiv& b_2 & \pmod{a_2} \,,
            \end{array}
          \right.
        \end{equation}
        finden, und das System \eqref{equation: first two equations} anschließend durch die einzelne Kongruenz
        \[
          x \equiv b  \pmod{(a_1 a_2)}
        \]
        ersetzen.
        Wir können daher das System \eqref{equation: multiple equations} durch das äquivalente System
        \[
          \left\{
            \begin{array}{rcll}
              x &\equiv&  b       & \pmod{(a_1 a_2)}  \,, \\
              x &\equiv&  b_3     & \pmod{a_3}        \,, \\
                &\vdots&          &                       \\
              x &\equiv&  b_{n-1} & \pmod{a_{n-1}}    \,, \\
              x &\equiv&  b_n     & \pmod{a_n}
            \end{array}
          \right.
        \]
        ersetzen.
        Da $a_1, a_2, a_3, \dotsc, a_n$ teilerfremd sind, können wir dieses Vorgehen iterativ fortsetzen, um schließlich eine Lösung von \eqref{equation: multiple equations} zu erhalten.
    \end{itemize}
    
  \item
    Für alle $i = 1, \dotsc, n$ sind $a_i$ und $a_1 \dotsm a_{i-1} a_{i+1} \dotsm a_n$ teilerfremd, weshalb sich mit dem euklidischen Algorthimus $c^{(i)}_1, c^{(i)}_2 \in R$ bestimmen lassen, so dass
    \[
      c^{(i)}_1 a_i + c^{(i)}_2 a_1 \dotsm a_{i-1} a_{i+1} \dotsm a_n = 1
    \]
    gilt.
    Für das Element $k_i \define c^{(i)}_2 a_1 \dotsm a_{i-1} a_{i+1} \dotsm a_n$ gilt dann
    \[
      \left\{
        \begin{array}{rcll}
          k_i &\equiv&  0 & \pmod{a_1}  \,, \\
              &\vdots&    &               \\
          k_i &\equiv&  1 & \pmod{a_i}  \,, \\
              &\vdots&    &               \\
          k_i &\equiv&  0 & \pmod{a_n}  \,.
        \end{array}
      \right.
    \]
    Deshalb ist
    \[
              b
      \define b_1 k_1 + \dotsb + b_n k_n
    \]
    eine konkrete Lösung.
\end{itemize}





\addtocounter{subsection}{2}





\subsection{}

Es gilt das System
\[
  \left\{
    \begin{array}{rcll}
      x &\equiv& 4 & \pmod{7}   \,, \\
      x &\equiv& 7 & \pmod{12}
    \end{array}
  \right.
\]
zu lösen.
Da es nur zwei Kongruenzen gibt, ist die Rechnung für beide Möglichkeiten die gleiche:
Es gilt
\[
  1 = c_1 \cdot 7 + c_2 \cdot 12
\]
für die Koeffizienten $c_1 = -5$ und $c_2 = 3$.
Eine konkrete Lösung ist also durch
\[
    b
  = 4 \cdot (3 \cdot 12) + 7 \cdot ((-5) \cdot 7)
  = 144 - 245
  = -101
\]
gegeben.
Die Lösungsmeneg ist also durch
\[
    -101 + 84 \Integer
  =   67 + 84 \Integer
\]
gegeben.





\subsection{}

Es gilt das System
\[
  \left\{
    \begin{array}{rcrl}
      x &\equiv& 4  & \pmod{6}  \,, \\
      x &\equiv& 33 & \pmod{35} \,, \\
      x &\equiv& 10 & \pmod{11}
    \end{array}
  \right.
\]
zu lösen.
Wir geben drei mögliche Vorgehensweisen an:
\begin{itemize}
  \item
    Wir schreiben das System zunächst zu
    \[
      \left\{
        \begin{array}{rcrl}
          x &\equiv& -2  & \pmod{6} \,, \\
          x &\equiv& -2 & \pmod{35} \,, \\
          x &\equiv& -1 & \pmod{11}
        \end{array}
      \right.
    \]
    um.
    Für die ersten beiden Gleichungen ist $-2$ eine konkrete Lösung, weshalb wir das System durch
    \[
      \left\{
        \begin{array}{rcrl}
          x &\equiv& -2 & \pmod{210}  \,, \\
          x &\equiv& -1 & \pmod{11}
        \end{array}
      \right.
    \]
    ersetzen können.
    Es gilt
    \[
      1 = c_1 \cdot 210 + c_2 \cdot 11
    \]
    mit $c_1 = 1$ und $c_2 = -19$.
    Eine konkrete Lösung ist also durch
    \[
        b
      = -2 \cdot ((-19) \cdot 11) -1 \cdot (1 \cdot 210)
      = 208
    \]
    gegeben.
    Die Lösungsmenge ist somit
    \[
      208 + 2310 \Integer \,.
    \]
    
  \item
    Wir lösen zunächst das System
    \begin{equation}
      \label{equation: first two equations second example}
      \left\{
        \begin{array}{rcrl}
          x &\equiv& 4  & \pmod{6}  \,, \\
          x &\equiv& 33 & \pmod{35} \,.
        \end{array}
      \right.
    \end{equation}
    Es gilt
    \[
      1 = c_1 \cdot 6 + c_2 \cdot 35
    \]
    mit $c_1 = 6$ und $c_2 = -1$.
    Eine konkrete Lösung ist somit durch
    \[
        b'
      = 4 \cdot ((-1) \cdot 35) + 33 \cdot (6 \cdot 6)
      = 1048
    \]
    gegeben.
    Das System \eqref{equation: first two equations second example} können wir also durch die einzelne Kongruenz
    \[
      x \equiv 1048 \pmod{210}
    \]
    ersetzen, also die Kongruenz
    \[
      x \equiv 208 \pmod{210} \,.
    \]
    Wie bereits oben gesehen, ist
    \[
        1
      = c_1 \cdot 210 + c_2 \cdot 11
    \]
    für $c_1 = 1$ und $c_2 = -19$, und es ergibt sich nun die konkrete Lösung
    \[
        b
      = 208 \cdot ((-19) \cdot 11) + 10 \cdot (1 \cdot 210)
      = -41372 \,.
    \]
    Die Lösungsmenge ist somit
    \[
        -41372 + 2310 \Integer
      =   4828 + 2310 \Integer
      =    208 + 2310 \Integer \,.
    \]
    
  \item
    Es gelten
    \begin{gather*}
      \begin{array}{lclcl}
        1 & = & c_1 \cdot 6  & + & c_2 \cdot 35 \cdot 11  \\
        1 & = & d_1 \cdot 35 & + & d_2 \cdot 6  \cdot 11   \\
        1 & = & e_1 \cdot 11 & + & e_2 \cdot 6  \cdot 35
      \end{array}
    \intertext{für die Koeffizienten}
      c_1 = -64, c_2 =  1 \,, \qquad
      d_1 =  17, d_2 = -9 \,, \qquad
      e_1 = -19, e_2 =  1 \,.
    \end{gather*}
    Eine konkrete Lösung ist deshalb
    \[
        b
      =   4  \cdot c_2 \cdot 35 \cdot 11
        + 33 \cdot d_2 \cdot  6 \cdot 11
        + 10 \cdot e_2 \cdot  6 \cdot 35
      = -15962 \,.
    \]
    Die Lösungsmenge ist somit
    \[
        -15962 + 2310 \Integer
      =   7138 + 2310 \Integer
      =    208 + 2310 \Integer \,.
    \]
\end{itemize}

