\section{}

Es sei $R$ ein euklidischer Ring und es seien $a_1, \dotsc, a_n \in R$ paarweise teilerfremd.
Wir wollen erklären, wie man für $b_1, \dotsc, b_n \in R$ das System simultaner Kongruenzen
\[
  \left\{
    \begin{array}{rcll}
      x &\equiv& b_1 & \mod a_1 \,, \\
      x &\equiv& b_2 & \mod a_2 \,, \\
        &\vdots&     &              \\
      x &\equiv& b_n & \mod a_n \,,
    \end{array}
  \right.
\]
löst.
Nach dem chinesischen Restsatz ist die Lösungsmenge von der Form
\[
  b + (a_1 \dotsm a_n) R,
\]
wobei $b$ ein konkrete Lösung ist.
Es gilt also, eine entsprechende Lösung zu finden.
Wir geben zwei Möglichkeiten an, wie dies mithilfe des euklidischen Algorithmus getan werden kann:
\begin{itemize}
  \item
    \begin{itemize}
      \item
        Wir betrachten zunächst den Fall $n = 2$, also das System von Kongruenzen
        \begin{equation}
          \left\{
            \begin{array}{rclc}
              x &\equiv& b_1 & \mod a_1 \,, \\
              x &\equiv& b_2 & \mod a_2 \,.
            \end{array}
          \right.
        \end{equation}
        Da $a_1$ und $a_2$ teilerfremd sind, gibt es $c_1, c_2 \in R$ mit $1 = c_1 a_1 + c_2 a_2$.
        Dann gilt
        \begin{gather*}
          \left\{
            \begin{array}{rclc}
              c_2 a_2 &\equiv& 1  & \mod a_1 \,, \\
              c_1 a_1 &\equiv& 1  & \mod a_2 \,,
            \end{array}
          \right.
        \shortintertext{und somit}
          \left\{
            \begin{array}{rclc}
              b_1 c_2 a_2 &\equiv& b_1  & \mod a_1 \,, \\
              b_2 c_1 a_1 &\equiv& b_2  & \mod a_2 \,.
            \end{array}
          \right.
        \end{gather*}
        Also ist $b \coloneqq b_1 c_2 a_2 + b_2 c_1 a_1$ eine der gesuchten Lösungen.
        Nach dem chinesischen Restsatz ist \eqref{equation: two equations} somit äquivalent zu der einzelnen Kongruenz
        \[
          x \equiv b \mod (a_1 a_2) \,.
        \]
      \item
        Im allgemeinen Fall
        \begin{equation}
          \label{equation: multiple equations}
          \left\{
            \begin{array}{rcll}
              x &\equiv&  b_1 & \mod a_1  \,, \\
              x &\equiv&  b_2 & \mod a_2  \,, \\
              x &\equiv&  b_3 & \mod a_3  \,, \\
                &\vdots&      &               \\
              x &\equiv&  b_n & \mod a_n \,,
            \end{array}
          \right.
        \end{equation}
        können wir betrachten Fall eine Lösung $b$ der Systems
        \begin{equation}
          \label{equation: two equations}
          \left\{
            \begin{array}{rclc}
              x &\equiv& b_1 & \mod a_1 \,, \\
              x &\equiv& b_2 & \mod a_2 \,,
            \end{array}
          \right.
        \end{equation}
        finden.
        Nach dem chinesischen Restsatz ist dann das System \eqref{equation: two equations} äquivalent zu der einzelnen Kongruenz
        \[
          x \equiv b  \mod (a_1 a_2).
        \]
        Wir können daher das System \eqref{equation: multiple equations} durch das äquivalente System
        \[
          \left\{
            \begin{array}{rcll}
              x &\equiv&  b   & \mod (a_1 a_2)  \,, \\
              x &\equiv&  b_3 & \mod a_3        \,, \\
                &\vdots&      &                     \\
              x &\equiv&  b_n & \mod a_n \,,
            \end{array}
          \right.
        \]
        ersetzen.
        Da $a_1 a_2, a_3, \dotsc, a_n$ teilerfremd sind, können wir dieses Vorgehen fortsetzen, um eine Lösung von \eqref{equation: multiple equations} zu erhalten.
    \end{itemize}
    
  \item
    Für alle $i = 1, \dotsc, n$ sind $a_i$ und $a_1 \dotsm a_{i-1} a_{i+1} \dotsm a_n$ teilerfremd, weshalb sich mit dem euklidischen Algorthimus $c^{(i)}_1, c^{(i)}_2 \in R$ bestimmen lassen, so dass
    \[
      c^{(i)}_1 a_i + c^{(i)}_2 a_1 \dotsm a_{i-1} a_{i+1} \dotsm a_n = 1
    \]
    gilt.
    Für das Element $k_i \define c^{(i)}_2 a_1 \dotsm a_{i-1} a_{i+1} \dotsm a_n$ gilt dann
    \[
      \left\{
        \begin{array}{rcll}
          k_i &\equiv&  0 & \mod a_1  \,, \\
              &\vdots&    &               \\
          k_i &\equiv&  1 & \mod a_i  \,, \\
              &\vdots&    &               \\
          k_i &\equiv&  0 & \mod a_n  \,,
        \end{array}
      \right.
    \]
    Deshalb ist
    \[
              b
      \define b_1 k_1 + \dotsb + b_n k_n
    \]
    eine konkrete Lösung.
\end{itemize}




