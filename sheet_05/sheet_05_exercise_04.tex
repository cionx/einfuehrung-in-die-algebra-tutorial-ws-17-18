\section{}





\subsection{}

Für alle $(a,s) \in R \times S$ gilt $(a,s) \sim (a,s)$, denn für $1 \in S$ gilt
\[
    1 \cdot (a s - a s
  = 0  \,.
\]
Also ist $\sim$ reflexiv.
Für alle $(a, s), (a', s') \in R \times S$ mit $(a,0s) \sim (a', s')$ gibt es ein $t \in S$ mit
\[
    t \cdot (a s' - a' s)
  = 0  \,.
\]
Dann gilt
\[
    t \cdot (a' s - a s')
  = t \cdot (-(a s' - a' s))
  = -(t \cdot (a s' - a' s))
  = -0
  = 0  \,,
\]
und somit ebenfalls $(a', s') \sim (a, s)$.
Das zeigt, dass $\sim$ symmetrisch ist.

Für alle $(a, s), (a', s'), (a'', s'') \in R \times S$ mit $(a, s) \sim (a', s')$ und $(a', s') \sim (a'', s'')$ gibt es $t, u \in S$ mit
\begin{gather*}
    t \cdot (a' s - a s')
  = 0
  \quad\text{und}\quad
    u \cdot (a'' s' - a' s'')
  = 0  \,,
\shortintertext{also mit}
    t \cdot a s'
  = t \cdot a' s
  \quad\text{und}\quad
    u \cdot a' s''
  = u \cdot a'' s'  \,.
\end{gather*}
Diese Gleichungen sollte man so lesen, dass sich in Anwesenheit des Elements $t$ die Ersetzung $a s' \to a' s$ durchführen lässt, und in Anwesenheit des Elements $u$ die Ersetzung $a' s'' \to a'' s'$.
In Anwesenheit des Elementes $s'tu$ lässt sich dann auch die Ersetzung $a s'' \to a'' s$ durchführen, da
\[
    s'tu \cdot a'' s
  = st \cdot u \cdot a'' s'
  = st \cdot u \cdot a' s''
  = s''u \cdot t \cdot a' s
  = s''u \cdot t \cdot a s'
  = s' t u \cdot a s''
\]
gilt.
Das zeigt die Transitivität von $\sim$.

Ingesamt zeigt dies, dass $\sim$ eine Äquivalenzrelation ist.
Anstelle von $[a,s]$ schreiben wir im Folgenden
\[
  \frac{a}{s}
\]
oder $a/s$ für die Äquivalenzklasse von $(a,s) \in R \times S$.





\subsection{}

Es seien $(a, s), (a', s'), (b, t), (b', t') \in R \times S$ mit $(a, s) \sim (a', s')$ und $(b, t) \sim (b', t')$.
Dann gibt es $u_1, u_2 \in S$ mit
\begin{gather*}
  u_1 \cdot (a s' - a' s) = 0
  \quad\text{und}\quad
  u_2 \cdot (b t' - b' t) = 0 \,,
\shortintertext{also mit}
    u_1 \cdot a s'
  = u_1 \cdot a' s
  \quad\text{und}\quad
    u_2 \cdot b t'
  = u_2 \cdot b' t \,.
\end{gather*}
Dann gilt
\begin{align*}
      u_1 u_2 \cdot (at + bs) s' t'
  &=  (u_1 u_2 \cdot ats't') + (u_1 u_2 \cdot b s s' t')  \\
  &=  (u_2 t t' \cdot u_1 \cdot a s') + (u_1 s s' \cdot u_2 \cdot b t') \\
  &=  (u_2 t t' \cdot u_1 \cdot a' s) + (u_1 s s' \cdot u_2 \cdot b' t) \\
  &=  (u_1 u_2 \cdot a' s t t') + (u_1 u_2 \cdot b' s s' t)
   =  u_1 u_2 \cdot (a' t' + b' s') st
\end{align*}
und somit
\begin{gather*}
    u_1 u_2 \cdot ( (a t + b s) s' t' - (a' t' + b' s') )
  = 0 \,,
\shortintertext{also}
    \frac{a t + b s}{s t}
  = \frac{a' t' + b' s'}{s' t'} \,.
\end{gather*}
Das zeigt, dass die Addition
\[
            \frac{a}{s} + \frac{b}{t}
  \coloneqq \frac{at + bs}{st}
\]
auf $S^{-1} R$ wohldefiniert ist.

Außerdem gilt
\begin{gather*}
  u_1 u_2 \cdot a b s' t'
  = (u_1 \cdot a s') (u_2 \cdot b t')
  = (u_1 \cdot a' s) (u_2 \cdot b' t)
  = u_1 u_2 \cdot a' b' s t
\shortintertext{und somit}
    u_1 u_2 \cdot (a b s' t' - a' b' s t)
  = 0 \,,
\shortintertext{also}
    \frac{a b}{s t}
  = \frac{a' b'}{s' t'}.
\end{gather*}
Das zeigt, dass die Multiplikation
\[
    \frac{a}{s} \cdot \frac{b}{t}
  = \frac{a b}{s t}
\]
auf $S^{-1} R$ wohldefiniert ist.

Das folgende Lemma erweist sich zum Rechnen in $S^{-1} R$ als sehr nützlich:

\begin{lemma}[Kürzen von Brüchen]
  \label{lemma: cancelation rules for fractions}
  Für alle $a/s \in S^{-1} R$ und $t \in S$ gilt
  \[
      \frac{a}{s}
    = \frac{at}{st} \,.
  \]
\end{lemma}

\begin{proof}
  Für $1 \in S$ gilt $1 \cdot (a st - at s) = 0$, also gilt $(a,s) \sim (at,st)$.
\end{proof}

Hieraus ergibt sich insbesondere, dass $0/1 = 0/s$ für alle $s \in S$ gilt, da
\[
    \frac{0}{s}
  = \frac{0 \cdot s}{1 \cdot s}
  = \frac{0}{1}
\]
gilt.

Die Assoziativität und Kommutativität der Addition und Multiplikation, sowie die Distributivität folgen durch direktes Nachrechnen.
Das Einselement in $S^{-1} R$ ist $1/1$, denn für alle $a/s \in S^{-1} R$ gilt
\[
    \frac{a}{s} \cdot \frac{1}{1}
  = \frac{a \cdot 1}{s \cdot 1}
  = \frac{a}{s} \,.
\]
Das Nullelement ist $0/1$, denn für alle $a/s \in S^{-1} R$ gilt
\[
    \frac{a}{s} + \frac{0}{1}
  = \frac{a \cdot 1 + 0 \cdot s}{s \cdot 1}
  = \frac{a}{s} \,.
\]
Das additiv Inverse Element zu $a/s \in S^{-1} R$ ist $(-a)/s$, denn es gilt
\[
    \frac{a}{s} + \frac{-a}{s}
  = \frac{a s - a s}{s^2}
  = \frac{0}{s^2}
  = \frac{0}{1} \,.
\]
Ingesamt zeigt dies, dass $S^{-1} R$ mit der gegebenen Addition und Multiplikation einen kommutativen Ring ergibt.

\begin{remark}
  \label{remark: universal property of localization}
  Ist $R$ ein kommutativer Ring und $S \subseteq R$ eine multiplikative Teilmenge, so ist die Abbildung $f \colon R \to R_S$ ein Ringhomomorphismus.
  (Dies ergibt sich durch direktes Nachrechnen.)
  
  Für jedes $s \in S$ ist das Element $f(s) = s/1 \in S^{-1} R$ eine Einheit, da
  \[
      \frac{s}{1} \cdot \frac{1}{s}
    = \frac{s}{s}
    = \frac{1}{1}
    = 1_{S^{-1} R}
  \]
  gilt.
  In dem Ring $S^{-1} R$ werden die Elemente aus $S$ also Einheiten.
  
  Der Ring $S^{-1} R$ (zusammen mit dem Homomorphismus $f$) ist universell mit dieser Eigenschaft:
  Ist $T$ ein Ring und $g \colon R \to T$ ein Ringhomomorphismus, so dass $g(s)$ für jedes $s \in S$ eine Einheit ist, so gibt es einen eindeutigen Ringhomomorphismus $\hat{\varphi} \colon S^{-1} R \to T$, der das folgende Diagramm zum Kommutieren bringt:
  \[
    \begin{tikzcd}
        S^{-1} R
        \arrow{r}{\hat{g}}
      & T
      \\
        R
        \arrow{u}{f}
        \arrow[swap]{ru}{g}
      & {}
    \end{tikzcd}
  \]
  Der Homomorphismus $\hat{g}$ ist gegeben durch
  \[
      \hat{g}\left( \frac{a}{s} \right)
    = \frac{g(a)}{g(s)}
    = g(a) g(s)^{-1}
    \qquad
    \text{für alle $\frac{a}{s} \in S^{-1} R$} \,.
  \]
  
  Man bezeichnet dies als die \emph{universelle Eigenschaft der Lokalisierung}.
\end{remark}

\begin{remark}
  \label{remark: injectivity of the canonical map}
  Man beachte aber, dass der Ringhomomorphismus $f \colon R \to S^{-1} R$, $r \mapsto r/1$ im Allgemeinen nicht injektiv ist:
  Für alle $r \in R$ gilt
  \[
          r \in \ker f
    \iff  \frac{r}{1} = \frac{0}{1}
    \iff  \exists s \in S: rs = 0 \,.
  \]
  Somit ist $f$ genau dann injektiv, wenn für alle $s \in S$ und $r \in R$ mit $rs = 0$ bereits $r = 0$ folgt, d.h.\ wenn $S$ keine Nullteiler enthält.
  
  Ist inbesondere $R$ ein Integritätsbereich, so ist im Fall $0 \notin S$ der Ringhomomorphismus $f \colon R \to S^{-1} R$ stets injektiv.
  Dann lässt sich $R$ als ein Unterring von $S^{-1} R$ auffassen.
\end{remark}





\subsection{}

Da $R$ ein Integritätsbereich ist, gilt $1 \neq 0$, und somit $1 \in S$.
Für alle $s, t \in S$ gilt $s, t \neq 0$, wegen der Nullteilerfreiheit von $R$ also $st \neq 0$ und somit $st \in S$.
Das zeigt, dass $S$ eine multiplikative Menge ist.

Bevor wir zeigen, dass $\Quot{R}$ ein Körper ist, wollen wir anmerken, dass sich die Gleichheitsregel für Brüche im Falle in der gegebenen Situation vereinfachen:
Für zwei Brüche $a/s, b/t \in \Quot{R}$ gilt genau dann $a/s = b/t$, wenn es ein $u \in S$ mit
\[
    u \cdot (at - bs) = 0
\]
gibt.
Dabei gilt $u \neq 0$ (da $S = R \smallsetminus \{0\}$), weshalb dies nach der Nullteilerfreiheit von $R$ bereits äquivalent dazu ist, dass $at - bs = 0$ gilt.
Es gilt also
\begin{equation}
  \label{equation: comparing fractions over integral domains}
    \frac{a}{s}
  = \frac{b}{t}
  \iff
    at = bs \,.
\end{equation}
Wir können Brüche in $\Quot{R}$ also auf die \enquote{naive} Art und Weise vergleichen.

\begin{remark}
  Ist allgemeiner $R$ ein Integritätsbereich und $S \subseteq R$ eine multiplikative Teilmenge mit $0 \notin S$, so gilt für $a/s, b/t \in \Quot{R}$ genau dann $a/s = b/t$, wenn $at = bs$ gilt.
  Dies ergibt sich unverändert aus der obigen Rechnung.
  Für nicht-triviale (also vom Nullring $0$ verschiedene) Lokalisierungen von Integritätsbereichen gilt also die \enquote{naive} Gleichheitsregel für Brüche.
\end{remark}

Da $R$ ein Integritätsbereich ist, gilt $0_R \neq 1_R$.
Deshalb gilt auch $0_{\Quot{R}} \neq 1_{\Quot{R}}$, denn es gilt
\[
        \frac{0}{1} = \frac{1}{1}
  \iff  0 \cdot 1 = 1 \cdot 1
  \iff  0 = 1 \,.
\]
Es sei nun $a/s \in S^{-1} R$ mit $a/s \neq 0$.
Dann gilt $a \neq 0$, weshalb der Bruch $s/a \in \Quot{R}$ wohldefiniert ist.
Es gilt
\[
    \frac{a}{s} \cdot \frac{s}{a}
  = \frac{as}{sa}
  = \frac{1}{1}
  = 1_{S^{-1} R} \,,
\]
was zeigt, dass $a/s$ eine Einheit in $\Quot{R}$ ist.

Zusammen zeigt dies, dass der kommutative Ring $\Quot{R}$ bereits ein Körper ist.

\begin{remark}
  Nach Bemerkung~\ref{remark: injectivity of the canonical map} lässt sich $R$ durch den Ringhomomorphismus $R \to \Quot{R}$, $r \mapsto r/1$ als einen Unterring von $\Quot{R}$ auffassen.
  Da jeder Unterring eines Körpers auch ein Integritätsbereich ist, erhalten wir damit eine Charakterisierung von Integritätsbereichen:
  \begin{center}
    \emph{Integritätsbereiche sind genau die Unterringe von Körpern.}
  \end{center}
  
  Dies liefert auch eine mögliche Erklärung, warum der Nullring kein Integritätsbereich ist:
  Es handelt sich nicht um den Unterring eines Körpers.
\end{remark}

\begin{remark}
  Ist allgemeiner $R$ ein kommutativer Ring und $P \ideal R$ ein Primideal, so ist $S_P \coloneqq R \smallsetminus P$ eine multiplikative Teilmenge:
  Es gilt $1 \notin P$ da $P \neq R$, und somit $1 \in S_P$.
  Für alle $x, y \in S_P$ gilt $x, y \notin P$, somit auch $xy \notin P$ (da $P$ prim ist), und deshalb $xy \in S_P$.
  
  Man bezeichnet den Ring $R_P \coloneqq S_P^{-1} R$ als die \emph{Lokalisierung von $R$ an $P$}.
  Bei $R_P$ behandelt es sich um einen sogennanten \emph{lokalen Ring}, d.h.\ $R_P$ besitzt genau ein maximales Ideal (nämlich $S_P^{-1} P$).
  Diese Konstruktion spielt eine wichtige Rolle in der kommutativen Ringe und algebraischen Geometrie.
  
  Ist dabei $R$ ein Integritätsbereich, so ist $0 \ideal R$ ein Primideal, und es folgt, dass $S = S_0 = R \smallsetminus \{0\}$ eine multiplikative Teilmenge ist.
  Zudem ist dann $S_0^{-1} 0 = S^{-1} 0 = 0$ das eindeutige maximale Ideal von $S^{-1} R$.
  Inbesondere ist das Nullideal in $S^{-1} R$ maximal, und $S^{-1} R$ somit ein Körper.
\end{remark}





\subsection{}

Die Abbildung $\tilde{\varphi} \colon \Integer \to \Rational$, $n \mapsto n/1$ ist injektiv.
Für $S \coloneqq \Integer \smallsetminus \{0\}$ gilt deshalb
\[
            \tilde{\varphi}(S)
  =         \tilde{\varphi}(\Integer \smallsetminus \{0\})
  =         \tilde{\varphi}(\Integer) \smallsetminus \{0\}
  \subseteq \Rational \smallsetminus \{0\}
  =         \Rational^\times \,.
\]
Nach der universellen Eigenschaft der Lokalsierung (siehe Bemerkung~\ref{remark: universal property of localization}) induziert $\tilde{\varphi}$ einen Ringhomomorphismus
\[
          \varphi
  \colon  \Quot{\Integer}
  =       S^{-1} \Integer
  \to     \Rational,
  \quad   \frac{p}{q}
  \mapsto \tilde{\varphi}(p) \tilde{\varphi}(q)^{-1}
  =       p q^{-1}
  =       \frac{p}{q}.
\]
Der Ringhomomorphismus $\varphi$ ist surjektiv, und als Körperhomomorphismus auch injektiv.
Die Injektivität lässt sich auch von Hand nachrechen, denn für alle $p/q \in \Quot{\Integer}$ gilt
\[
    \varphi\left( \frac{p}{q} \right) = 0
  \implies
    \text{$\frac{p}{q} = 0$ (in $\Rational$)}
  \implies
    \text{$p = 0$ (in $\Integer$)}
  \implies
    \text{$\frac{p}{q} = 0$ (in $\Quot{\Integer}$)}.
\]
Also ist $\varphi$ ein Isomorphismus.

\begin{remark}
  Sofern die universelle Eigenschaft der Lokalisierung noch nicht zur Verfügung steht, so muss von Hand begründet werden, warum $\varphi$ ein wohldefinierter Ringhomomorphismus ist:
  \begin{enumerate}
    \item
      Für alle $p_1, p_2 \in \Integer$ und $q_1, q_2 \in S$ mit $p_1 / q_1 = p_2 / q_2$ in $\Quot{\Integer}$ gilt $p_1 q_2 = p_2 q_1$ (siehe \eqref{equation: comparing fractions over integral domains}) und somit $p_1 / q_1 = p_2 / q_1$ in $\Rational$.
      Somit ist $\varphi$ wohldefiniert.
    \item
      Dass $\varphi$ ein Ringhomomorphismus ist, ergibt sich durch direktes Nachrechnen.
  \end{enumerate}
\end{remark}

\begin{remark}
  Ist $R$ ein beliebiger Integritätsbereich und $K$ ein Körper, so erhalten wir analog zur obigen Rechnung, dass jeder injektive Ringhomomorphismus $j \colon R \to K$ einen eindeutigen Körperhomomorphismus $\induced{j} \colon \Quot{R} \to K$ induziert, der für die Inklusion $i \colon R \to \Quot{R}$, $r \mapsto r/1$ das folgende Diagramm zum Kommutieren bringt:
  \[
    \begin{tikzcd}
        \Quot{R}
        \arrow{r}{\induced{j}}
      & K
      \\
        R
        \arrow{u}{i}
        \arrow[swap]{ur}{j}
      & {}
    \end{tikzcd}
  \]
  Man bemerke, dass dabei $\induced{j}$ injektiv ist (da $\Quot{R}$ ein Körper ist, und $K \neq 0$ gilt), also $\Quot{R}$ durch $\overline{j}$ mit einem Unterkörper von $K$ identifiziert wird.
  Anschaulich bedeutet dies:
  \begin{center}
    \emph{Jeder Körper $K$, der $R$ enthält, enthält auch schon $\Quot{R}$.}
  \end{center}
  Im Falle von $R = \Integer$ und $K = \Rational$ wird $K$ als Körper bereits von $R$ erzeugt, weshalb $\Quot{R}$ bereits ganz $\Rational$ seien muss.
\end{remark}





\subsection{}

Wir gehen wie im vorherigen Aufgabenteil vor:
Für die Inklusion $i \colon \Integer \to \Rational$, $a \mapsto a$ gilt
\[
            \varphi(S)
  =         S
  \subseteq \Rational \setminus \{0\}
  =         \Rational^\times \,.
\]
Deshalb induziert $i$ nach der universellen Eigenschaft der Lokalisierung einen Ringhomomorphismus
\[
          \induced{i}
  \colon  S^{-1} \Integer
  \to     \Rational,
  \quad   \frac{a}{p^k}
  \mapsto \frac{i(n)}{i(p^k)}
  =       \frac{a}{p^k} \,.
\]
Insbesondere ist
\[
    \im \induced{i}
  = \left\{
      \frac{a}{p^k}
    \suchthat*
      a, k \in \Integer,
      k \geq 0
    \right\}
\]
ein Unterring von $\Rational$.
Die Injektivität von $\induced{i}$ ergibt sich wie im vorherigen Aufgabenteil.
Damit ergibt sich dann, dass $\induced{i}$ eine Isomorphie $S^{-1} \Integer \cong \im \induced{i}$ induziert.
