\section{}





\subsection{}

Für alle $(a,s) \in R \times S$ gilt $(a,s) \sim (a,s)$, denn für $1 \in S$ gilt
\[
    1 \cdot (a s - a s
  = 0
  \,.
\]
Also ist $\sim$ reflexiv.
Für alle $(a, s), (a', s') \in R \times S$ mit $(a,0s) \sim (a', s')$ gibt es ein $t \in S$ mit
\[
    t \cdot (a s' - a' s)
  = 0
  \,.
\]
Dann gilt
\[
    t \cdot (a' s - a s')
  = t \cdot (-(a s' - a' s))
  = -(t \cdot (a s' - a' s))
  = -0
  = 0
  \,,
\]
und somit ebenfalls $(a', s') \sim (a, s)$.
Das zeigt, dass $\sim$ symmetrisch ist.

Für alle $(a, s), (a', s'), (a'', s'') \in R \times S$ mit $(a, s) \sim (a', s')$ und $(a', s') \sim (a'', s'')$ gibt es $t, u \in S$ mit
\[
    t \cdot (a' s - a s')
  = 0
  \quad\text{und}\quad
    u \cdot (a'' s' - a' s'')
  = 0
  \,,
\]
also mit
\[
    t \cdot a s'
  = t \cdot a' s
  \quad\text{und}\quad
    u \cdot a' s''
  = u \cdot a'' s'
  \,.
\]
Diese Gleichungen sollte man so lesen, dass sich in Anwesenheit des Elements $t$ die Ersetzung $a s' \to a' s$ durchführen lässt, und in Anwesenheit des Elements $u$ die Ersetzung $a' s'' \to a'' s'$.
In Anwesenheit des Elementes $s'tu$ lässt sich dann auch die Ersetzung $a s'' \to a'' s$ durchführen, da
\[
    s'tu \cdot a'' s
  = st \cdot u \cdot a'' s'
  = st \cdot u \cdot a' s''
  = s''u \cdot t \cdot a' s
  = s''u \cdot t \cdot a s'
  = s' t u \cdot a s''
\]
gilt.
Das zeigt die Transitivität von $\sim$.
Ingesamt zeigt dies, dass $\sim$ eine Äquivalenzrelation ist.




