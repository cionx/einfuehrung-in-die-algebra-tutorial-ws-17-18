\section{}





\subsection{}

Für alle $(a,s) \in R \times S$ gilt $(a,s) \sim (a,s)$, denn für $1 \in S$ gilt
\[
    1 \cdot (a s - a s
  = 0  \,.
\]
Also ist $\sim$ reflexiv.
Für alle $(a, s), (a', s') \in R \times S$ mit $(a,0s) \sim (a', s')$ gibt es ein $t \in S$ mit
\[
    t \cdot (a s' - a' s)
  = 0  \,.
\]
Dann gilt
\[
    t \cdot (a' s - a s')
  = t \cdot (-(a s' - a' s))
  = -(t \cdot (a s' - a' s))
  = -0
  = 0  \,,
\]
und somit ebenfalls $(a', s') \sim (a, s)$.
Das zeigt, dass $\sim$ symmetrisch ist.

Für alle $(a, s), (a', s'), (a'', s'') \in R \times S$ mit $(a, s) \sim (a', s')$ und $(a', s') \sim (a'', s'')$ gibt es $t, u \in S$ mit
\begin{gather*}
    t \cdot (a' s - a s')
  = 0
  \quad\text{und}\quad
    u \cdot (a'' s' - a' s'')
  = 0  \,,
\shortintertext{also mit}
    t \cdot a s'
  = t \cdot a' s
  \quad\text{und}\quad
    u \cdot a' s''
  = u \cdot a'' s'  \,.
\end{gather*}
Diese Gleichungen sollte man so lesen, dass sich in Anwesenheit des Elements $t$ die Ersetzung $a s' \to a' s$ durchführen lässt, und in Anwesenheit des Elements $u$ die Ersetzung $a' s'' \to a'' s'$.
In Anwesenheit des Elementes $s'tu$ lässt sich dann auch die Ersetzung $a s'' \to a'' s$ durchführen, da
\[
    s'tu \cdot a'' s
  = st \cdot u \cdot a'' s'
  = st \cdot u \cdot a' s''
  = s''u \cdot t \cdot a' s
  = s''u \cdot t \cdot a s'
  = s' t u \cdot a s''
\]
gilt.
Das zeigt die Transitivität von $\sim$.

Ingesamt zeigt dies, dass $\sim$ eine Äquivalenzrelation ist.
Anstelle von $[a,s]$ schreiben wir im Folgenden
\[
  \frac{a}{s}
\]
oder $a/s$ für die Äquivalenzklasse von $(a,s) \in R \times S$.





\subsection{}

Es seien $(a, s), (a', s'), (b, t), (b', t') \in R \times S$ mit $(a, s) \sim (a', s')$ und $(b, t) \sim (b', t')$.
Dann gibt es $u_1, u_2 \in S$ mit
\begin{gather*}
  u_1 \cdot (a s' - a' s) = 0
  \quad\text{und}\quad
  u_2 \cdot (b t' - b' t) = 0 \,,
\shortintertext{also mit}
    u_1 \cdot a s'
  = u_1 \cdot a' s
  \quad\text{und}\quad
    u_2 \cdot b t'
  = u_2 \cdot b' t \,.
\end{gather*}
Dann gilt
\begin{align*}
      u_1 u_2 \cdot (at + bs) s' t'
  &=  (u_1 u_2 \cdot ats't') + (u_1 u_2 \cdot b s s' t')  \\
  &=  (u_2 t t' \cdot u_1 \cdot a s') + (u_1 s s' \cdot u_2 \cdot b t') \\
  &=  (u_2 t t' \cdot u_1 \cdot a' s) + (u_1 s s' \cdot u_2 \cdot b' t) \\
  &=  (u_1 u_2 \cdot a' s t t') + (u_1 u_2 \cdot b' s s' t)
   =  u_1 u_2 \cdot (a' t' + b' s') st
\end{align*}
und somit
\begin{gather*}
    u_1 u_2 \cdot ( (a t + b s) s' t' - (a' t' + b' s') )
  = 0 \,,
\shortintertext{also}
    \frac{a t + b s}{s t}
  = \frac{a' t' + b' s'}{s' t'} \,.
\end{gather*}
Das zeigt, dass die Addition
\[
            \frac{a}{s} + \frac{b}{t}
  \coloneqq \frac{at + bs}{st}
\]
auf $S^{-1} R$ wohldefiniert ist.

Außerdem gilt
\begin{gather*}
  u_1 u_2 \cdot a b s' t'
  = (u_1 \cdot a s') (u_2 \cdot b t')
  = (u_1 \cdot a' s) (u_2 \cdot b' t)
  = u_1 u_2 \cdot a' b' s t
\shortintertext{und somit}
    u_1 u_2 \cdot (a b s' t' - a' b' s t)
  = 0 \,,
\shortintertext{also}
    \frac{a b}{s t}
  = \frac{a' b'}{s' t'}.
\end{gather*}
Das zeigt, dass die Multiplikation
\[
    \frac{a}{s} \cdot \frac{b}{t}
  = \frac{a b}{s t}
\]
auf $S^{-1} R$ wohldefiniert ist.

Das folgende Lemma erweist sich zum Rechnen in $S^{-1} R$ als sehr nützlich:

\begin{lemma}[Kürzen von Brüchen]
  \label{lemma: cancelation rules for fractions}
  Für alle $a/s \in S^{-1} R$ und $t \in S$ gilt
  \[
      \frac{a}{s}
    = \frac{at}{st} \,.
  \]
\end{lemma}

\begin{proof}
  Für $1 \in S$ gilt $1 \cdot (a st - at s) = 0$, also gilt $(a,s) \sim (at,st)$.
\end{proof}

Hieraus ergibt sich insbesondere, dass $0/1 = 0/s$ für alle $s \in S$ gilt, da
\[
    \frac{0}{s}
  = \frac{0 \cdot s}{1 \cdot s}
  = \frac{0}{1}
\]
gilt.

Die Assoziativität und Kommutativität der Addition und Multiplikation, sowie die Distributivität folgen durch direktes Nachrechnen.
Das Einselement in $S^{-1} R$ ist $1/1$, denn für alle $a/s \in S^{-1} R$ gilt
\[
    \frac{a}{s} \cdot \frac{1}{1}
  = \frac{a \cdot 1}{s \cdot 1}
  = \frac{a}{s} \,.
\]
Das Nullelement ist $0/1$, denn für alle $a/s \in S^{-1} R$ gilt
\[
    \frac{a}{s} + \frac{0}{1}
  = \frac{a \cdot 1 + 0 \cdot s}{s \cdot 1}
  = \frac{a}{s} \,.
\]
Das additiv Inverse Element zu $a/s \in S^{-1} R$ ist $(-a)/s$, denn es gilt
\[
    \frac{a}{s} + \frac{-a}{s}
  = \frac{a s - a s}{s^2}
  = \frac{0}{s^2}
  = \frac{0}{1} \,.
\]
Ingesamt zeigt dies, dass $S^{-1} R$ mit der gegebenen Addition und Multiplikation einen kommutativen Ring ergibt.




