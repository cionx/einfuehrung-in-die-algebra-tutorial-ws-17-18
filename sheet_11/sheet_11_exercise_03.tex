\section{}





\subsection{}

Mit $f(t) = \sum_i a_i t^i$ und $g(t) = \sum_i b_i t^i$ gilt $(r \cdot f + s \cdot g)(t) = \sum_i (r \cdot a_i + s \cdot b_i) t^i$, und somit
\begin{align*}
      (r \cdot f + s \cdot g)'(t)
   =  \sum_i i (r \cdot a_i + s \cdot b_i) t^{i-1}
  &=  r \cdot \sum_i i a_i t^{i-1} + r \cdot \sum_i i b_i t^{i-1}
  \\
  &=  r \cdot f'(t) + s \cdot g'(t) \,.
\end{align*}





\subsection{}

Mit $f(t) = \sum_i a_i t^i$ und $g(t) = \sum_j b_j t^j$ gilt $(f \cdot g)(t) = \sum_{i,j} a_i b_j t^{i+j}$ und somit
\begin{align*}
      (f \cdot g)'(t)
  &=  \sum_{i,j} a_i b_j (i+j) t^{i+j-1}
   =  \sum_{i,j} a_i b_j i t^{i+j-1}
    + \sum_{i,j} a_i b_j j t^{i+j-1}
  \\
  &=  \left( \sum_i a_i i t^{i-1} \right) \left( \sum_j b_j t^j \right)
    + \left( \sum_i a_i t^i \right) \left( \sum_j b_j j t^{j-1} \right)
  \\
  &= f'(t) \cdot g(t) + f(t) \cdot g'(t) \,.
\end{align*}





\subsection{}

Hat $f(t)$ keine $9$ verschiedenen Nullstellen, so hat $f(t)$ in $\Finite{9}$, und somit auch in $\overline{\Finite{3}}$ eine mehrfache Nullstelle.
Da $f(t)$ irreduzibel ist, gilt somit (wie in der Vorlesung gezeigt), dass $f'(t) = 0$.
Insbesondere ist dann jedes $x \in \Finite{9}$ eine Nullstelle von $f'(t)$.





\subsection{}

Mit \enquote{doppelten} Nullstellen sind in dieser Aufgaben die mehrfachen Nullstellen gemeint.



\subsubsection{}

Für das gegebene Polynom $f(t) \coloneqq t^6 + t^5 - t^4 - t^3 - t^2 + t$ gilt
\[
    f'(t)
  = 5 t^4 - 4 t^3 - 2 t + 1
  = 2 t^4 + 2 t^3 + t + 1
  = t^4 + t^3 + 2 t + 2 \,.
\]
Es gibt nun (mindestens) zwei Vorgehensweisen:
Wir bestimmen jeweils die gemeinsamen Nullstellen von $f(t)$ und $f'(t)$;
dies sind dann genau die mehrfachen Nullstellen von $f(t)$ und $f'(t)$.

\begin{itemize}
  \item
    Es gilt
    \begin{align*}
          f'(t)
      &=  t^4 + t^3 + 2 t + 2
       =  t^3 (t+1) + 2 (t+1) \\
      &=  (t^3 + 2)(t + 1)
       =  (t + 2)^3 (t + 1)
       =  (t - 1)^3 (t - 1) \,.
    \end{align*}
    (Dabei nutzen wir für die Umfomung $(t^3 + 2) = (t+2)^3$, dass $\ringchar{\Finite{9}} = 3$ gilt.)
    Also zerfällt $f'(t)$ bereits über $\Finite{3}$ in Linearfaktoren, und die auftretenden Nullstellen sind $1$ und $-1$.
    Dabei ist $1$ auch eine Nullstelle von $f(t)$, $-1$ hingegen nicht.
    
    Somit ist $1$ die einzige gemeinsame Nullstelle von $f(t)$ und $f'(t)$, und somit auch die einzige mehrfache Nullstelle von $f(t)$.
  \item
    Mithife des euklidischen Algorithmus ergibt sich, dass $(t-1)^3$ der größte gemeinsame Teiler von $f(t)$ und $f'(t)$ ist.
    Da die gemeinsamen Nullstellen von $f(t)$ und $f'(t)$ genau die Nullstellen dieses größten gemeinsamen Teilers sind, ist $1$ die einzige mehrfache Nullstelle des Polynoms $f(t)$.
\end{itemize}



\subsection{}

Für das gegebene Polynom $g(t) \coloneqq t^{12} + t^6 + t^4 + 2 t^3 + t$ gilt
\[
    g'(t)
  = 4 t^3 + 1
  = t^3 + 1
  = (t + 1)^3 \,.
\]
Da $-1$ eine Nullstelle von $g(t)$ ist, ist somit $-1$ die einzige gemeinsame Nullstelle des Polynoms $g(t)$.





