\section{}





\subsection{}

Da $\Rational \subseteq K$ der Primkörper von $K$ ist, gilt für den Körperhomomorphismus $f$, dass $\restrict{f}{\Rational} = \id_\Rational$.
Für das Minimalpolynom $m_a(t) = \sum_i p_i t^i \in \Rational[t]$ gilt $p_i \in \Rational$ für alle $i$;
für jede Nullstelle $b \in S_a$ von $m_a(t)$ gilt somit
\[
    m_a( f(b) )
  = \sum_i p_i f(b)^i
  = \sum_i f(p_i) f(b)^i
  = f\left( \sum_i p_i b^i \right)
  = f( m_a(b) )
  = f( 0 )
  = 0 \,,
\]
und somit auch $f(b) \in S_a$.





\subsection{}

Für jedes $a \in K$ ist die Menge $S_a$ endlich, da das Polynom $m_a(t) \in \Rational[t] \subseteq K[t]$ nur endlich viele Nullstellen hat.
Die Einschränkung $\restrict{f}{S_a}[S_a] \colon S_a \to S_a$ ist injektiv, da $f$ als Körperhomomorphismus injektiv ist, und wegen der Endlichkeit von $S_a$ somit auch surjektiv.
Also gilt $f(S_a) = S_a$.





\subsection{}

Für jedes $a \in K$ gilt $a \in S_a$, weshalb $K = \bigcup_{a \in K} S_a$ gilt.
Damit folgt, dass
\[
    f(K)
  = f\left( \bigcup_{a \in K} S_a \right)
  = \bigcup_{a \in K} f(S_a)
  = \bigcup_{a \in K} S_a
  = K
\]
gilt.
Somit ist der Körperhomomorphismus $f$ surjektiv, und somit bereits ein Körperisomorphismus (als Körperhomomorphismus ist $f$ insbesondere injektiv).
