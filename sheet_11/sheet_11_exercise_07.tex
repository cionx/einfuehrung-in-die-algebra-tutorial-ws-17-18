\section{}

\begin{lemma}
  Es sei $p$ prim, und es seien $n, m \geq 1$.
  Dann gibt es genau dann eine Einbettung $\Finite{p^n} \hookrightarrow \Finite{p^m}$, also einen Körperhomomorphismus $\Finite{p^n} \to \Finite{p^m}$, wenn $n \divides m$ gilt.
\end{lemma}

\begin{proof}
  Falls es eine Einbettung $\Finite{p^n} \hookrightarrow \Finite{p^m}$ gibt, so können wir o.B.d.A.\ den Körper $\Finite{p^n}$ als einen Unterkörper von $\Finite{p^m}$ auffassen.
  Nach der Multiplikativität des Grades gilt dann
  \[
      m
    = [\Finite{p^m} : \Finite{p}]
    = [\Finite{p^m} : \Finite{p^n}] [\Finite{p^n} : \Finite{p}]
    = [\Finite{p^m} : \Finite{p^n}] \cdot n \,,
  \]
  und somit $n \divides m$.
  Gilt andererseits $n \divides m$, so lässt sich auf verschiedene Weisen vorgehen:
  \begin{itemize}
    \item
      Das Polynom $t^{p^n} - t$ ein Teiler des Polynoms $t^{p^m} - t$.
      Dies lässt sich auf verschiedene Weisen sehen:
      \begin{itemize}
        \item
          Ist $\overline{\Finite{p}}$ ein algebraischer Abschluss von $\Finite{p}$, so zerfallen beide Polynome über $\overline{\Finite{p}}$ in paarweise verschieden Linearfaktoren (denn beide Polynome sind separabel).
          Für jede Nullstelle $x \in \overline{\Finite{p}}$ von $t^{p^n} - t$ gilt $x^{p^n} = x$.
          Dann gilt auch $x^{p^m} = x$, da $m$ ein Vielfaches von $n$ ist (und somit die Abbildung $y \mapsto y^{p^m}$ durch $(m/n)$-faches Anwenden der Abbildung $y \mapsto y^{p^n}$ gegeben ist).
          Also ist jeder Linearfaktor von $t^{p^n} - t$ auch ein Linearfaktor von $t^{p^m} - t$.
        \item
          Indem wir beide Polynome durch $t$ teilen, genügt es $(t^{(p^n-1)} - 1) \divides (t^{(p^m - 1)} - 1)$ zu zeigen.
          
          \begin{claim*}
            Es sei $R$ ein Ring und es seien $a \in R$, $k \geq 1$.
            Dann gilt $(a - 1) \divides (a^k - 1)$.
          \end{claim*}
          \begin{proof}
            Es gilt $a^k - 1 = (a-1)(a^{k-1} + a^{k-2} + \dotsb + 1)$.
          \end{proof}
          
          Nach der Behauptung genügt es zu zeigen, dass $(p^n - 1) \divides (p^m - 1)$ gilt, denn dann ist $t^{(p^m - 1)}$ eine Potenz von $t^{(p^n - 1)}$.
          Durch erneutes Anwenden der Behauptung genügt es hierfür zu zeigen, dass $p^m$ eine Potenz von $p^n$ ist.
          Dies ergibt sich aus $n \divides m$.
      \end{itemize}
      
      Da $\Finite{p^m}$ ein Zerfällungkörper des Polynoms $t^{p^m} - t$ ist, folgt damit, dass $\Finite{p^m}$ einen Zerfällungskörper des Polynoms $t^{p^n} - t$, also $\Finite{p^n}$ enthält.
  \end{itemize}
  Zum besseren Verständnis der zu zeigenden Aussage möchten wir auch die folgende Argumentation angeben, die zur Abgabe des Übungszetteln noch nicht zur Verfügung stand:
  \begin{itemize}[resume]
    \item
      Die Erweiterung $\Finite{p^m}/\Finite{p}$ ist galoissch mit Galoisgruppe $\Gal(\Finite{p^m}/\Finite{p}) \cong \Integer/m$.
      Da $n \divides m$ gilt, enthält $\Gal(\Finite{p^m}/\Finite{p})$ somit eine Untergruppe $G \subgroup \Gal(\Finite{p^m}/\Finite{p})$ der Ordnung $m/n$.
      Nach dem Hauptsatz der Galoistheorie gilt für den zugehörigen Fixkörper $K \coloneqq (\Finite{p^m})^G$, dass die Körpererweiterung $\Finite{p^m}/K$ galoissch mit Galoisgruppe $\Gal(\Finite{p^m}/K) = G$ ist, und (somit) den Grad $[\Finite{p^m} : K] = \card{G} = m/n$ hat.
      Nach der Multiplikativität des Grades gilt nun
      \[
          m
        = [\Finite{p^m} : \Finite{p}]
        = [\Finite{p^m} : K] [K : \Finite{p}]
        = \frac{m}{n} \cdot [K : \Finite{p}] \,,
      \]
      und somit $[K : \Finite{p}] = n$.
      Nach der Klassifikation endlicher Körper gilt für den Unterkörper $K \subseteq \Finite{p^m}$ deshalb $K \cong \Finite{p^n}$.
  \end{itemize}
  Mit ein wenig Abänderung der obigen Idee wird nicht der gesamte Hauptsatz der Galoistheorie benötigt.
  Es genügt bereits der Spezialfall der Galois-Korrespondenz für endliche Körper (Satz~19.1), der in der Vorlesung zum Zeitpunkt der Abgabe bereits bekannt war:
  \begin{itemize}[resume]
    \item
      Wie bereits gesehen, ist für jeden Unterkörper $K \subseteq \Finite{p^m}$ der Grad $[K : \Finite{p}]$ ein Teiler des Grades $[\Finite{p^m} : \Finite{p}] = m$.
      Dabei ist $K$ durch den Grad $[K : \Finite{p}] = d$ nach der Klassifikation endlicher Körper bereits eindeutig als
      \[
          K
        = \left\{
            x \in K
          \suchthat*
            x^{p^d} = x
          \right\}
      \]
      bestimmt.
      Es gibt also höchsten so viele Unterkörper $K \subseteq \Finite{p^m}$, wie es positive Teiler $d \divides m$ gibt.
      
      Nach Satz~19.1 gibt eine 1:1-Korrespondenz
      \[
                              \{ \text{Unterkörper $K \subseteq \Finite{p^m}$}  \}
        \xleftrightarrow{1:1} \{ \text{Untergruppen $G \subgroup \Aut{\Finite{p^m}}$} \}  \,,
      \]
      wobei $\Aut{\Finite{p^m}} = \generated{\Fr} \cong \Integer/m$ gilt.
      Da $\Integer/m$ für jeden positiven Teiler $d \divides m$ eine eindeutige Untergruppe der Ordnung $d$ besitzt\footnote{
      Die vorherige Argumenation zeigt, dass es für einen positiven Teiler $d \divides m$ eigentlich natürlicher ist, die eindeutige Untergruppe von Ordnung $m/d$ zu betrachten.
      Es geht es uns in dieser Argumentation aber nur um die Anzahl der Untergruppen, weshalb wir auch die hier genutzte Formulierung wählen können.},
      enthält $\Finite{p^m}$ tatsächlich schon so viele Unterkörper $K \subseteq \Finite{p^m}$ wie es positive Teiler $d \divides m$ gibt.
      
      Zusammen erhalten wir somit, dass $\Finite{p^m}$ für jeden positven Teiler $d \divides m$ einen Unterkörper $K \subseteq \Finite{p^m}$ mit $[K : \Finite{p}] = d$ besitzt.
      Inbesondere gilt dies für $d = n$.
  \qedhere
  \end{itemize}

\end{proof}

