\section{}





\subsection{}

Wir bestimmen zunächst, wieviele $z \in K$ von der Form $z = x^2$ für passendes $x \in K$ sind, d.h.\ wie viele Zahlen $z \in K$ bereits Quadratzahlen sind.
Dabei genügt es im Folgenden, die Elemente $z \in K \smallsetminus \{0\} = K^\times$ zu betrachten, da $0^2 = 0$ gilt.

Die Abbildung
\[
          q
  \colon  K^\times
  \to     K^\times \,,
  \quad   x
  \mapsto x^2
\]
ist ein Gruppenhomomorphismus mit
\[
    \ker q
  = \{
      x \in K^\times
    \suchthat
      x^2 = 1
    \}
  = \{1, -1\} \,.
\]
(Denn dies sind genau die Nullstellen des Polynoms $t^2 - 1 = (t+1)(t-1) \in K[t]$.)
\begin{enumerate}
  \item
    Gilt $\ringchar{K} = 2$, so gilt $1 = -1$, weshalb dann $\ker q = 1$ gilt.
    In diesem Fall ist $q$ injektiv, und wegen der Endlichkeit von $q$ somit bereits bijektiv.
    Also ist dann jedes $z \in K^\times$ ein Quadrat.
  \item
    Gilt $\ringchar{K} \neq 2$, so gilt $1 \neq -1$, und somit $\card{\ker q} = 2$.
    Dann gilt
    \[
        \card{\im q}
      = \frac{\card{K^\times}}{\card{\ker q}}
      = \frac{\card{K} - 1}{2} \,.
    \]
    Dann ist also genau die Hälfte aller $z \in K^\times$ ein Quadrat.
\end{enumerate}
Im Fall $\ringchar{K} = 2$ folgt mit $0^2 = 0$, dass jedes $z \in K$ ein Quadrat ist, die Abbildung $K \to K$, $x \mapsto x^2$ also bijektiv ist.
(Man bemerke, dass dies genau der Frobenius-Automorphismus ist.)
Im Fall $\ringchar{K} \neq 2$ folgt mit $0^2 = 0$ hingegen, dass
\[
    \card{ \{ x^2 \suchthat x \in K \} }
  = 1 + \card{ \{ x^2 \suchthat x \in K^\times \} }
  = 1 + \frac{\card{K} - 1}{2}
  = \frac{\card{K} + 1}{2} \,.  
\]

Da $a \in K^\times$ gilt, ist die Abbildung $K \to K$, $z \mapsto az$ bijektiv, und somit
\[
    \card{ \{a x^2 \suchthat x \in K\} }
  = \card{ \{x^2 \suchthat x \in K\} }
  = \begin{cases}
      \card{K}        & \text{falls $\ringchar{K} = 2$}     \,, \\
      (\card{K}+1)/2  & \text{falls $\ringchar{K} \neq 2$}  \,.
    \end{cases}
\]





\addtocounter{subsection}{1}
\subsection{}

Es gilt $\ringchar{K} = 2$, da $\card{K}$ gerade ist.
Wie bereits gesehen ist die Abbildung $K \to K$, $x \mapsto a x^2$ deshalb bijektiv.
Also ist auch die Abbildung $K \to K$, $x \mapsto 1 + a x^2$ bijektiv.
Es gibt also für jedes $z \in K$ ein eindeutiges Element $x \in K$ mit $z = 1 + a x^2$.





\addtocounter{subsection}{-2}
\subsection{}

Gilt $\ringchar{K} = 2$, ist also $\card{K}$ gerade, so haben wir bereits gezeigt, dass sich sogar noch $y = 0$ wählen lässt.
Es bleibt also nur noch der Fall $\ringchar{K} \neq 2$ zu betrachten.
Wie bereits gesehen, gelten dann
\begin{gather*}
    \card{ \{ 1 + a x^2 \suchthat x \in K \} }
  = \card{ \{ a x^2 \suchthat x \in K \} }
  = \card{ \{ x^2 \suchthat x \in K \} }
  = \frac{\card{K} + 1}{2}
\shortintertext{und}
    \card{ \{ -b y^2 \suchthat y \in K \} }
  = \card{ \{ y^2 \suchthat y \in K \} }
  = \frac{\card{K} + 1}{2} \,.
\end{gather*}
Dabei gilt
\[
    \frac{\card{K}+1}{2}
  + \frac{\card{K}+1}{2}
  = \card{K} + 1
  > \card{K} \,,
\]
weshalb nach dem Schubfachprinzip
\[
        \{ 1 + a x^2 \suchthat x \in K \} \cap \{ -b y^2 \suchthat y \in K \}
  \neq  \emptyset
\]
gilt.
Es gibt also $x, y \in K$ mit $1 + a x^2 = -b y^2$, also mit $1 + a x^2 + b y^2 = 0$.
















