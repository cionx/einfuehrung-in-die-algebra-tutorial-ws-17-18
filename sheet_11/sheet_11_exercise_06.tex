\section{}





\subsection{}

Die in dieser Aufgabe gezeigte Aussage, dass jede endliche abelsche Gruppe $A$ zu einem direkten Produkt von zyklischen $p$-Gruppen isomorph ist (wobei $p$ die Primfaktoren der Ordnung $\card{A}$ durchläuft), verallgemeinert sich zum \emph{Fundamentalsatz über endlich erzeugte abelsche Gruppen}.

\begin{theorem}[Fundamentalsatz über endlich erzeugte abelsche Gruppen]
  \label{theorem: fundamental theorem for finitely generated abelian groups}
  Ist $A$ eine endlich erzeugte abelsche Gruppe, so gilt
  \[
          A
    \cong \Integer^r \times \prod_{i=1}^s (\Integer/p_i^{\nu_i})
  \]
  mit $r, s \geq 0$, $\nu_1, \dotsc, \nu_s \geq 1$ und $p_1, \dotsc, p_s$ prim.
  Dabei sind die Zahlen $r, s$ eindeutig, und die Paare $(p_1, \nu_1), \dotsc, (p_s, \nu_s)$ eindeutig bis auf Permutation.
\end{theorem}

Ist $A$ eine endliche abelsche Gruppe, so muss $r = 0$ gelten, und wir erhalten die Aussage der Aufgabe.
Wir erhalten sogar noch eine Eindeutigkeitsaussage (bis auf Permutation der Faktoren).

\begin{remark}
  In der Praxis stellt sich häufig die Frage, wie für gegebene Elemente $a_1, \dotsc, a_m \in \Integer^n$ die Gruppe
  \[
              A
    \coloneqq \Integer^n / \generated{a_1, \dotsc, a_m}
  \]
  aussieht.
  Da $\Integer^n$ als abelsche Gruppe von den (endlich vielen) Standardbasisvektoren $e_1, \dotsc, e_n \in \Integer^n$ erzeugt wird, ist auch $A$ endlich erzeugt.
  Nach Satz~\ref{theorem: fundamental theorem for finitely generated abelian groups} gilt deshalb
  \[
          A
    \cong \Integer^r \times \prod_{i=1}^s (\Integer/p_i^{\nu_i}) \,.
  \]
  Es gibt nun einen Algorithmus, um die rechte Seite zu berechnen.
  \begin{enumerate}
    \item
      Man trage zunächst die $a_1, \dotsc, a_n$ als Spalten in eine Matrix $A \in \mnatrices{m}{n}{\Integer}$ ein.
    \item
      Durch elementare Zeilen- und Spaltenumformungen (wobei man beim Skalieren der Zeilen und Spalten nur die Einheiten $\pm 1 \in \Integer$ nutzen darf) bringe man $A$ in eine $(2 \times 2)$-Blockgestalt
      \[
          D
        = \left(
          \begin{array}{ccc|c}
            d_1 &         &         & 0       \\
                & \ddots  &         & \vdots  \\
                &         & d_s     & 0       \\
            \hline
            0   & \cdots  & 0       & 0
          \end{array}
          \right) \,,
      \]
      mit $d_1, \dotsc, d_s \geq 1$.
    \item
      Es gilt dann
      \[
              A
        \cong \Integer^{m-s} \times (\Integer/d_1) \times \dotsb \times (\Integer/d_s) \,.
      \]
      Mithilfe des chinesischen Restklassensatzes lassen sich dann die Faktoren $\Integer/d_i$ noch weiter in direkte Produkte von zyklischen $p$-Gruppen zerlegen.
  \end{enumerate}
  Insbesondere lassen sich der Fundamentalsatz über endlich erzeugte abelsche Gruppen (inklusive der von uns gezeigten Aussagen über endliche abelsche Gruppen) auch rein algorithmisch begründen.
  (Zumindest die Existenz der entsprechenden Zerlegungen.)
\end{remark}



\begin{remark}
  Der Fundamentalsatz über endlich erzeugte abelsche Gruppen verallgemeinert sich weiter zum \emph{Fundamentalsatz über endlich erzeugte Moduln über Hauptidealringen} (den wir hier nicht angeben werden).
  Für den Hauptidealring $\Integer$ erhält man hieraus den Fundamentalsatz über endlich erzeugte abelsche Gruppen;
  für den Hauptidealring $K[t]$, wobei $K$ ein algebraisch abgeschlossener Körper ist, erhält man (unter anderem) die Jordan-Normalform aus der linearen Algebra.
  
  Insbesondere lassen sich die Beweise zu Aufgaben~5~und~6 in einen Beweis für die Existenz der Jordan-Normalform über algebraisch abgeschlossenen Körpern umschreiben.
  (Die Zerlegung einer endlichen abelschen Gruppe in ihre $p$-Sylowuntergruppen wird dabei durch die Zerlegung eines endlichdimensionalen Vektorraums in die verallgemeinerten Eigenräume, bzw.\ Haupträume ersetzt.)
  Hieraus ergibt sich auch, dass die Lösungen zu Aufgabe 5 nicht sehr viel besser als die Beweise für die Existenz der Jordan-Normalform seien können.
\end{remark}





\subsection{}


Ein \emph{Partition} einer Menge $X$ ist eine Kollektion von nicht-leeren Teilmenge $\mathcal{Y} \subseteq \mathcal{P}(X)$ mit $X = \bigcup_{Y \in \mathcal{Y}} Y$ und $Y_1 \cap Y_2 = \emptyset$ für alle $Y_1, Y_2 \in \mathcal{Y}$.
Partitionen von $X$ entsprechen also den Möglichkeiten, die Menge $X$ in paarweise disjunkte, nicht-leere Teilmengen zu zerlegen.
Für alle $n \geq 1$ sei $P(n)$ die Anzahl der Partitionen einer $n$-elementigen Menge.
(Es gelten etwa $P(1) = 1$, $P(2) = 2$, $P(3) = 5$ und $P(4) = 15$.)

Es sei nun $N \geq 1$ mit Primfaktorzerlegung $N = p_1 \dotsm p_n$.
Dann liefert jede Partition $\{1, \dotsc, n\} = A_1 \cup \dotsb \cup A_r$ in paarweise disjunkte Teilmengen $A_1, \dotsc, A_r \subseteq \{1, \dotsc, n\}$ eine endliche abelsche Gruppe
\[
          \left. \Integer \middle/ \left( \prod_{p \in A_1} p \right) \right.
  \times  \dotsb
  \times  \left. \Integer \middle/ \left( \prod_{p \in A_r} p \right) \right.
\]
der Ordnung $N$.
Nach den vorherigen Ergebnissen ist jede abelsche Gruppe der Ordnung $N$ isomorph zu einer Gruppe dieser Form.
Es genügt daher zu zeigen, dass
\[
  P(n) \leq n^n
\]
für alle $n \geq 1$ gilt.
Wir zeigen dies per Induktion über $n$:

Für $n = 1$ gilt $P(n) = 1 = 1^1 = n^n$.
Es sei nun $n \geq 2$ und es gelte $P(m) \leq m^m$ für alle $m = 1, \dotsc, n-1$.
Um $P(n)$ entsprechend abzuschätzen, betrachten wir alle Teilmengen $A \subseteq \{1, \dotsc, n\}$, die $1$ enthalten.  Für jedes $k \geq 1$ gibt es genau $\binom{n-1}{k-1}$ viele $k$-elementige solche Teilmengen.
Für jede solche Menge gibt es dann $P(n-k)$ viele Partitionen für das Komplement $\{1, \dotsc, n\} \setminus A$, also Möglichkeiten, dieses Komplement weiter disjunkt zu zerlegen.
Wir erhalten somit, dass
\begin{align*}
        P(n)
  &=    \sum_{k=1}^n \binom{n-1}{k-1} P(n-k)
   =    \sum_{k=0}^{n-1} \binom{n-1}{k} P(n-k-1)
  \\
  &\leq \sum_{k=0}^{n-1} \binom{n-1}{k} (n-k-1)^{(n-k-1)}
   \leq \sum_{k=0}^{n-1} \binom{n-1}{k} (n-1)^{(n-1-k)} \cdot 1^k
  \\
  &=    (n-1 + 1)^{n-1}
   =   n^{(n-1)}
   \leq  n^n \,.
\end{align*}

