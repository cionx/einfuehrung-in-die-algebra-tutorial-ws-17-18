\section{}





\addtocounter{subsection}{1}





\subsection{}

Wir haben im Tutorium gesehen, dass für $A \in \matrices{n}{K}$ die Implikationen
\begin{gather*}
            \text{$A$ ist nicht injektiv}
  \implies  \text{$A$ ist ein Linksnullteiler}
\shortintertext{und}
            \text{$A$ ist nicht surjektiv}
  \implies  \text{$A$ ist ein Rechtsnullteiler}
\end{gather*}
gelten.
Dabei handelt es sich tatsächlich schon um Äquivalenzen.
Aus der linearen Algebra wissen wir dabei, dass wegen der Endlichdimensionalität von $K^n$ die Injektivität und Surjektivität von $A$ äquivalent sind.
Deshalb kann der Matrizenring $\matrices{n}{K}$ keine Beispiele liefern.

Im Tutorium haben wir das Problem dadurch gelöst, dass wir den endlichdimensionalen $K$-Vektorraums $K^n$ durch einen unendlichdimensionalen $K$-Vektorraum $V$ ersetzt haben, und anstelle $\matrices{n}{K} \cong \End{K^n}$ den Endomorphismenring $\End{V}$ betrachtet haben.

Ein anderer Ansatz besteht darin, die Einträge der Matrizen nicht aus einem Körper $K$ zu wählen.





\addtocounter{subsection}{1}





\addtocounter{subsection}{1}





\addtocounter{subsection}{1}





\addtocounter{subsection}{1}





\addtocounter{subsection}{1}




