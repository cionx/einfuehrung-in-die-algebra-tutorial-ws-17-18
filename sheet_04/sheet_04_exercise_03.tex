\section{}





\addtocounter{subsection}{1}





\subsection{}

Wir haben im Tutorium gesehen, dass für $A \in \matrices{n}{K}$ die Implikationen
\begin{gather*}
            \text{$A$ ist nicht injektiv}
  \implies  \text{$A$ ist ein Linksnullteiler}
\shortintertext{und}
            \text{$A$ ist nicht surjektiv}
  \implies  \text{$A$ ist ein Rechtsnullteiler}
\end{gather*}
gelten.
Dabei handelt es sich tatsächlich schon um Äquivalenzen.
Aus der linearen Algebra wissen wir dabei, dass wegen der Endlichdimensionalität von $K^n$ die Injektivität und Surjektivität von $A$ äquivalent sind.
Deshalb kann der Matrizenring $\matrices{n}{K}$ keine Beispiele liefern.

Im Tutorium haben wir dieses Problem dadurch gelöst, dass wir den endlichdimensionalen $K$-Vektorraums $K^n$ durch einen unendlichdimensionalen $K$-Vektorraum $V$ ersetzt haben, und anstelle von $\matrices{n}{K} \cong \End{K^n}$ den Endomorphismenring $\End{V}$ betrachtet haben.

Ein anderer Ansatz besteht darin, die Einträge der Matrizen nicht aus einem Körper $K$ zu wählen:

\begin{itemize}
  \item
    Ein erster Ansatz besteht darin, anstelle eines Körpers $K$ einen passenden kommutativen Ring $R$ zu betrachten, und dann im Matrizenring $\matrices{n}{R}$ nach Beispielen zu suchen.
    Aufgrund der folgenden Resultate aus der kommutativen Algebra wird dies allerdings nicht zum Erfolg führen:
    
    \begin{lemma}
      Ist $R$ ein kommmutativer Ring, so gilt für $A \in \matrices{n}{R}$ genau dann $\ker A \neq 0$, wenn $\det A \in R$ ein Nullteiler ist.
    \end{lemma}

    \begin{corollary}
      Ist $R$ ein kommutativer Ring, so sind für $A \in \matrices{n}{R}$ die folgenden Bedingungen äquivalent:
      \begin{enumerate}[leftmargin=*, label=\roman*)]
        \item
          Die Matrix $A$ ist ein Linksnullteiler in $\matrices{n}{R}$.
        \item
          Der Skalar $\det A$ ist ein Nullteiler in $R$.
        \item
          Die Matrix $A$ ist ein Rechtsnullteiler in $\matrices{n}{R}$.
      \end{enumerate}
    \end{corollary}
  
  \item
    Der obige Ansatz lässt sich dadurch reparieren, dass man die Matrixeinträge aus verschiedenen kommutativen Ringen wählt.
    So kann man etwa
    \[
        R
      = \begin{pmatrix}
          \Integer  & \Integer/2 \\
          0         & \Integer
        \end{pmatrix}
      = \left\{
          \begin{pmatrix}
            a & b \\
            0 & c
          \end{pmatrix}
        \suchthat*
          a, c \in \Integer,
          b \in \Integer/2
        \right\}
    \]
    mit der Addition
    \[
      \begin{pmatrix}
        a & b \\
        0 & c
      \end{pmatrix}
      +
      \begin{pmatrix}
        a'  & b'  \\
        0   & c'
      \end{pmatrix}
      =
      \begin{pmatrix}
        a+a'  & b+b'  \\
        0     & c+c'
      \end{pmatrix}
    \]
    und der Multiplikation
    \[
      \begin{pmatrix}
        a & \class{b} \\
        0 & c
      \end{pmatrix}
      \cdot
      \begin{pmatrix}
        a'  & \class{b'}  \\
        0   & c'
      \end{pmatrix}
      =
      \begin{pmatrix}
        aa' & \class{a b' + b c'}  \\
        0   & cc'
      \end{pmatrix}
    \]
    betrachten.
    Dass $R$ mit den oberen Operationen tatsächlich einen Ring bildet, erkennt man durch direktes Nachrechnen;
    wir werden auch später noch ein weiteres Argument hierfür sehen.
    In diesem Ring lässt sich dann die Matrix
    \[
                A
      \coloneqq \begin{pmatrix}
                  2 & 0 \\
                  0 & 1
                \end{pmatrix}
      \in       R
    \]
    betrachten.
    Es gilt
    \[
        \begin{pmatrix}
          2 & 0 \\
          0 & 1
        \end{pmatrix}
        \begin{pmatrix}
          0 & \class{1} \\
          0 & 0
        \end{pmatrix}
      = \begin{pmatrix}
          0 & \class{2} \\
          0 & 0
        \end{pmatrix}
      = 0,
    \]
    weshalb $A$ ein Linknullteiler ist.
    Andererseits gilt
    \[
        \begin{pmatrix}
          a & b \\
          0 & c
        \end{pmatrix}
        \begin{pmatrix}
          2 & 0 \\
          0 & 1
        \end{pmatrix}
      = \begin{pmatrix}
          2a  & b \\
              & c
        \end{pmatrix};
    \]
    wobei genau dann $2a = 0$, wenn $a = 0$.
    Deshalb ist $A$ kein Rechtsnullteiler.
    
  \item
    Im Tutorium kam die Frage auf, ob es auch Beispiele in passenden endlichen Ringen gibt.
    Wie sich herausstellt\footnote{Siehe \url{https://math.stackexchange.com/a/45220/300783}.}, ist dies nicht möglich:
    
    \begin{lemma}
      Ist $R$ ein endlicher \textup(nicht notwendigerweise kommutativer\textup) Ring und $r \in R$ kein Links-, bzw.\ Rechtsnullteiler, so ist $r$ eine Einheit.
    \end{lemma}
    
    \begin{proof}
      Wir betrachten den Fall, dass $r$ kein Linksnullteiler ist;
      der Fall, dass $r$ kein Rechtsnullteiler ist, verläuft analog.
      
      Nach Annahme ist dann die Abbildung
      \[
                \pi_r
        \colon  R
        \to     R,
        \quad   a
        \mapsto ra
      \]
      injektiv, und wegen der Endlichkeit von $R$ somit bereits bijektiv.
      Die Bjiektion $\pi_r$ ist also ein Element der symmetrischen Gruppe
      \[
          S(R)
        = \{
            \pi \colon R \to R
          \suchthat
          \text{$\pi$ ist eine Bijektion}
          \}.
      \]
      Die Gruppe $S(R)$ ist endlich, da $R$ endlich ist.
      Also hat $\pi_r$ endliche Ordnung, d.h.\ es gibt $n \geq 1$ mit $\pi_r^n = \id_R$.
      Dabei ist die Abbildung $\pi_r^n$ ist durch Multiplikation mit $r^n$ gegeben.
      Also ist $r^n$ linksneutral bezüglich der Multiplikation, und somit bereits $r^n = 1$.
      Folglich ist $r$ eine Einheit mit $r^{-1} = r^{n-1}$.
    \end{proof}

    \begin{corollary}
      Ist $R$ ein endlicher \textup(nicht notwendigerweise kommutativer\textup) Ring, so ist jedes Element $r \in R$ entweder eine Einheit oder ein beidseitiger Nullteiler.
    \end{corollary}
    
\end{itemize}





\addtocounter{subsection}{1}





\addtocounter{subsection}{1}





\addtocounter{subsection}{1}





\addtocounter{subsection}{1}





\addtocounter{subsection}{1}




