\section{}





\subsection{}

Durch Reduzieren bezüglich der Primzahl $p = 2$ erhalten wir das Polynom
\[
      \induced{f}(t)
  =   t^3 + t + 1
  \in \Finite{2}[t] \,.
\]
Dies ist ein Polynom von Grad $3$ über dem Körper $\Finite{2}$, dass keine Nullstellen (in $\Finite{2}$) hat.
Folglich ist $\induced{f}(t)$ irreduzibel.
Damit ist nach dem Reduktionskriterium auch $f(t)$ irreduzibel, da $f(t)$ primitiv ist (und somit normiert).

Der Körper $\Rational(t)$ ist perfekt, da $\ringchar \Rational = 0$ gilt.
Das irreduzible Polynom $f(t) \in \Rational[t]$ ist deshalb separabel, d.h.\ $f(t)$ hat keine mehrfache Nullstelle in $L$.

\begin{remark}
  Mit etwas mit Hintergrundswissen, als aus der Vorlesung bekannt ist, ließe sich auch wie folgt argumentieren:
  
  Da $\Rational$ ein Körper ist, genügt es auch zu zeigen, dass $f(t)$ keine rationale Nullstelle besitzt.
  
  Da $f(t)$ ein normiertes Polynom mit ganzzahligen Koeffizienten ist, lässt sich zeigen, dass jede rationale Nullstelle von $f(t)$ bereits ganzzahlig ist (dies ist eine nicht-triviale Aussage, die \emph{nicht} aus der Vorlesung bekannt ist).
  Damit genügt es dann zu zeigen, dass $f(t)$ keine ganzzahlige Nullstelle hat.
  Jede konstante Nullstelle von $f(t)$ muss den konstanten Koeffizienten von $f(t)$ teilen, also ein Teiler von $-1$ sein.
  Somit sind $1$ und $-1$ die einzigen beiden möglichen rationalen Nullstellen von $f(t)$.
  
  Da aber $f(1) = -3$ und $f(-1) = 1$ gelten, hat $f(t)$ somit keine rationale Nullstelle, ist also irreduzibel.
\end{remark}






\subsection{}

Es seien $x_1, x_2, x_3 \in L$ die drei paarweise verschiedenen Nullstellen von $f(t)$ in $L$.

Die Galoisgruppe $\Gal(L/\Rational)$ wirkt auf der Menge der Wurzeln $\{x_1, x_2, x_3\}$, und diese Wirkung entspricht einem Gruppenhomomorphismus $\alpha \colon \Gal(L/\Rational) \to S(\{x_1, x_2, x_3\})$.
Indem wir eine Bijektion zwischen den Mengen $\{x_1, \dotsc, x_n\}$ und $\{1, 2, 3\}$ wählen (etwa vermöge der Abbildung $x_i \mapsto i$) erhalten wir einen induzierten Gruppenisomorphismus $\varphi \colon S(\{x_1, x_2, x_3\}) \to S_3$.
Damit erhalten wir insgesamt einen Grupenhomomorphismus $\beta \define \varphi \circ \alpha \colon \Gal(L/\Rational) \to S_3$.

Die Wirkung der Galoisgruppe $\Gal(L/\Rational)$ auf der Menge der Nullstellen $\{x_1, x_2, x_3\}$ ist treu, der Gruppenhomomorphismus $\alpha$ also injektiv.
Also ist auch $\beta$ injektiv.
Außerdem ist die Wirkung von $\Gal(L/\Rational)$ auf $\{x_1, x_2, x_3\}$ transitiv, da $f(t)$ irreduzibel ist;
deshalb ist auch die Wirkung von $\im \beta$ auf der Menge $\{1, 2, 3\}$ transitiv.

\begin{lemma}
  Die Untergruppen der symmetrischen Gruppe $S_3$ sind die triviale Gruppe $1$, die zweielementigen Gruppen $\generated{(1\;2)}$, $\generated{(1\;3)}$, $\generated{(2\;3)}$, sowie die beiden Gruppen $A_3$ und $S_3$.
\end{lemma}

\begin{proof}
  Ist $H < G$ eine echte Untergruppe, so muss $\ord{H}$ ein echter Teiler von $\ord{S_3} = 6$ sein, also $\ord{H} \in \{1, 2, 3\}$ gelten.
  Inbesondere muss $H$ zyklisch sein.
  Damit ergeben sich die echten Untergruppen $\generated{1} = 1$, $\generated{(1\;2)}$, $\generated{(1\;3)}$, $\generated{(2\;3)}$ und $\generated{(1\;2\;3)} = \generated{(1\;3\;2)} = A_3$.
\end{proof}

Die einzigen Untergruppen von $S_3$, die transitiv auf $\{1, 2, 3\}$ wirken, sind also $A_3$ und $S_3$.
Somit gilt $\im \beta = A_3$ oder $\im \beta = S_3$.

\begin{remark}
  Eine Untergruppe $H \subgroup S_n$, die transitiv auf der Menge $\{1, \dotsc, n\}$ wirkt (bezüglich der Einschränkung der kanonischen Wirkung von $S_n$ auf $\{1, \dotsc, n\}$) ist eine \emph{transitive Untergruppe}.
\end{remark}





\subsection{}

Wir geben mehrere mögliche Vorgehensweisen an:

\begin{itemize}
  \item
    Zunächst rechnen wir nach, dass $a^2 - a - 2$ nd $2 - a^2$ Nullstellen von $f(t)$ sind.
    \begin{itemize}
      \item
        Hierfür setzen wir beide Werte in das Polynom $f(t)$ ein und Vereinfachen anschließend dei Ergebnisse mithilfe der Identität $a^3 = 3a + 1$ (die eine Umformulierung der Annahme $0 = f(a) = a^3 - 3a + 1$ ist):
        Es gilt
        \begin{align*}
           &\,  f(a^2 - a - 2)  \\
          =&\,  (a^2 - a - 2)^3 - 3(a^2 - a - 2) - 1  \\
          =&\,  a^6 - 3a^5 - 3a^4 + 11a^3 + 3a^2 - 9a - 3 \\
          =&\,  a^3(3a + 1) - 3a^2(3a + 1) - 3a(3a + 1) + 11(3a + 1) + 3a^2 - 9a - 3  \\
          =&\,  3a^4 - 8a^3 - 9 a^2 + 21a + 8 \\
          =&\,  3a(3a+1) - 8(3a+1) - 9a^2 + 21a + 8
          = \,  0 \,,
        \end{align*}
        und es gilt
        \begin{align*}
                f(2 - a^2)
          &=   (2 - a^2)^3 - 3(2 - a^2) - 1
            =   -a^6 + 6a^4 - 9a^2 + 1  \\
          &=   -a^3(3a+1) + 6a^1(3a+1) - 9a^2 + 1
            =   -3a^4 - a^3 + 9a^2 + 6a + 1 \\
          &=   -3a(3a+1) - (3a+1) + 9a^2 + 6a + 1
            =   0 \,.
        \end{align*}
      \item
        Mithilfe von Polynomdivison erhalten wir, dass
        \begin{align*}
              f(a^2 - a - 2)
          &=  a^6 - 3a^5 - 3a^4 + 11a^3 + 3a^2 - 9a - 3 \\
          &=  (a^3 - 3a^2 + 3)\underbrace{(a^3 - 3a - 1)}_{=0}
          =   0 \,,
        \end{align*}
        und dass
        \[
              f(a^2 - a - 2)
          =  -a^6 + 6a^4 - 9a^2 + 1
          =  (-a^3 + 3a - 1)\underbrace{(a^3 - 3a - 1)}_{=0}
          =   0 \,.
        \]
    \end{itemize}
    Es bleibt zu zeigen, dass die Nullstellen $a$, $a^2 - a - 2$ und $2 - a^2$ paarweise verschieden sind, dass es sich also tatsächlich um alle drei Nullstellen von $a$ handelt.
    Dies ergibt sich daraus, dass $f(t)$ das Minimalpolynom von $a$ ist (denn $f(t)$ ist irreduzibel mit $f(a) = 0$), und somit die Potenzen $1, a, a^2$ linear unabhängig über $\Rational$ sind.
    
  \item
    Es gilt
    \begin{align*}
       &\,  (t - a)(t - (a^2 - a - 2))(t - (2 - a^2)) \\
      =&\,  t^3 + (-a^4 + a^3 + 3a^2 - 2a - 4) t + (a^5 - a^4 - 4a^3 + 2a^2 + 4a)
    \end{align*}
    Analog zu den obigen Rechnungen lassen sich die Koeffizienten vereinfachen, wodurch sich das Polynom $f(t)$ ergibt.
\end{itemize}

Wir erhalten nun, dass die einfache Körpererweiterung $\Rational(a)$ bereits alle Nullstellen des Polynoms $p(t)$ erhalten.
Somit ist $\Rational(a)$ bereits ein Zerfälllungskörper von $f(t)$, weshalb bereits $L = \Rational(a)$ gilt.

Die Körpererweiterung $L/\Rational$ ist galoissch, da $L$ ein Zerfällungskörper des separablen Polynoms $f(t) \in \Rational[t]$ ist.
Damit erhalten wir, dass
\[
    |{\Gal(L/\Rational)}|
  = [L : \Rational]
  = [\Rational(a) : \Rational]
  = \deg m_a(t)
  = \deg f(t)
  = 3 \,.
\]
Damit erhalten wir, dass es sich bei $\Gal(L/\Rational)$ nicht um $S_3$ handelt, sondern um $A_3$.





\subsection{}

Da die Erweiterung $L/\Rational$ galoissch ist, entsprechen die Zwischenkörper $\Rational \subseteq K \subseteq \Rational$ in bijektiver Weise den Zwischengruppen $1 \subgroup H \subgroup \Gal(L/\Rational)$, also Untergruppen von $\Gal(L/\Rational)$.
Da $\Gal(L/\Rational) \cong A_3 \cong \Integer/3$ gilt, sind $1$ und $\Gal(L/\Rational)$ die einzigen beiden Untergruppen von $\Gal(L/\Rational)$.
Folglich sind $\Rational$ und $L$ die einzigen beiden Zwischenkörper von $L$.




