\section{}





\subsection{}

Die Aussage ist \emph{wahr}:

Da jedes Element $x \in M$ separabel über $K$ ist, gilt dies insbesondere für jedes $x \in L$.
Also ist auch die Erweiterung $L/K$ separabel.

Jedes $x \in M$ ist Nullstelle eines separablen Polynoms $f(t) \in K[t]$, da $M/K$ separabel ist.
Indem wir $f(t)$ als ein Polynom $f(t) \in L[t]$ auffassen, ist $x$ dann auch Nullstelle eines separablen Polynoms $f(t) \in L[t]$.
Also ist auch die Erweiterung $M/L$ separabel.





\subsection{}

Die Aussage ist \emph{wahr}, denn die Erweiterung $\Rational(\pi,i)/\Rational(\pi)$ ist normal und separabel:

Da $\ringchar \Rational(\pi) = 0$ gilt, ist der Körper $\Rational(\pi)$ perfekt, und die Körpererweiterung $\Rational(\pi,i)/\Rational(\pi)$ somit separabel.

Das Polynom $t^2 + 1 \in \Rational(\pi)$ zerfällt über der Körperweiterung $\Complex/\Rational(\pi)$ in die Linearfaktoren $(t-i)(t+i)$, weshalb $\Rational(\pi,i) = \Rational(\pi)(i,-i)$ der Zerfällungskörper von $t^2 + 1$ in $\Complex/\Rational(\pi)$ ist.
Somit ist $\Rational(\pi,i)/\Rational(\pi)$ ein Zerfällungskörper von $t^2 + 1$, die Erweiterung also normal.





\subsection{}

Die Aussage ist \emph{falsch}:
Man betrachte etwa die Körpererweiterung $L/\Rational$ aus Aufgabe~1:
Dort gilt $[L : \Rational] = 3$, aber die Erweiterung $L/\Rational$ hat nur zwei Zwischenkörper.





\subsection{}

Die Aussage ist \emph{falsch}:

Ist $K$ ein beliebiger Körper, so ist die Erweiterung $K/K$ eine endliche Galoiserweiterung, so dass es keine echten Zwischenkörper $K \subsetneq L \subsetneq K$ gibt.
Aber $[K : K] = 1$ ist nicht prim.

\begin{remark}
  Schließt man den Fall $L = K$ aus, so \emph{stimmt} die Aussage:
  Dann ist $G \define \Gal(L/\Rational)$ eine endliche Gruppe, und nach dem Hauptsatz der Galoistheorie entsprechen die Zwischenkörper $K \subseteq M \subseteq L$ bijektiv den Untergruppen der Galoisgruppe $G$.
  Die Gruppe $G$ ist nicht-trivial, da $|G| = [L:K] > 1$ gilt, und nach Annahme besitzt $G$ keine Untergruppen außer $1$ und $G$.
  \begin{claim*}
    Es gilt $G \cong \Integer/p$ mit $p$ prim.
  \end{claim*}
  \begin{proof}
    Für $g \in G$ mit $g \neq 1$ ist die Untergruppe $\generated{g}$ eine nicht-triviale Untergruppe von $G$, und somit nach Annahme $G = \generated{g}$.
    Also ist die nicht-trivale Gruppe $G$ zyklisch, und somit gilt $G \cong \Integer/n$ für ein $n \geq 2$.
    Ist $n$ nicht prim, so besitzt $\Integer/n$ nicht-triviale echte Untergruppen.
  \end{proof}
  Somit ist $[L : K] = |G| = p$ prim, da $L/K$ galoissch ist.
\end{remark}






