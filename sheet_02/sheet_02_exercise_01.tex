\section{}





\subsection{}

Die Identität $\id \colon G \to G$ ist ein Gruppenhomomorphismus mit $\ker \id = \{e\}$.
Also ist die Untergruppe $\{e\}$ normal mit $G/\{e\} \cong \im \id = G$.





\subsection{}

Wie auf dem ersten Übungszettel gesehen, ist für die Gruppe der Einheitswurzeln,
\[
            \mu_\infty(\Complex)
  \coloneqq \{ z \in \Complex \suchthat z^n = 1 \},
\]
die Abbildung
\[
          f
  \colon  \Rational
  \to     \mu_\infty(\Complex),
  \quad   x
  \mapsto e^{2 \pi i x}
\]
ein surjektiver Gruppenhomomorphismus mit $\ker f = \Integer$.
Folglich ist $\Integer$ eine normale Untergruppe von $\Rational$ (dies ergibt sich auch daraus, dass $\Rational$ abelsch ist), und $f$ induziert einen Isomorphismus
\[
          \induced{f}
  \colon  \Rational/\Integer
  \to     \mu_\infty(\Complex),
  \quad   \class{x}
  \mapsto f(x)
  =       e^{2 \pi i x}.
\]
Inbesondere gilt $\Rational/\Integer \cong \mu_\infty(\Complex)$.





\subsection{}

Die Abbildung $\Im \colon \Complex \to \Real$, $z = x + iy \mapsto y$ ist ein surjektiver Gruppenhomomorphismus mit $\ker \Im = \Real$.
Folglich ist $\Real$ eine normale Untergruppe von $\Complex$ (diese ergibt sich auch daraus, dass $\Complex$ abelsch ist), und $\Im$ induziert einen Isomorphismus
\[
          \induced{\Im}
  \colon  \Complex/\Real
  \to     \Real,
  \quad   \class{z}
  \mapsto \Im z.
\]





\subsection{}

Die beiden Abbildungen
\begin{gather*}
          \mult{\Complex}
  \to     S^1,
  \quad   z
  \mapsto \frac{z}{\abs{z}}
\shortintertext{und}
          S^1
  \to     S^1,
  \quad   z
  \mapsto z^2
\end{gather*}
sind surjektive Gruppenhomomorphismen.
Deshalb ist auch ihre Verknüpfung, also die Abbildung
\[
          f
  \colon  \mult{\Complex}
  \to     S^1,
  \quad   z
  \mapsto \left( \frac{z}{\abs{z}} \right)^2
\]
ist ein surjektiver Gruppenhomomorphismus.
Für $z \in \mult{\Complex}$ gilt dabei, dass
\[
        f(z) = 1
  \iff  \left( \frac{z}{\abs{z}} \right)^2 = 1
  \iff  \frac{z}{\abs{z}} = \pm 1
  \iff  z = \pm \abs{z}
  \iff  z \in \Real,
\]
weshalb $\ker f = \mult{\Real}$.





\subsection{}

Die Untergruppe $\sorthogonal{2} \subgroup \sorthogonal{3}$ ist nicht normal, denn für
\[
            R
  \coloneqq \begin{pmatrix*}[r]
              0 & -1  & 0 \\
              1 &  0  & 0 \\
              0 &  0  & 1
            \end{pmatrix*}
  \in       \sorthogonal{2}
  \quad\text{and}\quad
            S
  \coloneqq \begin{pmatrix}
              0 & 0 & 1 \\
              1 & 0 & 0 \\
              0 & 1 & 0
            \end{pmatrix}
  \in       \sorthogonal{3}
\]
gilt
\begin{align*}
          S R S^{-1}
  &=      \begin{pmatrix}
            0 & 0 & 1 \\
            1 & 0 & 0 \\
            0 & 1 & 0
          \end{pmatrix}
          \begin{pmatrix*}[r]
            0 & -1  & 0 \\
            1 &  0  & 0 \\
            0 &  0  & 1
          \end{pmatrix*}
          \begin{pmatrix}
            0 & 0 & 1 \\
            1 & 0 & 0 \\
            0 & 1 & 0
          \end{pmatrix}^{-1}
  \\
  &=      \begin{pmatrix}
            0 & 0 & 1 \\
            1 & 0 & 0 \\
            0 & 1 & 0
          \end{pmatrix}
          \begin{pmatrix*}[r]
            0 & -1  & 0 \\
            1 &  0  & 0 \\
            0 &  0  & 1
          \end{pmatrix*}
          \begin{pmatrix}
            0 & 1 & 0 \\
            0 & 0 & 1 \\
            1 & 0 & 0
          \end{pmatrix}
   =      \begin{pmatrix*}[r]
            1 & 0 &  0  \\
            0 & 0 & -1  \\
            0 & 1 &  0
          \end{pmatrix*}
   \notin \sorthogonal{2}
\end{align*}
(Anschaulich gesehen ist $R$ die Drehung um die $z$-Achse zum $90^\circ$, während $S$ eine Rotation ist, welche die drei Koordinatenachsen in der Reihenfolge $x \to y \to z \to x$ zyklisch permutiert.
Vertauscht man die Achsen $z \to x$, so wird die Drehung um die $z$-Achse zur Drehung um die $x$-Achse.)

\begin{remark}
  Es sei $S^2 \coloneqq \{ x \in \Real^3 \suchthat \abs{x} = 1 \}$ die zweidimensionale Sphäre.
  Die orthogonale Gruppe $\sorthogonal{3}$ wirkt auf $S^2$ durch
  \[
              R.x
    \coloneqq Rx
    \qquad
    \text{für alle $S \in \sorthogonal{3}$, $x \in S^2$}.
  \]
  Diese Wirkung ist transitiv, d.h.\ für alle $x, y \in S^2$ gibt es eine Rotation $R \in \sorthogonal{3}$ mit $R.x = y$.
  Der Stabilisator $\stab{\sorthogonal{3}}{e_3}$ aus all jeden Rotation, welche die $z$-Achse fixieren.
  Es ist also $\stab{\sorthogonal{3}}{e_3} = \sorthogonal{2}$.
  Wir erhalten somit eine Bijektion
  \[
            \sorthogonal{3}/\sorthogonal{2}
    \to     S^2,
    \quad   \class{R}
    \mapsto R.e_3
    =       R e_3.
  \]
  Wir können uns $\sorthogonal{3}/\sorthogonal{2}$ also als die zweidimensionale Sphäre vorstellen.
  Allgemeiner gilt für alle $n \geq 1$, dass $\sorthogonal{n+1}/\sorthogonal{n} \cong S^n$ als topologische Räume.
\end{remark}





\subsection{}

Die Untergruppe $\orthogonal{2} \subgroup \GL{2}{\Real}$ ist nicht normal, denn für
\[
    S
  = \begin{pmatrix}
      0 & 1 \\
      1 & 0
    \end{pmatrix}
  \in \orthogonal{2}
  \quad\text{und}\quad
    T
  = \begin{pmatrix}
      1 & 1 \\
      0 & 1
    \end{pmatrix}
\]
gilt
\begin{align*}
      T S T^{-1}
  &=  \begin{pmatrix}
        1 & 1 \\
        0 & 1
      \end{pmatrix}
      \begin{pmatrix}
        0 & 1 \\
        1 & 0
      \end{pmatrix}
      \begin{pmatrix}
        1 & 1 \\
        0 & 1
      \end{pmatrix}^{-1}
  \\
  &=  \begin{pmatrix}
        1 & 1 \\
        0 & 1
      \end{pmatrix}
      \begin{pmatrix}
        0 & 1 \\
        1 & 0
      \end{pmatrix}
      \begin{pmatrix*}[r]
        1 & -1 \\
        0 &  1
      \end{pmatrix*}
  =   \begin{pmatrix*}[r]
        1 &  0  \\
        1 & -1
      \end{pmatrix*}
  \notin \orthogonal{2}
\end{align*}

(Anschaulich gesehen ist $S$ die Spiegelung an der Winkelhalbierenden, d.h.\ $S$ vertauscht die beiden Standardbasisvektoren $e_1$ und $e_2$.
Die Matrix $T$ erfüllt $Te_1 = e_1$ und $Te_2 = e_1 + e_2$.
Die obige Komposition vertauscht deshalb die beiden Vektoren $e_1$ und $e_1 + e_2$, und ist somit nicht längenerhaltend.)





\subsection{}

Die Abbildung $\det \colon \orthogonal{n} \to \Real^\times$ ist wegen der Multiplikativität der Determinante ein Gruppenhomomorphismus, und es gilt $\ker \orthogonal{n} = \sorthogonal{n}$.
Für jede orthogonale Matrix $S \in \orthogonal{n}$ gilt $S^2 = \Id$ und somit
\[
    1
  = \det(\Id)
  = \det(S^2)
  = \det(S)^2,
\]
also $\det(S) = \pm 1$.
Andererseits gilt für
\[
            S_{\pm}
  \coloneqq \begin{pmatrix}
              \pm 1 &   &         &     \\
                    & 1 &         &     \\
                    &   & \ddots  &     \\
                    &   &         & 1
            \end{pmatrix}
  \in       \orthogonal{n}
\]
dass $\det S_{\pm} = \pm 1$.
Somit gilt $\im \restrict{\det}{\orthogonal{n}} = \{1, -1\}$.
Also induziert $\restrict{\det}{\orthogonal{n}}$ einen Isomorphismus
\[
          \orthogonal{n} / \sorthogonal{n}
  \to     \im \restrict{\det}{\orthogonal{n}},
  \quad   \class{S}
  \mapsto \det S.
\]
Inbesondere gilt $\orthogonal{n} / \sorthogonal{n} \cong \im \restrict{\det}{\orthogonal{n}} \cong \Integer/2$.





\subsection{}

Für $n = 1$ ist $S_1$ trivial, und somit $S_1 \leq WB_1$ normal.
Für $n \geq 2$ ist $S_n \leq WB_n$ nicht normal:
Wir betrachten die Permutation
\[
            \tau
  \coloneqq \begin{pmatrix}
              1 & 2 & 3 & \dotsm & n  \\
              2 & 1 & 3 & \dotsm & n
            \end{pmatrix}
  \in       S_n
\]
und die Vorzeichen-Permutation
\[
            \pi
  \coloneqq \begin{pmatrix*}[r]
               1 & 2 & \dotsm & n  \\
              -1 & 2 & \dotsm & n
            \end{pmatrix*}
  \in       WB_n.
\]
(D.h.\ es gilt $\pi(\pm 1) = \mp 1$ und $\pi(\pm i) = \pm i$ für alle $2 \leq i \leq n$.)
Dann gilt
\[
    \pi \tau \pi^{-1}
  = \pi \tau \pi
  = \begin{pmatrix*}[r]
       1 &  2 & 3 & \dotsm & n  \\
      -2 & -1 & 3 & \dotsm & n
    \end{pmatrix*}
  \notin S_n.
\]





\subsection{}

Die Untergruppe $\GL{2}{\Real} \subgroup \GL{2}{\Complex}$ is nicht normal, denn für
\[
            A
  \coloneqq \begin{pmatrix}
              0 & 1 \\
              1 & 0
            \end{pmatrix}
  \in       \GL{2}{\Real}
  \quad\text{und}\quad
            S
  \coloneqq \begin{pmatrix}
              i & 0 \\
              0 & 1
            \end{pmatrix}
  \in       \GL{2}{\Complex}
\]
gilt
\[
    S A S^{-1}
  = \begin{pmatrix}
      i & 0 \\
      0 & 1
    \end{pmatrix}
    \begin{pmatrix}
      0 & 1 \\
      1 & 0
    \end{pmatrix}
    \begin{pmatrix*}[r]
      -i  & 0 \\
       0  & 1
    \end{pmatrix*}
  = \begin{pmatrix*}[r]
       0  & i \\
      -i  & 0
    \end{pmatrix*}
  \notin \GL{2}{\Complex}
\]
