\section{}





\subsection{}

\begin{remark}
  Für Permuationen in Zykelschreibe gelten die folgenden nützlichen Rechenregeln:
  \begin{itemize}
    \item
      Für alle paarweise verschiedenen Zahlen $1 \leq a_1, \dotsc, a_r \leq n$ gilt
      \[
          (a_1 \; \dotsm \; a_r)
        = (a_1 \; a_2) \dotsm (a_{r-1} \; a_r).
      \]
    \item
      Für alle paarweise verschiedenen Zahlen $1 \leq a_1, \dotsc, a_r \leq n$ und jede Permutation $\pi \in S_n$ gilt
      \[
          \pi (a_1 \; \dotsm \; a_r) \pi^{-1}
        = ( \pi(a_1) \; \dotsm \; \pi(a_r) ).
      \]
  \end{itemize}
\end{remark}

Für $n = 1$ ist $S_1$ trivial, und $S_1 \subgroup WB_1$ somit normal.
Inbesondere ist die angegebene Verknüpfung dann wohldefiniert.
Wir betrachten daher im Folgenden nur den Fall $n \geq 2$.

Wäre die angegebene binäre Verknüpfung wohldefiniert, so würde für die Transposition $\tau \coloneqq (1 \; (n+1)) \in S_{n+1}$ gelten, dass
\[
    \tau S_n
  = (\id \tau) S_n
  = \id S_n \cdot \tau S_n
  = (1 \; 2) S_n \cdot \tau S_n
  = (1 \; 2)\tau S_n.
\]
Dies gilt aber nicht, da
\[
          \tau^{-1} (1 \; 2) \tau
  =       ( \tau^{-1}(1) \; \tau^{-1}(2) )
  =       (2 \; (n+1))
  \notin  S_n.
\]
Wäre $S_n$ normal in $S_{n+1}$, so wäre die angegebene Verknüpfung wohldefiniert;
folglich kann $S_n$ nicht normal in $S_{n+1}$ sein.

\begin{remark}
  Ist allgemeiner $G$ ein Gruppe und $H \subgroup G$ eine nicht-normale Untergruppe, so gibt es $h \in H$ und $g \in G$ mit $g^{-1} h g \notin H$.
  Führt man dann in der obigen Rechnung die Ersetzungen
  \[
    S_{n+1}   \to G,
    \quad
    S_n       \to H,
    \quad
    \tau      \to g,
    \quad
    (1 \; 2)  \to h,
    \quad
    \id       \to 1_G
  \]
  durch, so ergibt sich, dass die binäre Verknüpfung
  \[
            (G/H) \times (G/H)
    \to     G/H,
    \quad   (\class{g_1}, \class{g_2})
    \mapsto \class{g_1 g_2}
  \]
  nicht wohldefiniert ist.
  Die Verknüpfung ist also \emph{genau dann} wohldefiniert, wenn $H$ normal in $G$ ist.
\end{remark}





\subsection{}
\label{subsection: subgroups of quotient}

Es seien
\begin{gather*}
            \Gsubs
  \coloneqq \{
              U
            \suchthat
              K \subgroup U \subgroup G
            \}
  \quad\text{und}\quad
            \Hsubs
  \coloneqq \{
              W
            \suchthat
              W \subgroup H,
            \}
\shortintertext{sowie}
            \Gnorms
  \coloneqq \{
              U
            \suchthat
              K \subgroup U \nsubgroup G
            \}
  \quad\text{und}\quad
            \Hnorms
  \coloneqq \{
              W
            \suchthat
              W \nsubgroup H
            \}.
\end{gather*}



\subsubsection*{Alle Untergruppen}

Es gilt zunächst zu zeigen, dass die Abbildung
\[
          \Phi
  \colon  \Gsubs
  \to     \Hsubs,
  \quad   U
  \mapsto \varphi(U)
\]
eine wohldefinierte Bijektion ist.

Die Abbildung $\Phi$ ist wohldefiniert, denn für jede Untergruppe $U \subgroup G$ ist die Einschränkung $\restrict{\varphi}{U}$ ebenfalls ein Gruppenhomomorphismus, und somit $\varphi(U) = \im \restrict{\varphi}{U}$ eine Untergruppe von $H$.

Zum Nachweis der Bijektivität betrachten wir die Abbildung
\[
          \Phi'
  \colon  \Hsubs
  \to     \Gsubs,
  \quad   W
  \mapsto \preim{\varphi}{W}
\]
Die Abbildung $\Phi'$ ist wohldefiniert:

\begin{lemma}
  \label{lemma: preimages of subgroups}
  Für jeden Gruppenhomomorphismus $\varphi \colon G \to H$ und jede Untergruppe $W \subgroup H$ ist das Urbild $\preim{\varphi}{W}$ eine Untergruppe von $G$.
\end{lemma}

\begin{proof}
  Wir geben zwei mögliche Beweise:
  \begin{itemize}
    \item
      Es gilt $\varphi(1) = 1 \in W$ und somit $1 \in \preim{\varphi}{W}$.
      Für alle $g_1, g_2 \in \preim{\varphi}{W}$ gilt $\varphi(g_1), \varphi(g_2) \in W$ und somit auch $\varphi(g_1 g_2) = \varphi(g_1)\varphi(g_2) \in W$, also $g_1 g_2 \in \preim{\varphi}{W}$.
      Für jedes $g \in \preim{\varphi}{W}$ gilt $\varphi(g) \in W$ und somit auch $\varphi(g^{-1}) = \varphi(g)^{-1} \in W$, also $g^{-1} \in \preim{\varphi}{W}$.
    
    \item
      Die Gruppe $H$ wirkt auf $H/W$ durch
      \[
                  h.\class{h'}
        \coloneqq \class{hh'}
        \quad     \text{für alle $h \in H$, $\class{h'} \in H/W$},
      \]
      und es gilt
      \[
          \stab{H}{\class{1}}
        = \{ h \in H \suchthat h.\class{1} = \class{1} \}
        = \{ h \in H \suchthat \class{h} = \class{1} \}
        = \{ h \in H \suchthat h \in W \}
        = W.
      \]
      Über $\varphi$ lässt sich diese Gruppenwirkung zu einer Wirkung von $G$ auf $H/W$ mit
      \[
                  g.\class{h}
        \coloneqq \varphi(g).\class{h}
        =         \class{\varphi(g)h}
        \quad     \text{für alle $g \in G$, $\class{h} \in H/W$}
      \]
      zurückziehen.
      Dies lässt sich auf (mindestens) zwei Weisen sehen:
      
      \begin{itemize}
        \item
          Es gilt $1_G.\class{h} = \varphi(1_G).\class{h} = 1_H.\class{h} = \class{h}$, und für alle $g_1, g_2 \in G$ gilt
          \[
              g_1.(g_2.\class{h})
            = \varphi(g_1).(\varphi(g_2).\class{h})
            = (\varphi(g_1) \varphi(g_2)).\class{h}
            = \varphi(g_1 g_2).\class{h}
            = (g_1 g_2).\class{h}.
          \]
        \item
          Die Gruppenwirkung von $H$ auf $H/W$ entspricht dem Gruppenhomomorphismus $\alpha \colon H \to \symm{H/W}$ mit $\alpha(h)(\class{h'}) = \class{h h'}$.
          Die Komposition $\alpha \circ \varphi \colon G \to \symm{H/W}$ ist dann ebenfalls ein Gruppenhomomorphismus, und entspricht einer Gruppenwirkung von $G$ auf $H/W$ mit
          \[
              g.\class{h}
            = (\alpha \circ \varphi)(g)(h)
            = \alpha(\varphi(g))(h)
            = \class{\varphi(g) h}.
          \]
      \end{itemize}
      Es folgt, dass $\stab{G}{\class{1}}$ eine Untergruppe von $G$ ist, wobei
      \begin{align*}
            \stab{G}{\class{1}}
        &=  \{ g \in G \suchthat g.\class{1} = \class{1} \}
         =  \{ g \in G \suchthat \varphi(g).\class{1} = \class{1} \}
        \\
        &=  \{ g \in G \suchthat \varphi(g) \in \stab{H}{\class{1}} \}
         =  \{ g \in G \suchthat \varphi(g) \in W \}
         =  \preim{\varphi}{W}.
      \qedhere
      \end{align*}
    \qedhere
  \end{itemize}
\end{proof}

Aus der Surjektivität ergibt sich für jede Untergruppe $W \subgroup H$, dass
\[
    \Phi(\Phi'(W))
  = \varphi( \preim{\varphi}{W} )
  = W,
\]
weshalb $\Phi \circ \Phi' = \id_{\Hsubs}$ gilt.
Für die Komposition $\Phi' \circ \Phi$ nutzen wir die folgende Beobachtung:

\begin{lemma}
  Ist $\varphi \colon G \to H$ ein Gruppenhomomorphismus, so gilt mit $K \coloneqq \ker \varphi$ für jede Untergruppe $U \subgroup G$, dass $\preim{\varphi}{\varphi(U)} = UK$.
\end{lemma}

\begin{proof}
  Es gelten $U \subseteq \preim{\varphi}{\varphi(U)}$ und $K = \preim{\varphi}{1} \subseteq \preim{\varphi}{\varphi(U)}$, und deshalb $U K \subseteq \preim{\varphi}{\varphi(U)}$.
  Ist andererseits $g \in \preim{\varphi}{\varphi(U)}$, so gilt $\varphi(g) \in \varphi(U)$, weshalb es $u \in U$ mit $\varphi(g) = \varphi(u)$ gibt.
  Dann gilt $g = u u^{-1} g$ mit $u^{-1} g \in \ker \varphi$, denn $\varphi(u^{-1} g) = \varphi(u)^{-1} \varphi(g) = 1$.
\end{proof}

Für jede Untergruppe $U \subgroup G$ mit $K \subgroup U$ gilt somit, dass
\[
    \Phi'(\Phi(U))
  = \preim{\varphi}{\varphi(U)}
  = UK
  = U,
\]
weshalb $\Phi' \circ \Phi = \id_{\Gsubs}$ gilt.

Ingesamt zeigt dies, dass $\Phi$ eine Bijektion ist, und dass $\Phi^{-1} = \Phi'$ gilt.





\subsubsection*{Normale Untergruppen}

Die Aussage gilt auch dann noch, wenn man \enquote{Untergruppen} durch \enquote{normale Untergruppen} ersetzt.
Hierfür genügt es zu zeigen, dass sich $\Phi$ und $\Phi'$ zu Abbildungen $\Gnorms \to \Hnorms$ und $\Hnorms \to \Gnorms$ einschränken.
Dies ergibt sich aus dem Folgenden:

\begin{lemma}
  Es sei $\varphi \colon G \to H$ ein Gruppenhomomorphismus.
  \begin{enumerate}[label=\arabic*)]
    \item
      Für jede normale Untergruppe $W \nsubgroup H$ ist auch $\preim{\varphi}{W} \subgroup G$ normal.
    \item
      Ist $\varphi$ surjektiv, so ist für jede normale Untergruppe $U \nsubgroup G$ auch $\varphi(U) \subgroup H$ normal.
  \end{enumerate}
\end{lemma}

\begin{proof}
  \leavevmode
  \begin{enumerate}[label=\arabic*)]
    \item
      Wir geben zwei Beweise:
      \begin{itemize}
        \item
          Für $h \in \preim{\varphi}{W}$ gilt für jedes $g \in G$ dass
          \[
                \varphi( g h g^{-1} )
            =   \varphi(g) \underbrace{\varphi(h)}_{\in W} \varphi(g)^{-1}
            \in W,
          \]
          und somit dass $g h g^{-1} \in \preim{\varphi}{W}$.
        \item
          Die kanonische Projektion $p \colon H \to H/W$, $h \mapsto \class{h}$ ist ein Gruppenhomomorphismus mit $W = \ker p = \preim{p}{1}$.
          Deshalb ist $p \circ \varphi \colon G \to H/W$ ein Gruppenhomomorphismus mit
          \[
              \ker (p \circ \varphi)
            = \preim{(p \circ \varphi)}{1}
            = \preim{\varphi}{\preim{p}{1}}
            = \preim{p}{W}.
          \]
      \end{itemize}
    \item
      Für jedes $h \in H$ gibt es ein $g \in G$ mit $h = \varphi(g)$, weshalb
      \[
          h \varphi(U) h^{-1}
        = \varphi(g) \varphi(U) \varphi(g)^{-1}
        = \varphi( g U g^{-1} )
        = \varphi(U).
        \qedhere
      \]
  \end{enumerate}
\end{proof}

\begin{remark}
  Ist $H = G/K$ und $\varphi \colon G \to H$, $g \mapsto \class{g}$ die kanonische Projektion, so wollen wir noch die folgenden Beobachtungen festhalten:
  \begin{itemize}
    \item
      Für jede Untergruppe $U \leq G$ mit $K \leq U$ ist die entsprechende Untergruppe $\Phi(U) \subgroup H = G/K$ genau $U/K$, denn
      \[
          \Phi(U)
        = \varphi(U)
        = \{ \varphi(u) \suchthat u \in U \}
        = \{ uK \suchthat u \in U \}
        = K/U.
      \]
    \item
      Es seien $W' \nsubgroup W \subgroup H$ Untergruppen.
      Für die entsprechenden Untergruppen $U' \subgroup U \subgroup G$ mit $U = \Phi^{-1}(U)$, $U' = \Phi^{-1}(W')$ gilt dann bereits $U' \nsubgroup U$ mit $U/U' \cong W/W'$:
      \begin{itemize}
        \item
          Wie bereits gesehen gilt $W = U/K$.
          Die Einschränkung $\psi \coloneqq \restrict{\varphi}{U} \colon U \to W$, $u \mapsto \varphi(u) = uK$ ist die kanonische Projektion $U \to U/K$.
          Aus der Normalität von $W' \nsubgroup W$ folgt deshalb, dass $U' = \Phi^{-1}(W') = \preim{\varphi}{W'}$ normal in $U$ ist.
        \item
          Es gelten $W = U/K$ und $W' = U'/K$, und somit
          \[
                  U/U'
            \cong (U/K) / (U'/K)
            =     W/W'
          \]
          nach einem der Isomorphiesätze.
      \end{itemize}
  \end{itemize}
\end{remark}

\begin{example}
  Ist $K \nsubgroup G$ eine beliebige normale Untergruppe, $p \colon G \to G/K$, $g \mapsto \class{g}$ die kanonische Projektion und
  \[
                1
    =           H_0
    \nsubgroup  H_1
    \nsubgroup  H_2
    \nsubgroup  \dotsb
    \nsubgroup  H_n
    =           G/K
  \]
  eine Normalenreihe für $G/K$, so erhält man für die zurückgezogenen Untergruppen $G_i \coloneqq \preim{p}{H_i} \subgroup G$ eine Normalenreihe
  \[
                1
    \nsubgroup  K
    =           G_0
    \nsubgroup  G_1
    \nsubgroup  G_2
    \nsubgroup  \dotsb
    \nsubgroup  G_n
    =           G
  \]
  mit $G_i/G_{i-1} \cong H_i/H_{i-1}$ für alle $i$.
  Auf diese Weise lassen sich induktiv Normalenreihen konstruieren, und ggf.\ Aussagen über die entsprechenden auftretenden Quotienten treffen.
\end{example}





\subsection{}

Es gibt (mindestens) zwei Möglichkeiten, die Untergruppen von $\Integer/12$ zu bestimmen:

\begin{itemize}
  \item
    Da die Gruppe $\Integer/12$ zyklisch ist, sind auch alle Untergruppen von $\Integer/12$ zyklisch.
    Hieraus ergeben sich die Untergruppen
    \begin{gather*}
          \generated{ \class{1} }
       =  \generated{ \class{5} }
       =  \generated{ \class{7} }
       =  \generated{ \class{11} }
       =  \{ \class{0}, \class{1}, \dotsc, \class{11} \},
      \qquad
          \generated{ \class{2} }
       =  \generated{ \class{10} }
       =  \{ \class{0}, \class{2}, \class{4}, \class{6}, \class{8}, \class{10} \},
      \\
          \generated{ \class{3} }
       =  \generated{ \class{9} }
       =  \{ \class{0}, \class{3}, \class{6}, \class{9}\},
      \qquad
          \generated{ \class{4} }
       =  \generated{ \class{8} }
       =  \{ \class{0}, \class{4}, \class{8} \},
      \qquad
          \generated{ \class{6} }
       =  \{ \class{0}, \class{6} \},
      \qquad
          \generated{ \class{0} }
       =  \{ \class{0} \}.
    \end{gather*}
  \item
    Nach Aufgabenteil~\ref{subsection: subgroups of quotient} entsprechen die Untergruppen von $\Integer/12$ bijektiv den Untergruppen von $\Integer$, die $12\Integer$ enthalten.
    Jede Untergruppe von $\Integer$ ist zyklisch, da $\Integer$ zyklisch ist, und somit von der Form $n\Integer$ für $n \in \Integer$.
    Dabei gilt genau dann $12\Integer \subseteq n\Integer$, wenn $12 \in n\Integer$, wenn also $n \divides 12$.
    Die Untergruppen von $\Integer/12$ entsprechen also bijektiv den Teilern von $12$, und sind gegeben durch
    \begin{gather*}
       \Integer/12  = \{ \class{0}, \class{1}, \dotsc, \class{11} \},
      \qquad
      2\Integer/12  = \{ \class{0}, \class{2}, \class{4}, \class{6}, \class{8}, \class{10} \},
      \qquad
      3\Integer/12  = \{ \class{0}, \class{3}, \class{6}, \class{9} \},
      \\
      4\Integer/12  = \{ \class{0}, \class{4}, \class{8} \},
      \qquad
      6\Integer/12  = \{ \class{0}, \class{6} \},
      \qquad
      12\Integer/12 = \{ \class{0} \}.
    \end{gather*}
\end{itemize}

Alle Untergruppen von $\Integer/12$ sind normal, da $\Integer/12$ abelsch ist.
Die entsprechenden Quotienten lassen sich auf (mindestens) zwei verschiedene Weisen berechnen:

\begin{itemize}
  \item
    Nach einem der Isomorphiesätze gilt für jeden Teiler $d$ von $12$, dass
    \[
            (\Integer/12) / (d\Integer/12) \cong \Integer/d.
    \]
    Die entsprechenden Quotienten sind also
    \begin{gather*}
      (\Integer/12) / (\Integer/12)   \cong \Integer/1 \cong 0,
      \quad
      (\Integer/12) / (2\Integer/12)  \cong \Integer/2,
      \quad
      (\Integer/12) / (3\Integer/12)  \cong \Integer/3,
    \\
      (\Integer/12) / (4\Integer/12)  \cong \Integer/4,
      \quad
      (\Integer/12) / (6\Integer/12)  \cong \Integer/6,
      \quad
      (\Integer/12) / (12\Integer/12) \cong \Integer/12.
    \end{gather*}
  \item
    Jeder Quotient von $\Integer/12$ ist zyklisch, da $\Integer/12$ zyklisch ist.
    Für jede Untergruppe $H \subgroup \Integer/12$ gilt $| (\Integer/12)/H | = 12/|H|$, und somit
    \[
            (\Integer/12)/H
      \cong \Integer/( 12/|H| ).
    \]
    Für jedes $d \in \{0, 1, 2, 3, 4, 6\}$ gilt somit
    \begin{gather*}
            (\Integer/12)/\generated{d}
      \cong \Integer/(12/|\generated{d}|)
      =     \Integer/(12/(12/d))
      =     \Integer/d.
    \end{gather*}
\end{itemize}




