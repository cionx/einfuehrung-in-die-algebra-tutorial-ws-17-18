\section{}

Für $g, h \in G$ schreiben wir im Folgenden abkürzend $\comm{g}{h} \coloneqq ghg^{-1}h^{-1}$.
Dann ist $\comm{G}{G} = \generated{ \comm{g}{h} \suchthat g, h \in G }$.

\begin{lemma}
  Für $S \subseteq G$ und jeden Gruppenhomomorphismus $\varphi \colon G \to H$ gilt
  \[
      \varphi( \generated{S} )
    = \generated{ \varphi(S) }.
  \]
\end{lemma}

\begin{proof}
  Wir geben zwei mögliche Beweise an:
  \begin{itemize}
    \item
      Es gilt
      \begin{align*}
            \varphi( \generated{S} )
        &=  \varphi(\{
                      s_1^{\varepsilon_1} \dotsm s_n^{\varepsilon_n}
                    \suchthat 
                      n \geq 0,
                      s_1, \dotsc, s_n \in S,
                      \varepsilon_i = \pm 1
                    \})
        \\
        &=  \{
              \varphi(s_1)^{\varepsilon_1} \dotsm \varphi(s_n)^{\varepsilon_n}
            \suchthat 
              n \geq 0,
              s_1, \dotsc, s_n \in S,
              \varepsilon_i = \pm 1
            \}
        \\
        &=  \{
              t_1^{\varepsilon_1} \dotsm t_n^{\varepsilon_n}
            \suchthat 
              n \geq 0,
              t_1, \dotsc, t_n \in \varphi(S),
              \varepsilon_i = \pm 1
            \}
        =  \generated{ \varphi(S) }.
      \end{align*}
      
    \item
      Es ist $\varphi(\generated{S})$ eine Untergruppe von $H$, denn Bilder von Untergruppen sind Untergruppen.
      Es gilt $\varphi(S) \subseteq \varphi(\generated{S})$ da $S \subseteq \generated{S}$.
      Somit gilt $\generated{\varphi(S)} \subseteq \varphi(\generated{S})$, denn $\generated{\varphi(S)}$ ist die kleinste Untergruppe von $H$, die $\varphi(S)$ enthält.
      
      Andererseits ist $\varphi^{-1}( \generated{\varphi(S)} )$ eine Untergruppe von $G$, denn Urbilder von Untergruppen sind Untergruppen.
      Dabei gilt $S \subseteq \varphi^{-1}( \generated{\varphi(S)} )$ da $\varphi(S) \subseteq \generated{\varphi(S)}$.
      Somit gilt auch $\generated{S} \subseteq \varphi^{-1}( \generated{\varphi(S)} )$, denn $\generated{S}$ ist die kleinste Untergruppe von $G$, die $S$ enthält.
      Deshalb gilt auch $\varphi(\generated{S}) \subseteq \generated{\varphi(S)}$.
    \qedhere
  \end{itemize}
\end{proof}





\subsection{}

\begin{lemma}
  Es sei $\varphi \colon G \to H$ ein Gruppenhomomorphismus.
  \begin{enumerate}
    \item
      Für alle $g_1, g_2 \in G$ gilt $\varphi(\comm{g_1}{g_2}) = \comm{\varphi(g)}{\varphi(g_2)}$.
    \item
      Es gilt $\varphi(\comm{G}{G}) = \comm{\varphi(G)}{\varphi(G)}$.
  \end{enumerate}
\end{lemma}

\begin{proof}
  \begin{enumerate}
    \item
      Es gilt
      \[
          \varphi(\comm{h_1}{h_2})
        = \varphi( h_1 h_2 h_1^{-1} h_2^{-1} )
        = \varphi(h_1) \varphi(h_2) \varphi(h_1)^{-1} \varphi(h_2)^{-1}
        = \comm{ \varphi(h_1) }{ \varphi(h_2) }
      \]
    \item
      Es gilt
      \begin{align*}
           \varphi( \comm{G}{G} )
        &= \varphi( \generated{ \comm{g_1}{g_2} \suchthat g_1, g_2 \in G } )
         = \generated{ \varphi(\comm{g_1}{g_2}) \suchthat g_1, g_2 \in G }
        \\
        &= \generated{ \comm{\varphi(g_1)}{\varphi(g_2)} \suchthat g_1, g_2 \in G }
         = \generated{ \comm{h_1}{h_2} \suchthat h_1, h_2 \in \varphi(G) }
         = \comm{\varphi(G)}{\varphi(G)}.
      \end{align*}
    \qedhere
  \end{enumerate}
\end{proof}

Da für jedes $g \in G$ die Konjugationsabbildung $\varphi_g \colon G \to G$, $h \mapsto ghg^{-1}$ ein Gruppenautomorphismus ist, gilt
\[
    g \comm{G}{G} g^{-1}
  = \varphi_g( \comm{G}{G} )
  = \comm{\varphi_g(G)}{\varphi_g(G)}
  = \comm{G}{G}.
\]


\begin{remark}
  \label{remark: notion of characteristic subgroups}
  Eine Untergruppe $H \subgroup G$ heißt \emph{charakteristisch}, falls $\varphi(H) = H$ für jeden Gruppenautomorphismus $\varphi \colon G \to G$ gilt.
  Die obige Rechnung zeigt eigentlich, dass
  \begin{itemize}
    \item
      $\comm{G}{G}$ eine charakteristische Untergruppe von $G$ ist, und
    \item
      charakteristische Untergruppen stets normal sind.
  \end{itemize}
  Anschaulich gesehen ist eine Untergruppe $H \subgroup G$ charakteristisch, wenn sie sich durch die Gruppenstruktur von $G$ beschreiben lässt, bzw.\ sich aus dieser ergibt:
  Weitere Beispiele für eine charakteristische Untergruppen sind das Zentrum $\groupcenter{G}$, sowie die iterierten Kommutatorgruppen $G^{(n)}$ mit $G^{(1)} = G$ und $G^{(n+1)} = \comm{ G^{(n)} }{ G^{(n)} }$.
  
\end{remark}




\subsection{}
\label{subsection: perfect subgroup contained in commutator}

\begin{lemma}
  Für jede Untergruppe $H \subgroup G$ gilt $\comm{H}{H} \subgroup \comm{G}{G}$.
\end{lemma}

\begin{proof}
  Es gilt
  $
          \comm{H}{H}
    =     \generated{ \comm{h_1}{h_2} \suchthat h_1, h_2 \in H }
    \subgroup  \generated{ \comm{g_1}{g_2} \suchthat g_1, g_2 \in G }
    =     \comm{G}{G}.
  $
\end{proof}

Für jede perfekte Untergruppe $P \subgroup G$ gilt $P = \comm{P}{P} \subgroup \comm{G}{G}$.





\subsection{}

Es sei $\perfects \coloneqq \{P \subgroup G \suchthat \text{$P$ ist perfekt}\}$ die Menge der perfekten Untergrupppen von $G$.
Dann ist
\[
            \Perf{P}
  \coloneqq \generated*{ \bigcup_{P \in \perfects} P },
\]
die kleinste Untergruppe von $G$, die alle perfekten Untergruppen enthält.
Für jede perfekte Untergruppe $P \in \perfects$ gilt
\[
            \comm{\Perf{G}}{\Perf{G}}
  =         \generated{ \comm{g}{h} \suchthat g, h \in \Perf{G} }
  \supseteq \generated{ \comm{g}{h} \suchthat g, h \in P }
  =         \comm{P}{P}
  =         P,
\]
und somit $\bigcup_{P \in \perfects} P \subseteq \comm{\Perf{G}}{\Perf{G}}$.
Somit gilt auch
\[
            \Perf{G}
  =         \generated*{ \bigcup_{P \in \perfects} P }
  \subseteq \comm{\Perf{G}}{\Perf{G}},
\]
was zeigt, dass $\Perf{G}$ perfekt ist.
Per Konstruktion enthält $\Perf{G}$ jede perfekte Untergruppe von $G$.





\subsection{}

\begin{lemma}
  Ist $P$ perfekt, so ist für jeden Gruppenhomomorphismus $\varphi \colon P \to H$ auch $\varphi(P)$ perfekt, d.h.\ Bilder von perfekten Gruppen sind ebenfalls perfekt.
\end{lemma}

\begin{proof}
  Es gilt $\comm{ \varphi(P) }{ \varphi(P) } = \varphi( \comm{P}{P} ) = \varphi(P)$.
\end{proof}

Für jeden Gruppenhomomorphismus $\varphi \colon G \to G$ gilt
\begin{align*}
              \varphi( \Perf{G} )
   =          \varphi\left( \generated*{ \bigcup_{P \in \perfects} P } \right)
   =          \generated*{ \varphi\left( \bigcup_{P \in \perfects} P \right) }
  &=          \generated*{ \bigcup_{P \in \perfects} \varphi(P) }
  \\
  &\subseteq  \generated*{ \bigcup_{P' \in \perfects} P' }
   =          \Perf{G}.
\end{align*}
Für jedes $g \in G$ ist die Konjugationsabbildung $\varphi_g \colon G \to G$, $h \mapsto ghg^{-1}$ ein Gruppenhomomorphismus, und somit
\[
            g \Perf{G} g^{-1}
  =         \varphi_g( \Perf{G} )
  \subseteq \Perf{G}.
\]

\begin{remark}
  Jeder Gruppenautomorphismus $\varphi \colon G \to G$ induziert eine Bijektion $\perfects \to \perfects$, $P \mapsto \varphi(P)$, weshalb $\varphi(\Perf{G}) = \Perf{G}$.
  Die Untergruppe $\Perf{G} \subgroup G$ ist also charakteristisch im Sinne von Bemerkung~\ref{remark: notion of characteristic subgroups}, und somit normal.
\end{remark}





\subsection{}

Da $\Perf{G} \nsubgroup G$ normal ist, induziert die Gruppenstruktur von $G$ eine Gruppenstuktur auf $G/\Perf{G}$ mit
\[
          (G/\Perf{G}) \times (G/\Perf{G})
  \to     G/\Perf{G},
  \quad   (\class{g_1}, \class{g_2})
  \mapsto \class{g_1 g_2}.
\]





\subsection{}



\subsubsection{}

Es gilt
\[
        \text{$G/\Perf{G}$ ist trivial}
  \iff  G = \Perf{G}
  \iff  \text{$G$ ist perfekt}.
\]



\subsubsection{}

\begin{lemma}
  Ist $N \nsubgroup G$ eine normale Untergruppe, so ist $G/N$ genau dann abelsch, wenn $\comm{G}{G} \subgroup N$.
\end{lemma}

\begin{proof}
  Es gilt
  \begin{align*}
          \text{$G/N$ ist abelsch}
    &\iff \text{$\class{g} \class{h} = \class{h} \class{g}$ für alle $g, h \in G$}
    \\
    &\iff \text{$\class{g} \class{h} \class{g}^{-1} \class{h}^{-1} = 1$ für alle $g, h \in G$}
    \\
    &\iff \text{$\class{g h g^{-1} h^{-1}} = 1$ für alle $g, h \in G$}
    \\
    &\iff \text{$\class{\comm{g}{h}} = 1$ für alle $g, h \in G$}
    \\
    &\iff \text{$\comm{g}{h} \in N$ für alle $g, h \in G$}
     \iff \comm{G}{G} \subgroup N.
  \qedhere
  \end{align*}
\end{proof}

Es gilt
\[
        \text{$G/\Perf{G}$ ist abelsch}
  \iff  \comm{G}{G} \subgroup \Perf{G}
\]
Ist nun $\comm{G}{G}$ perfekt, so gilt  $\comm{G}{G} \subgroup \Perf{G}$.
Gilt andererseits $\comm{G}{G} \subgroup \Perf{G}$, so gilt nach Aufgabenteil~\ref{subsection: perfect subgroup contained in commutator} auch $\Perf{G} \subgroup \comm{G}{G}$ und somit $\comm{G}{G} = \Perf{G}$;
inbesondere ist $\comm{G}{G}$ dann perfekt.
































