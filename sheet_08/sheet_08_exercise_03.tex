\section{}





\addtocounter{subsection}{1}





\addtocounter{subsection}{1}





\addtocounter{subsection}{1}





\subsection{}

\begin{lemma}
  \label{lemma: existence of frobenius}
  Es sei $R$ ein kommutativer Ring von Charakteristik $\ringchar{R} = p$ prim.
  Dann ist die Abbildung
  \[
            \sigma
    \colon  R
    \to     R,
    \quad   x
    \mapsto x^p
  \]
  ein Ringhomomorphismus.
\end{lemma}

\begin{proof}
  Es gilt $\sigma(1) = 1^p = 1$, und für alle $x, y \in R$ gilt
  \[
      \sigma(xy)
    = (xy)^p
    = x^p y^p
    = \sigma(x) \sigma(y)\,,
  \]
  da $R$ kommutativ ist.
  Für alle $x, y \in R$ folgt aus der Kommutativität von $R$, dass
  \begin{equation}
    \label{equation: binomial theorem}
      \sigma(x + y)
    = (x + y)^p
    = \sum_{k=0}^p \binom{p}{k} x^k y^{p-k}
  \end{equation}
  gilt.
  Für alle $0 < k < p$ gelten dabei $p \ndivides k!$ und $p \ndivides (p-k)!$, und somit
  \[
    \left.        p
    \,\middle|\,  \frac{p!}{k! (p-k)!} \right.
    =             \binom{p}{k}\,.
  \]
  Damit folgt aus \eqref{equation: binomial theorem} die Gleichheit
  \[
      \sigma(x + y)
    = (x + y)^p
    = x^p + y^p
    = \sigma(x) + \sigma(y)\,.
    \qedhere
  \]
\end{proof}

\begin{remark}
  Man bezeichnet den Ringhomomorphismus $\sigma$ aus Lemma~\ref{lemma: existence of frobenius} als den \emph{Frobenius-Homomorphismus}.
\end{remark}


Wir bemerken, dass $\zeta = e^{2 \pi i / p^2}$ eine $p^2$-te primitive Einheitswurzel ist, und somit eine Nullstelle des Kreisteilungspolynoms $\Phi_{p^2}(t) \in \Integer[t] \subseteq \Rational[t]$.
Das Polynom $\Phi_{p^2}(t)$ ist normiert, und wir zeigen im Folgenden, dass es irreduzibel ist;
dann ist $\Phi_{p^2}(t)$ bereits das Minimalpolynom von $\zeta$ über $\Rational$.

In der Vorlesung haben wir die Irreduziblität von $\Phi_p(t)$ gezeigt, indem wir das Einstein-Kriterium für $\Phi_p(t+1)$ bezüglich der Primzahl $p$ angewendet haben.
Auf gleiche Weise zeigen wir, dass auch $\Phi_{p^2}(t)$ irreduzibel ist, d.h.\ wir zeigen, dass sich auf $f(t) \coloneqq \Phi_{p^2}(t+1)$ das Eisenstein-Kriterium mit der Primzahl $p$ anwenden lässt.
Hierfür nutzen wir, dass $\Phi_{p^2}(t) = \Phi_p(t^p)$ gilt.

\begin{itemize}
  \item
    Das Kreisteilungspolynom $\Phi_{p^2}(t)$ ist normiert, also ist es auch $f(t) = \Phi_{p^2}(t+1)$.
    Insbesondere ist der Leitkoeffizient von $f$ nicht durch $p$ teilbar.
  \item
    Wir müssen zeigen, dass alle anderen Koeffizienten von $f(t)$ durch $p$ teilbar sein.
    Hierfür betrachten wir den Ringhomomorphismus
    \[
              \Integer[t]
      \to     \Finite{p}[t],
      \quad   g
      =       \sum_i a_i t^i
      \mapsto \sum_i \class{a_i} t^i
      =       \class{g}.
    \]
    Für $g_1, g_2 \in \Integer[t]$ schreiben wir im Folgenden
    \[
      g_1 \equiv g_2 \mod p
    \]
    falls $\class{g_1} = \class{g_2}$ gilt.
    
    Wir wissen bereits, dass das Polynom $\Phi_p(t+1)$ mit $\deg \Phi_p(t+1) = \deg \Phi(t) = p-1$ das Eisenstein-Kriterium erfüllt, weshalb
    \[
      \Phi_p(t+1) \equiv t^{p-1} \mod p
    \]
    gilt.
    Indem wir für die Variable $t$ das Polynom $t^p$ einsetzen, erhalten wir, dass
    \[
      \Phi_p(t^p + 1) \equiv (t^p)^{p-1} =  t^{p(p-1)} \mod p
    \]
    gilt.
    Wir möchten zeigen, dass bis auf Leitkoeffizienten von $\Phi_{p^2}(t+1)$ alle Koeffizienten dieses Polynoms durch $p$ teilbar sind.
    Da
    \[
        \deg \Phi_{p^2}(t+1)
      = \deg \Phi_{p^2}(t)
      = \deg \Phi_p(t^p)
      = p \deg \Phi_p(t)
      = p(p-1)
    \]
    gilt, müssen wir also zeigen, dass
    \[
      \Phi_{p^2}(t+1) \equiv t^{p(p-1)} \mod p
    \]
    gilt.
    Setzen wir in der Gleichung $\Phi_{p^2}(t) = \Phi_p(t^p)$ für die Variable $t$ das Polynom $t+1$ ein, so erhalten wir dabei, dass $\Phi_{p^2}(t+1) = \Phi_p((t+1)^p)$ gilt.
    Wir müssen also zeigen, dass
    \[
      \Phi_p((t+1)^p) \equiv t^{p(p-1)} \mod p
    \]
    gilt.
    Nach Lemma~\ref{lemma: existence of frobenius} gilt
    \[
              (t+1)^p
      \equiv  t^p + 1^p
      =       t^p + 1
      \mod    p,
    \]
    und somit gilt
    \[
      \Phi_p((t+1)^p) \equiv \Phi_p(t^p + 1)  \mod p.
    \]
    Damit erhalten wir insgesamt, dass
    \[
              \Phi_{p^2}(t+1)
      =       \Phi_p((t+1)^p)
      \equiv  \Phi_p(t^p + 1)
      \equiv  t^{p(p-1)}
    \]
    gilt.
    Also sind alle Koeffizienten von $\Phi_{p^2}(t+1)$, bis auf den Leitkoeffizienten, durch $p$ teilbar.
    
  \item
    Wir müssen noch zeigen, dass der konstante Term von $\Phi_{p^2}(t+1)$ nicht durch $p^2$ teilbar ist.
    Dieser konstante Teil lässt sich dadurch bestimmen, dass wir für die Variable $t$ die Zahl $0 \in \Integer$ einsetzen.
    Wir erhalten die Gleichungskette
    \[
        \Phi_{p^2}(0+1)
      = \Phi_{p^2}(1)
      = \Phi_p(1^p)
      = \Phi_p(1)
      = 1^{p-1} + 1^{p-2} + \dotsb +  1^1 + 1^0
      = p.
    \]
    Der konstante Koeffizient von $\Phi_{p^2}(t+1)$ ist also $p$, und somit nicht durch $p^2$ teilbar.

\end{itemize}





% Wir möchten die Irreduziblität von $\Phi_{p^2}(t)$ mithilfe des Eisenstein-Kriteriums zeigen.
% Das Polynom $\Phi_{p^2}(t)$ selbst ist hierfür aber nicht geeignet, da (nach dem letzten Übungszettel) die Gleichheit
% \[
%     \Phi_{p^2}(t)
%   = \Phi_p(t^p)
%   = \sum_{\ell = 0}^{p-1} t^{\ell p}
% \]
% gilt.
% In der Vorlesung haben wir für das Polynom $\Phi_p(t)$ gesehen, 








\addtocounter{subsection}{1}





\addtocounter{subsection}{1}




