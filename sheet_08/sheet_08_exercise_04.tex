\section{}





\subsection{}

Die Idee hinter der Aussage ist, dass $\phi(1) = 1$ gelten, und sich alle Elemente des Primkörpers durch iteratives Anwenden den Körperoperationen (Addition, Subtraktion, Multiplikation, Division) aus $1$ ergeben.
Da $\phi$ mit diesen Operationen verträglich ist, sollte somit bereits $\phi(x) = x$ für alle $x \in P$ gelten.

Um diese Anschauung zu formalisieren, zeigen wir, dass die Menge
\[
    K^\phi
  = \{ x \in K \suchthat \phi(x) = x \}
\]
ein Unterkörper von $K$ ist.
Dann gilt $P \subseteq K$, da $P$ in jedem Unterkörper von $K$ enthalten ist.

Es gelten $\phi(0) = 0$ und $\phi(1) = 1$ und somit $0, 1 \in K$.
Für alle $x, y \in K$ gelten auch
\begin{gather*}
    \phi(x+y)
  = \phi(x) + \phi(y)
  = x + y
  \quad\text{und}\quad
    \phi(xy)
  = \phi(x) \phi(y)
  = xy,
\end{gather*}
und somit $x + y, xy \in K$.
Für jedes $x \in K$ gilt
\[
    \phi(-x)
  = -\phi(x)
  = -x
\]
und somit $-x \in K$, und falls zusätzlich $x \neq 0$ gilt, dann gilt auch
\[
    \phi(x^{-1})
  = \phi(x)^{-1}
  = x^{-1},
\]
und somit $x^{-1} \in K$.
Ingesamt zeigt dies, dass $K^\phi$ ein Unterkörper von $K$ ist.





\subsection{}

Die Abbildung $\phi \colon K \to K$ ist per Annahme bijektiv und additiv.
Für alle $\lambda \in P$, $x \in K$ gilt nach dem vorherigen Sinne, dass
\[
    \phi(\lambda x)
  = \phi(\lambda) \phi(x)
  = \lambda \phi(x).
\]
Dies zeigt insgesamt, dass $\phi$ ein $K$-Vektorraum-Automorphismus ist.
