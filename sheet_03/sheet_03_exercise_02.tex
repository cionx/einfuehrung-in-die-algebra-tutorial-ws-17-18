\section{}





\addtocounter{subsection}{1}





\addtocounter{subsection}{1}





\addtocounter{subsection}{1}





\subsection{}

Für jedes $S \in \GL{n}{\Real}$ ist die Abbildung $\Real^n \to \Real^n$, $x \mapsto Sx$ bijektiv.
Ist $A \subseteq \Real^n$ eine $k$-elementige Teilmenge, so ist deshalb die Teilmenge $S.A = \{ Sx \suchthat x \in A\}$ ebenfalls $k$-elementig.
Für jedes $A \in X$ gilt
\[
    I.A
  = \{ Ix \suchthat x \in A \}
  = \{ x \suchthat x \in A \}
  = A,
\]
und für alle $S, T \in \GL{n}{\Real}$ und $A \in X$ gilt
\[
    S.(T.A)
  = S.\{ Tx \suchthat x \in A \}
  = \{ STx \suchthat x \in A \}
  = (ST).A.
\]
Ingesamt zeigt dies, dass es sich um eine wohldefinierte Gruppenwirkung handelt.

\begin{lemma}
  \label{lemma: transitive on linear independent}
  Es seien $x_1, \dotsc, x_k \in \Real^n$.
  Dann gibt es $S \in \GL{n}{\Real}$ mit $Se_i = x_i$ für alle $i = 1, \dotsc, k$ \textup(wobei $e_1, \dotsc, e_n \in \Real^n$ die Standardbasis bezeichnet\textup).
\end{lemma}

\begin{proof}
  Durch Basisergänzung ergibt sich eine Basis $(x_1, \dotsc, x_n)$ von $\Real^n$.
  Die Matrix $S \coloneqq (x_1 x_2 \dotsb x_n) \in \GL{n}{\Real}$ leistet das Gewünschte.
\end{proof}

Um die Bahn und den Stabilisator von $A \in X$ zu bestimmen, unterscheiden wir zwischen zwei Fällen:

\begin{itemize}
  \item
    Ist $A = \{x_1, x_2\}$ linear unabhängig, so gilt $n \geq 2$.
    Dann gibt es nach Lemma~\ref{lemma: transitive on linear independent} ein Element $S \in G$ mit $S.\{e_1, e_2\} = \{x_1, x_2\} = A$.
    Das zeigt, dass alle linear unabhängigen Teilmengen eine Bahn bilden.
    Der Stabilisator von $B \coloneqq \{e_1, e_2\}$ besteht aus all jenen $S \in \GL{n}{\Real}$, so dass $Se_1 = e_1$ und $Se_2 = e_2$, oder $Se_1 = e_2$ und $Se_2 = e_1$.
    Also gilt
    \[
        G_B
      = \left\{
          \begin{pmatrix}
            S_1 & *   \\
            0   & S_2
          \end{pmatrix}
          \suchthat*
              S_1
          \in \left\{
                \begin{pmatrix}
                  1 &   \\
                    & 1
                \end{pmatrix},
                \begin{pmatrix}
                    & 1 \\
                  1 &
                \end{pmatrix}
              \right\},
              S_2
          \in \GL{n-2}{\Real}
        \right\}
    \]
    Die Stabilisatoren für irgendeine linear unabhängige Teilmenge $A \in X$ ist entsprechend konjugiert zu $G_B$.
  \item
    Ist $A = \{x_1, x_2\}$ linear abhängig, so gilt immer noch $x_1 \neq 0$ oder $x_2 \neq 0$ (da $\card{A} = 2$);
    wir können o.B.d.A.\ davon ausgehen, dass $x \coloneqq x_1 \neq 0$ gilt.
    Da $A$ linear unabhängig ist, gibt es dann ein $\lambda \in \Real$, $\lambda \neq 1$ mit $x_2 = \lambda x$, und somit $A = \{x, \lambda x\}$.
    Da $x \neq 0$ gilt, gibt es nach Lemma~\ref{lemma: transitive on linear independent} eine Matrix $S \in \GL{n}{\Real}$ mit $Se_1 = x$.
    Dann gilt
    \[
        A
      = \{ x, \lambda x \}
      = \{ Se_1, \lambda Se_1 \}
      = S.\{ e_1, \lambda e_1 \}.
    \]
    Zur Bestimmung der restlichen Bahnen genügt es also, die Mengen $B_\lambda \coloneqq \{e_1, \lambda e_1\}$ mit $\lambda \in \Real$, $\lambda \neq 1$ zu betrachten.
    
    Eine Menge $A \in X$ ist genau dann in der Bahn von $B_\lambda$, wenn eines der Elemente von $A$ das $\lambda$-fache des anderen ist.
    Es folgt, dass
    \begin{itemize}
      \item
        $B_0$ und $B_\lambda$ für $\lambda \neq 1, 0$ nie in derselben Bahn liegen,
      \item
        $B_\lambda$ und $B_\mu$ für $\lambda \neq 1, 0$ genau dann in der gleichen Bahn liegen, wenn $\mu = \lambda$ oder $\mu = 1/\lambda$ gilt.
    \end{itemize}
    Man bemerke, dass dabei für $\lambda = -1$ gilt, dass $1/\lambda = -1 = \lambda$.
    Mit Ausnahme von $B_0$ und $B_{-1}$ liegen also je zwei $B_\lambda$ in der gleichen Bahn.
    Ist $\Lambda \subseteq \Real$ mit
    \begin{itemize}
      \item
        $1 \notin \Lambda$,
      \item
        $0 \in \Lambda$,
      \item
        für all $\lambda \neq 1, 0, -1$ ist entweder $\lambda \in \Lambda$ oder $1/\lambda \in \Lambda$,
    \end{itemize}
    so  sind die Mengen $B_\lambda \in X$ mit $\lambda \in \Lambda$ also ein Repräsentantensystem für die Bahnen der linear abhängigen Mengen.
    (Man kann etwa $\Lambda = [-1, 1)$ wählen.)
    
\end{itemize}





\addtocounter{subsection}{1}





\addtocounter{subsection}{1}





\subsection{}

Für jedes $f \in X$ gilt
\[
    \id.f
  = f \circ \id^{-1}
  = f \circ \id
  = f,
\]
und für alle $\pi_1, \pi_2 \in S_n$ und jedes $f \in X$ gilt
\[
    \pi_1.(\pi_2.f)
  = \pi_1(f \circ \pi_2^{-1})
  = f \circ \pi_2^{-1} \circ \pi_1^{-1}
  = f \circ (\pi_1 \circ \pi_2)^{-1}
  = (\pi_1 \pi_2).f,
\]
weshalb es sich tatsächlich um eine Gruppenwirkung handelt.

Wir können eine Funktion $f \colon \{1, \dotsc, n\} \to \{1, \dotsc, n\}$ als ein Tupel $(f(1), \dotsc, f(n))$ schreiben:
\[
          f
  \equiv  (f(1), \dotsc, f(n)).
\]
Die Wirkung von $S_n$ auf $X$ ist dann durch
\[
    \pi.(a_1, \dotsc, a_n)
  = \left( a_{\pi^{-1}(1)}, \dotsc, a_{\pi^{-1}(n)} \right)
\]
gegeben.
(Man bemerke, dass sich diese Wirkung analog zu der Wirkung aus Aufgabenteil~(a) verhält.)
Zwei Tupel $(a_1, \dotsc, a_n), (b_1, \dotsc, b_n) \in X$ sind genau dann in der gleichen Bahn, wenn sie die gleichen Einträge mit jeweils gleicher Vielfachheit erhalten.
Ein Repräsentantensystem der Bahnen ist deshalb durch die Tupel
\[
  ( \underbrace{1, \dotsc, 1}_{m_1},
    \underbrace{2, \dotsc, 2}_{m_2},
    \dotsc,
    \underbrace{n, \dotsc, n}_{m_n} )
\]
mit $m_1, \dotsc, m_n \geq 0$, $m_1 + \dotsb + m_n = n$ gegeben.
Der Stabilisator eines solchen Repräsentanten
\[
  f = ( \underbrace{1, \dotsc, 1}_{m_1},
        \underbrace{2, \dotsc, 2}_{m_2},
        \dotsc,
        \underbrace{n, \dotsc, n}_{m_n} )
\]
ist durch
\begin{align*}
   &\,      G_f
  \\
  =&\,      \left\{
              \pi \in S_n
            \suchthat*
              \begin{tabular}{c}
                $\pi$ permutiert das diskrete Intervall \\
                $\{m_1 + \dotsb + m_k + 1, \dotsc, m_1 + \dotsb + m_{k+1}\}$  \\
                für jedes $k = 0, \dotsc, n-1$
              \end{tabular}
            \right\}
  \\
  \cong&\,  S_{m_1} \times \dotsb \times S_{m_n}
\end{align*}
gegeben.




