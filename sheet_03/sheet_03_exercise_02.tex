\section{}





\addtocounter{subsection}{1}





\addtocounter{subsection}{1}





\addtocounter{subsection}{1}





\addtocounter{subsection}{1}





\addtocounter{subsection}{1}





\addtocounter{subsection}{1}





\subsection{}

Für jedes $f \in X$ gilt
\[
    \id.f
  = f \circ \id^{-1}
  = f \circ \id
  = f,
\]
und für alle $\pi_1, \pi_2 \in S_n$ und jedes $f \in X$ gilt
\[
    \pi_1.(\pi_2.f)
  = \pi_1(f \circ \pi_2^{-1})
  = f \circ \pi_2^{-1} \circ \pi_1^{-1}
  = f \circ (\pi_1 \circ \pi_2)^{-1}
  = (\pi_1 \pi_2).f,
\]
weshalb es sich tatsächlich um eine Gruppenwirkung handelt.

Wir können eine Funktion $f \colon \{1, \dotsc, n\} \to \{1, \dotsc, n\}$ als ein Tupel $(f(1), \dotsc, f(n))$ schreiben:
\[
          f
  \equiv  (f(1), \dotsc, f(n)).
\]
Die Wirkung von $S_n$ auf $X$ ist dann durch
\[
    \pi.(a_1, \dotsc, a_n)
  = \left( a_{\pi^{-1}(1)}, \dotsc, a_{\pi^{-1}(n)} \right)
\]
gegeben.
(Man bemerke, dass sich diese Wirkung analog zu der Wirkung aus Aufgabenteil~(a) verhält.)
Zwei Tupel $(a_1, \dotsc, a_n), (b_1, \dotsc, b_n) \in X$ sind genau dann in der gleichen Bahn, wenn sie die gleichen Einträge mit jeweils gleicher Vielfachheit erhalten.
Ein Repräsentantensystem der Bahnen ist deshalb durch die Tupel
\[
  ( \underbrace{1, \dotsc, 1}_{m_1},
    \underbrace{2, \dotsc, 2}_{m_2},
    \dotsc,
    \underbrace{n, \dotsc, n}_{m_n} )
\]
mit $m_1, \dotsc, m_n \geq 0$, $m_1 + \dotsb + m_n = n$ gegeben.
Der Stabilisator eines solchen Repräsentanten
\[
  f = ( \underbrace{1, \dotsc, 1}_{m_1},
        \underbrace{2, \dotsc, 2}_{m_2},
        \dotsc,
        \underbrace{n, \dotsc, n}_{m_n} )
\]
ist durch
\begin{align*}
   &\,      G_f
  \\
  =&\,      \left\{
              \pi \in S_n
            \suchthat*
              \begin{tabular}{c}
                $\pi$ permutiert das diskrete Intervall \\
                $\{m_1 + \dotsb + m_k + 1, \dotsc, m_1 + \dotsb + m_{k+1}\}$  \\
                für jedes $k = 0, \dotsc, n-1$
              \end{tabular}
            \right\}
  \\
  \cong&\,  S_{m_1} \times \dotsb \times S_{m_n}
\end{align*}
gegeben.




