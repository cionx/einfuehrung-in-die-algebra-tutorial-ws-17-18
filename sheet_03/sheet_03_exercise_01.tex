\section{}





\subsection{}

Es gilt
\[
    n!
  = 1 \cdot 2 \cdot 3 \dotsm n.
\]
Wir zählen, wie häufig der Primfaktor $p$ in $n!$ vorkommt:

Jeder $p$-te Faktor ist durch $p$ teilbar, d.h.\ in $\floor{n/p}$ vielen der Faktoren kommt der Primfaktor $p$ vor.
In jedem $p^2$-ten Faktor kommt er sogar zweimal vor, und im jedem $p^3$-ten dreimal, usw.
Somit kommt der Primfaktore $p$ in dem Produkt $1 \cdot 2 \dotsm n$ ingesamt $\sum_{i=1}^\infty \floor{n/p^i}$ mal vor.
Inbesondere ist die Summe endlich.





\subsection{}

Wegen der Additivität von $\nu_p$ (es gilt $\nu_p(ab) = \nu_p(a) + \nu_p(b)$ für alle $n \in \Natural$) gilt
\[
    \nu_p\left( \binom{p^r m}{p^k} \right)
  = \nu_p\left( \frac{(p^r m)!}{(p^k)! (p^r m - p^k)!} \right)
  = \nu_p((p^r m)!) - \nu_p((p^k)!) - \nu_p((p^r m - p^k)!).
\]
Dabei gelten
\begin{align*}
      \nu_p( (p^r m!) )
  =   \sum_{i=1}^\infty \floor*{ \frac{p^r m}{p^i} }
  &=  \sum_{i=1}^k p^{r-i} m + \sum_{i={k+1}}^\infty \floor*{ \frac{p^r m}{p^i} }
  \\
  &=  \sum_{i=1}^k p^{r-i} m + \sum_{j=1}^\infty \floor*{ \frac{p^{r-k} m}{p^j} }
   =  \sum_{i=1}^k p^{r-i} m + \nu_p( (p^{r-k} m)! )
\end{align*}
sowie
\[
    \nu_p( (p^k)! )
  = \sum_{i=1}^\infty \floor*{ \frac{p^k}{p^i} }
  = \sum_{i=1}^k p^{k-i}
\]
und
\begin{align*}
    \nu_p( (p^r m - p^k)! )
  &= \sum_{i=1}^\infty \floor*{ \frac{p^r m - p^k}{p^i} }
   = \sum_{i=1}^k (p^{r-i} m - p^{k-i}) + \sum_{i=k+1}^\infty \floor*{ \frac{p^r m - p^k}{p^i} }
  \\
  &= \sum_{i=1}^k (p^{r-i} m - p^{k-i}) + \sum_{j=1}^\infty \floor*{ \frac{p^{r-k} m - 1}{p^j} }
  \\
  &= \sum_{i=1}^k (p^{r-i} m - p^{k-i}) + \nu_p((p^{r-k} m - 1)!).
\end{align*}
Damit folgt, dass
\begin{align*}
      \nu_p\left( \binom{p^r m}{p^k} \right)
  &=  \nu_p( (p^{r-k} m)! ) - \nu_p( (p^{r-k} m - 1)! )
  \\
  &=  \nu_p\left( \frac{(p^{r-k} m)!}{(p^{r-k} m - 1)!} \right)
   =  \nu_p( p^{r-k} m )
   =  r-k,
\end{align*}
wobei wir für die letzte Gleichheit nutzen, dass $p \ndivides m$.










\subsection{}

Es gilt $S \nsubgroup \normalizer{S}{G}$ nach Definition von $\normalizer{S}{G}$, und nach Annahme gilt $H \subgroup \normalizer{S}{G}$.
Nach einem der Isomorphiesätze ist deshalb $HS$ eine Untergruppe von $\normalizer{S}{G}$, sowie $H \cap S$ eine normale Untergruppe von $H$ mit $HS/S \cong H/(H \cap S)$.
Inbesondere ist $HS/S$ mit der Multiplikation $\class{g_1} \class{g_2} = \class{g_1 g_2}$ eine wohldefinierte Gruppe.
Es handelt sich um eine $p$-Gruppe da
\[
                \card{ HS/S }
  =             \card{ H/(H \cap S) }
  = \left.      \frac{ \card{H} }{ \card{H \cap S} }
  \,\middle|\,  \card{H}
    \right.
\]
und $\card{H}$ eine $p$-Gruppe ist.





\subsection{}
\label{subsection: contained in Sylow subgroup}

Es gilt
\[
    \card{HS}
  = \frac{\card{HS}}{\card{S}} \, \card{S}
  = \card{HS/S} \, \card{S}.
\]
Da $HS/S$ und $S$ beides $p$-Gruppen sind, ist deshalb auch $HS$ eine $p$-Gruppe.
Als $p$-Sylowuntergruppe ist $S$ kardinalitäts- und damit auch inklusionsmaximal unter allen $p$-Untergruppen von $G$;
zusammen mit $S \subgroup HS$ folgt damit, dass bereits $S = HS$ gilt.
Somit gelten $HS/S = S/S = 1$ und $H \subgroup HS = S$.





\subsection{}
\label{subsection: unique fixed point}

Für jede $p$-Sylowuntergruppe $S' \in \Sylow{p}{G}$ gilt
\begin{align*}
        S' \in \Sylow{p}{G}^S
  &\iff \forall s \in S: s.S' = S'
   \iff \forall s \in S: s S' s^{-1} = S'
  \\
  &\iff \forall s \in S: s \in \normalizer{S'}{G}
   \iff S \subgroup \normalizer{S'}{G}
  \\
  &\iff S \subgroup S'
   \iff S = S'.
\end{align*}
Dabei nutzen wir für die vorletzte Äquivalenz Aufgabenteil~\ref{subsection: contained in Sylow subgroup}.
Für die letzte Äquivalenz nutzen wir, dass $\card{S} = \card{S'}$ da $S$ und $S'$ zwei $p$-Sylowuntergruppen sind.
Ingesamt zeigt dies, dass $S$ der eindeutige Fixpunkt der gegebenen Wirkung ist.





\subsection{}

Nach der Bahnengleichung gilt
\begin{align*}
      \card{\Sylow{p}{G}}
  &=  \sum_{\orbit \in \Sylow{p}{G}/S} \card{\orbit}
   =    \card*{\Sylow{p}{G}^S}
      + \sum_{\substack{\orbit \in \Sylow{p}{G}/S \\ \card{\orbit} > 1}}
        \card{\orbit}
  \\
  &=    \card*{\Sylow{p}{G}^S}
      + \sum_{\substack{\orbit \in \Sylow{p}{G}/S \\ \card{\orbit} > 1, \, S' \in \orbit}}
        \groupindex{S}{S_{S'}}.
\end{align*}
Dabei gilt für $\orbit \in \Sylow{p}{G}/S$ und $S' \in \orbit$ mit $\groupindex{S}{S_{S'}} = \card{\orbit} > 1$ wegen $\groupindex{S}{S_{S'}} \divides \card{S}$, dass $\groupindex{S}{S_{S'}}$ eine nicht-trivale $p$-Potenz ist;
inbesondere ist $\groupindex{S}{S_{S'}}$ ein Vielfaches von $p$.
Zudem gilt nach Aufgabenteil~\ref{subsection: unique fixed point}, dass $\card{\Sylow{p}{G}} = 1$.
Ingesamt gilt somit
\[
          \card{\Sylow{p}{G}}
  =       \underbrace{ \card*{\Sylow{p}{G}^S} }_{=1}
          + \underbrace{
            \sum_{\substack{\orbit \in \Sylow{p}{G}/S \\ \card{\orbit} > 1, \, S' \in \orbit}}
            \groupindex{S}{S_{S'}}
            }_{\text{Vielfaches von $p$}}
  \equiv  1
  \mod    p.
\]





\subsection{}

Die Gruppe $G$ wirkt $\Sylow{p}{G}$ durch Konjugation, d.h.\ durch
\[
    g.S'
  = g S' g^{-1}
  \qquad
  \text{für alle $g \in G$, $S' \in \Sylow{p}{G}$}.
\]
Nach dem zweiten Sylowsatz ist diese Wirkung transitiv, d.h.\ für jedes $S' \in \Sylow{p}{G}$ gibt es ein $g \in G$ mit $g.S = S'$.
Dabei gilt
\[
    G_S
  = \{ g \in G \suchthat g.S = S \}
  = \{ g \in G \suchthat g S g^{-1} = S \}
  = \normalizer{S}{G}.
\]
Somit gilt
\[
    \card{\Sylow{p}{G}}
  = \card{G.S}
  = \groupindex{G}{G_S}
  = \groupindex{G}{\normalizer{S}{G}}.
\]
Dabei gilt
\[
    m
  = \frac{\card{G}}{\card{S}}
  = \groupindex{G}{S}
  = \groupindex{G}{\normalizer{S}{G}} \groupindex{\normalizer{S}{G}}{S}
\]
und somit
\[
            \card{\Sylow{p}{G}}
  =         \groupindex{G}{\normalizer{S}{G}}
  \divides  m.
\]
