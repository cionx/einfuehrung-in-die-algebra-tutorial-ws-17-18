\section{}





\addtocounter{subsection}{1}





\addtocounter{subsection}{1}





\subsection{}

Es gilt $S \nsubgroup \normalizer{S}{G}$ nach Definition von $\normalizer{S}{G}$, und nach Annahme gilt $H \subgroup \normalizer{S}{G}$.
Nach einem der Isomorphiesätze ist deshalb $HS$ eine Untergruppe von $\normalizer{S}{G}$, sowie $H \cap S$ eine normale Untergruppe von $H$ mit $HS/S \cong H/(H \cap S)$.
Inbesondere ist $HS/S$ mit der Multiplikation $\class{g_1} \class{g_2} = \class{g_1 g_2}$ eine wohldefinierte Gruppe.
Es handelt sich um eine $p$-Gruppe da
\[
                \card{ HS/S }
  =             \card{ H/(H \cap S) }
  = \left.      \frac{ \card{H} }{ \card{H \cap S} }
  \,\middle|\,  \card{H}
    \right.
\]
und $\card{H}$ eine $p$-Gruppe ist.





\subsection{}

Es gilt
\[
    \card{HS}
  = \frac{\card{HS}}{\card{S}} \, \card{S}
  = \card{HS/S} \, \card{S}.
\]
Da $HS/S$ und $S$ beides $p$-Gruppen sind, ist deshalb auch $HS$ eine $p$-Gruppe.
Als $p$-Sylowgruppe ist $S$ kardinalitäts- und damit auch inklusionsmaximal unter allen $p$-Untergruppen von $G$;
zusammen mit $S \subgroup HS$ folgt damit, dass bereits $S = HS$ gilt.
Somit gelten $HS/S = S/S = 1$ und $H \subgroup HS = S$.





\addtocounter{subsection}{1}





\addtocounter{subsection}{1}





\addtocounter{subsection}{1}
