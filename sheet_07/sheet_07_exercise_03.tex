\section{}





\subsection{}

Für jedes $n \geq 1$ gilt
\begin{align*}
    W_n
  &=  \{
        e^{2 \pi i  k / n}
      \suchthat
        k = 0, \dotsc, n-1
      \}
  &=  \{
        \cos(2 \pi k / n) + i \sin(2 \pi k /n )
      \suchthat
        k = 0, \dotsc, n-1
      \}
\end{align*}
Aus $\cos(2 \pi /3) = \cos(4 \pi / 3)  = -1/2$ und $\sin(2 \pi / 3) = \sqrt{3}/2$, $\sin(4 \pi /3) = -\sqrt{3}/2$ folgen damit, dass
\begin{align*}
  W_2 &= \{ 1, -1 \}\,, \\
  W_3 &= \left\{ 1, -\frac{1}{2} + i \frac{\sqrt{3}}{2}, -\frac{1}{2} - i \frac{\sqrt{3}}{2} \right\}\,, \\
  W_4 &= \{ 1, i, -1, -i \}\,.
\end{align*}





\subsection{}

Für alle $n \geq 1$ ist die Abbildung
\[
          \varphi_n
  \colon  \Integer
  \to     \Complex^\times\,,
  \quad   k
  \mapsto e^{2 \pi i k / n}
\]
ist ein Gruppenhomomorphismus mit $\im \varphi_n = W_n$ und $\ker \varphi = n \Integer$.
Somit ist $W_n$ eine Untergruppe von $\Complex^\times$, und $\varphi_n$ induziert nach der universellen Eigenschaft des Quotienten einen Gruppenisomorphismus
\[
          \Integer/n
  \to     W_n,
  \quad   \class{k}
  \mapsto \varphi_n(k)
  =       e^{2 \pi i k / n}.
\]

Die Abbildung
\[
          \varphi_\infty
  \colon  \Rational
  \to     \Complex^\times\,,
  \quad   \frac{p}{q}
  \mapsto e^{2 \pi i p /q}\,,
\]
ist ein Gruppenhomomorphismus mit $\im \varphi_\infty = \bigcup_{n \geq 1} W_n \eqqcolon W_\infty$ und $\ker \Integer$.
Somit ist $W_\infty$ eine Untergruppe von $\Complex^\times$, und $\varphi_\infty$ induziert nach der universellen Eigenschaft des Quotienten einen Gruppenisomorphismus
\[
          \Rational/\Integer
  \to     W_\infty\,,
  \quad   \class{ \frac{p}{q} }
  \mapsto \varphi_\infty\left( \frac{p}{q} \right)
  =       e^{2 \pi i p /q}\,.
\]

\begin{remark}
  Wir werden sehen, dass für einen beliebigen Körper $K$ die Gruppe der Einheitswurzeln
  \[
              W_n(K)
    \coloneqq \{x \in K \suchthat x^n = 1\}
  \]
  zyklisch ist;
  dies wird daraus folgen, dass jede endliche Untergruppe $H \subgroup K^\times$ zylisch ist.
  Gilt $\ringchar{K} = 0$, so hat die Gruppe $W_n(K)$ Ordnung $n$;
  gilt hingegen $\ringchar{K} = p > 0$, und ist $n = p^r m$ mit $p \ndivides m$, so hat die Gruppe $W_n(K)$ Ordnung $m$.
\end{remark}


