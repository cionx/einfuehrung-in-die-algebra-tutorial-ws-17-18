\section{}





\subsection{}

Für jedes $n \geq 1$ gilt
\begin{align*}
    W_n
  &=  \{
        e^{2 \pi i  k / n}
      \suchthat
        k = 0, \dotsc, n-1
      \}
  &=  \{
        \cos(2 \pi k / n) + i \sin(2 \pi k /n )
      \suchthat
        k = 0, \dotsc, n-1
      \}
\end{align*}
Aus $\cos(2 \pi /3) = \cos(4 \pi / 3)  = -1/2$ und $\sin(2 \pi / 3) = \sqrt{3}/2$, $\sin(4 \pi /3) = -\sqrt{3}/2$ folgen damit, dass
\begin{align*}
  W_2 &= \{ 1, -1 \}\,, \\
  W_3 &= \left\{ 1, -\frac{1}{2} + i \frac{\sqrt{3}}{2}, -\frac{1}{2} - i \frac{\sqrt{3}}{2} \right\}\,, \\
  W_4 &= \{ 1, i, -1, -i \}\,.
\end{align*}





\subsection{}

Für alle $n \geq 1$ ist die Abbildung
\[
          \varphi_n
  \colon  \Integer
  \to     \Complex^\times\,,
  \quad   k
  \mapsto e^{2 \pi i k / n}
\]
ist ein Gruppenhomomorphismus mit $\im \varphi_n = W_n$ und $\ker \varphi = n \Integer$.
Somit ist $W_n$ eine Untergruppe von $\Complex^\times$, und $\varphi_n$ induziert nach der universellen Eigenschaft des Quotienten einen Gruppenisomorphismus
\[
          \Integer/n
  \to     W_n,
  \quad   \class{k}
  \mapsto \varphi_n(k)
  =       e^{2 \pi i k / n}.
\]

Die Abbildung
\[
          \varphi_\infty
  \colon  \Rational
  \to     \Complex^\times\,,
  \quad   \frac{p}{q}
  \mapsto e^{2 \pi i p /q}\,,
\]
ist ein Gruppenhomomorphismus mit $\im \varphi_\infty = \bigcup_{n \geq 1} W_n \eqqcolon W_\infty$ und $\ker \Integer$.
Somit ist $W_\infty$ eine Untergruppe von $\Complex^\times$, und $\varphi_\infty$ induziert nach der universellen Eigenschaft des Quotienten einen Gruppenisomorphismus
\[
          \Rational/\Integer
  \to     W_\infty\,,
  \quad   \class{ \frac{p}{q} }
  \mapsto \varphi_\infty\left( \frac{p}{q} \right)
  =       e^{2 \pi i p /q}\,.
\]

\begin{remark}
  Wir werden sehen, dass für einen beliebigen Körper $K$ die Gruppe der Einheitswurzeln
  \[
              W_n(K)
    \coloneqq \{x \in K \suchthat x^n = 1\}
  \]
  zyklisch ist;
  dies wird daraus folgen, dass jede endliche Untergruppe $H \subgroup K^\times$ zylisch ist.
  Gilt $\ringchar{K} = 0$, so hat die Gruppe $W_n(K)$ Ordnung $n$;
  gilt hingegen $\ringchar{K} = p > 0$, und ist $n = p^r m$ mit $p \ndivides m$, so hat die Gruppe $W_n(K)$ Ordnung $m$.
\end{remark}





\subsection{}

Entscheidend ist die folgende Beobachtung:

\begin{lemma}
  \label{lemma: decomposition into cyclotomic polynomials}
  Für alle $n \geq 1$ gilt
  \[
      t^n - 1
    = \prod_{d \divides n} \Phi_d(t)\,.
  \]
\end{lemma}

\begin{proof}
  Für $\zeta \in W_n$ sei $d \coloneqq \ord{\zeta}$.
  Es gilt $\zeta^d = 1$, weshalb $\zeta$ eine $d$-te Einheitswurzel ist, d.h.\ es gilt $\zeta \in W_d$.
  Da
  \[
      \ord{\zeta}
    = d
    = \ord{W_d}
  \]
  gilt, ist $\zeta$ bereits ein zyklischer Erzeuger von $W_d$.
  Also ist $\zeta$ eine primitive $d$-te Einheitswurzel.
  Zudem gilt
  \[
              d
    =         \ord{\zeta}
    \divides  \ord{W_n}
    =         n\,.
  \]
  Damit erhalten wir ingesamt, dass
  \[
      W_n
    = \coprod_{d \divides n} \{ \zeta \in W_n \suchthat \ord{\zeta} = d \}
    = \coprod_{d \divides n} \{ \zeta \in W_d \suchthat \text{$\zeta$ ist primitiv} \}\,.
  \]
  (Hier steht $\coprod$ für die disjunkte Vereinigung.)
  \[
      t^n - 1
    = \prod_{\zeta \in W_n} (t - \zeta)
    = \prod_{d \divides n} \prod_{\substack{\zeta \in W_d \\ \text{primitiv}}} (t - \zeta)
    = \prod_{d \divides n} \Phi_d(t)\,.
  \]
  Das zeigt die Gleichheit.
\end{proof}

\begin{remark}
  Die obige Argumentation lässt sich dahingehend verallgemeinern, dass für jede Gruppe $G$ die disjunkte Zerlegung
  \begin{align*}
        G
    &=  \coprod_{H \subgroup G} \{\text{$h \in H$ ist ein zyklischer Erzeuger von $H$}\}  \\
    &=  \coprod_{\substack{H \subgroup G \\ \text{zyklisch}}} \{\text{$h \in H$ ist ein Erzeuger von $H$}\}
  \end{align*}
  gilt.
  Dies ist nur eine Umformulierung der Tatsache, dass jedes Element $g \in G$ eine eindeutige (zyklische) Untergruppe $\generated{g} \subgroup G$ erzeugt.
\end{remark}

\begin{remark}
  Aus Lemma~\ref{lemma: decomposition into cyclotomic polynomials} folgt insbesondere, dass für jede natürliche Zahl $n \geq 1$ die Gleichheit
  \[
      n
    = \deg(t^n - 1)
    = \sum_{d \divides n} \deg \Phi_d(t)
    = \sum_{d \divides n} \phi(t),
  \]
  gilt, wobei $\varphi$ die Eulersche Phi-Funktion bezeichnet.
\end{remark}

Aus der Vorlesung ist bereits bekannt, dass
\[
    \Phi_p(t)
  = t^{p-1} + t^{p-2} + \dotsb + t + 1
  = \frac{t^p - 1}{t-1}
\]
für jede Primzahl $p$.
Aus Lemma~\ref{lemma: decomposition into cyclotomic polynomials} ergibt sich eine Verallgemeinerung dieser Gleichheit:

\begin{corollary}
  \label{corollary: inductive calculation of cyclotomic polynomials}
  Für alle $n \geq 1$ gilt
  \[
      \Phi_n(t)
    = \frac{t^n - 1}{\prod_{d \divides n, d \neq n} \Phi_d(t)}\,.
  \]
\end{corollary}

Mithilfe von Korollar~\ref{corollary: inductive calculation of cyclotomic polynomials} und Polynomdivision lassen sich die Kreisteilungspolynome $\Phi_n(t)$ nun induktiv berechnen.
Für $n = 1, \dotsc, 8$ erhalten wir die folgenden Ergebnisse:
\begin{align*}
      \Phi_1(t)
  &=  t-1\,,
  \\
      \Phi_2(t)
  &=  t+1\,,
  \\
      \Phi_3(t)
  &=  t^2 + t + 1\,,
  \\
      \Phi_4(t)
  &=  \frac{t^4 - 1}{\Phi_1(t) \Phi_2(t)}
   =  \frac{t^4 - 1}{(t-1)(t+1)}
   =  \frac{t^4 - 1}{t^2 - 1}
   =  t^2 + 1\,,
  \\
      \Phi_5(t)
  &=  t^4 + t^3 + t^2 + t + 1\,,
  \\
      \Phi_6(t)
  &=  \frac{t^6 - 1}{\Phi_1(t) \Phi_2(t) \Phi_3(t)}
   =  \frac{t^6 - 1}{(t-1)(t+1)(t^2 + t + 1)}
   =  \frac{t^6 - 1}{t^4 + t^3 - t - 1}
   =  t^2 - t + 1\,,
  \\
      \Phi_7(t)
  &=  t^6 + t^5 + t^4 + t^3 + t^2 + t + 1\,,
  \\
      \Phi_8(t)
  &=  \frac{t^8 - 1}{\Phi_1(t) \Phi_2(t) \Phi_4(t)}
   =  \frac{t^8 - 1}{(t-1)(t+1)(t^2 + 1)}
   =  \frac{t^8 - 1}{t^4 - 1}
   =  t^4 +1\,.
\end{align*}

\begin{remark}
  Es fällt auf, dass alle bisher bekannten Kreisteilungspolynome (also $\Phi_p(t)$ mit $p$ prim, und $\Phi_n(t)$ für $n = 1, \dotsc, 8$) jeweils nur $1, 0, -1$ als Koeffizienten haben.
  Dieses Muster setzt sich bis $\Phi_{104}(t)$ vor; das Kreisteilungspolynome $\Phi_{105}(t)$ hat schließlich den Koeffizienten $-2$.
\end{remark}
