\section{Äußere direkte Produkte}





\subsection*{Definition}

Es seien $G$ und $H$ zwei Gruppen.
Auf der Menge $G \times H$ wird durch die Multiplikation
\[
          (g_1, h_1) \cdot (g_2, h_2)
  \define (g_1 g_2, h_1 h_2)
\]
eine Gruppenstruktur definiert.
Wir nennen die entstehende Gruppe das \emph{äußere direkte Produkt} von $G$ und $H$, und bezeichnen diesese mit $G \times H$.





\subsection*{Einbettungen von \texorpdfstring{$G$}{G} und \texorpdfstring{$H$}{H} in \texorpdfstring{$G \times H$}{G x H}}

Die Gruppe $G \times H$ enthält die beiden Untergruppen
\[
          \induced{G}
  \define G \times \{1\}
  \quad\text{und}\quad
          \induced{H}
  \define \{1\} \times H \,.
\]
Diese sind isomorph zu $G$, bzw.\ $H$, da sich die Gruppenmonomorphismen
\begin{alignat*}{2}
            i
  &\colon   G
   \to      G \times H,
   &
   \quad    g
  &\mapsto  (g,1) \,,
  \\
            j
  &\colon   H
   \to      G \times H,
   &
            h
  &\mapsto (1,h)
\end{alignat*}
zu Gruppenisomorphismen
\[
        \restrict{i}{}[\induced{G}]
 \colon G
 \to    \induced{G}
 \quad\text{und}\quad
        \restrict{j}{}[\induced{H}]
 \colon H
 \to    \induced{H}
\]
einschränken.
Wir können also $G$ und $H$ mit Untergruppen von $G \times H$ identifizieren.
Wir werden diese Identifikation im Folgenden \emph{nicht} implizit vornehmen, sondern stets explizit.
Hierfür bezeichnen wir für $g \in G$ und $h \in H$ die entsprechenden Elemente aus $G \times H$ mit
\[
          \induced{g}
  \define i(g)
  =       (g,1)
  \quad\text{und}\quad
          \induced{h}
  \define j(h)
  =       (1,h).
\]
Vorstellungsmäßig unterscheiden wir im Folgenden allerdings nicht zwischen den Elementen $g$ und $\induced{g}$, sowie den Elementen $h$ und $\induced{h}$.





\subsection*{Universelle Eigenschaft von \texorpdfstring{$G \times H$}{G x H}}

Stellen wir uns $G$ und $H$ als Untergruppen von $G \times H$ vor, so kommutieren $G$ und $H$ miteinander:
Für alle $g \in G$, $h \in H$ gilt
\[
    \induced{g} \induced{h}
  = (g, 1) (1, h)
  = (g, h)
  = (1, h) (g, 1)
  = \induced{h} \induced{g} \,.
\]
Die Inklusionen $i \colon G \to G \times H$ und $j \colon H \to G \times H$ betten also $G$ und $H$ in eine Gruppe $G \times H$ ein, in der $G$ und $H$ dann miteinander kommutieren.
Tatsächlich ist $G \times H$ bereits universell mit dieser Eigenschaft:

\begin{proposition}
  Es sei $T$ eine Gruppe, und es seien $\alpha \colon G \to T$ und $\beta \colon H \to T$ zwei Gruppenhomomorphismen, so dass
  \[
      \alpha(g) \beta(h)
    = \beta(h) \alpha(g)
    \qquad
    \text{für alle $g \in G$, $h \in H$}
  \]
  gilt.
  Dann gibt es einen eindeutigen Gruppenhomomorphismus $\varphi \colon G \times H \to T$, der die folgenden beiden Diagramme zum Kommutieren bringt:
  \[
    \begin{tikzcd}
        G \times H
        \arrow{rr}{\varphi}
      & {}
      & T
      \\
        {}
      & G
        \arrow[swap]{ru}{i}
        \arrow{lu}{\alpha}
      & {}
    \end{tikzcd}
    \qquad\qquad
    \begin{tikzcd}
        G \times H
        \arrow{rr}{\varphi}
      & {}
      & T
      \\
        {}
      & H
        \arrow[swap]{ru}{j}
        \arrow{lu}{\beta}
      & {}
    \end{tikzcd}
  \]
\end{proposition}

Die Gruppe $G \times H$ zusammen mit den beiden Inklusionen $i \colon G \to G \times H$ und $j \colon H \to G \times H$ sind also die \enquote{allgemeinste} Möglichkeit, $G$ und $H$ auf kommutierende Weise zu einer gemeinsamen Gruppen $G \times H$ zusammenzufassen.
