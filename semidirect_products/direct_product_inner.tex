\section{Innere direkte Produkte}





\subsection*{Definition}

Es sei $G$ eine Gruppe, und es seien $H, K \subgroup G$ zwei Untergruppen.
Wir wollen untersuchen, wann die Gruppe $G$ dem direkten Produkt $H \times K$ entspricht, d.h.\ wann die Abbildung
\[
          \varphi
  \colon  H \times K
  \to     G \,,
  \quad   (h,k)
  \mapsto hk
\]
ein Gruppenisomorphismus ist.
Ist dies der Fall, so bezeichnen wir $G$ als das \emph{innere direkte Produkt} der beiden Untergruppen $K$ und $H$.





\subsection*{Erste Charakterisierung}

Wir charakterisieren zunächst, unter welchen Bedingungen $\varphi$ ein Gruppenhomomorphismus ist, und wann $\varphi$ surjektiv, bzw.\ injektiv ist.

\begin{itemize}
  \item
    Für alle $(h_1, k_1), (h_2, k_2) \in K \times H$ gelten
    \begin{gather*}
        \varphi((h_1, k_1) (h_2, k_2))
      = \varphi((h_1 h_2, k_1 k_2))
      = h_1 h_2 k_1 k_2 \,,
      \\
        \varphi(h_1, k_1) \varphi(h_2, k_2)
      = h_1 k_1 h_2 k_2 \,.
    \end{gather*}
    Somit ist $\varphi$ genau dann ein Gruppenhomomorphismus, falls
    \[
        h_1 k_1 h_2 k_2
      = h_1 h_2 k_1 k_2
    \]
    für alle $h_1, h_2 \in K$, $k_1, k_2 \in H$ gilt.
    Indem man den Fall $h_1 = k_2 = 1$ betrachtet, ist dies ferner äquivalent dazu, dass
    \[
        k_1 h_2
      = h_2 k_1
    \]
    für alle $h_1 \in K$, $k_2 \in H$ gilt.
    Also ist $\varphi$ genau dann ein Gruppenhomomorphismus, wenn die Untergruppen $K$ und $H$ miteinander kommutieren.
    
  \item
    Die Abbildung $\varphi$ ist surjektiv, wenn sich jedes Element $g \in G$ als $g = hk$ mit $h \in H$ und $k \in K$ darstellen lässt, d.h.\ wenn $G = HK$ gilt.
    
  \item
    Für alle $(h_1, k_1), (h_2, k_2) \in H \times K$ gilt
    \[
            \varphi(h_1, k_1) = \varphi(h_2, k_2)
      \iff  h_1 k_1 = h_2 k_2
      \iff  h_2^{-1} h_1 = k_2 k_1^{-1} \,.
    \]
    Man bemerke, dass das Element $x \define h_2^{-1} h_1 = k_2 k_1^{-1}$ dann bereits in $H \cap K$ enthalten ist.
    
    \begin{itemize}
      \item
        Gilt $H \cap K = 1$, so gilt $x = 1$, und somit $h_1 = h_2$ und $k_1 = k_2$.
        Somit gilt $(h_1, k_1) = (h_2, k_2)$, weshalb $\varphi$ dann injektiv ist.
      \item
        Gilt andererseits $H \cap K \neq 1$, so gibt es ein Element $x \in H \cap K$ mit $x \neq 1$.
        Dann gilt $\varphi(x, x^{-1}) = 1 = \varphi(1,1)$ aber $(x, x^{-1}) \neq (1,1)$.
        Also ist $\varphi$ dann nicht injektiv.
    \end{itemize}
    
    Zusammen zeigt dies, dass $\varphi$ genau dann injektiv ist, wenn $H \cap K = 1$ gilt.
\end{itemize}

\begin{proposition}
  \label{proposition: first characterization}
  Eine Gruppe $G$ ist genau dann das innere direkte Produkt zweier Untergruppen $H, K \subgroup G$, falls $H$ und $K$ miteinander kommutieren, sowie $HK = G$ und $H \cap K = 1$ gelten.
\end{proposition}





\subsection*{Zweite Charakterisierung}

Wir wollen noch eine zweite Charakterisierung innerer direkte Produkte angeben:

\begin{itemize}
  \item
    Wir nehmen zunächst an, dass $G$ das innere direkte Produkt der Untergruppen $H$ und $K$ ist, dass also die Abbildung $\varphi \colon H \times K \to G$ ein Gruppenisomorphismus ist.
    Dann sind die Untergruppen $H, K \subgroup G$ bereits normal in $G$:
    
    \begin{itemize}
      \item
        Man bemerke, dass die Untergruppen $\induced{H}, \induced{K} \subgroup H \times K$ normal sind.
        So gilt beispielsweise
        \[
              (h, k) (h', 1) (h, k)^{-1}
          =   (h, k) (h', 1) (h^{-1}, k^{-1})
          =   (h h' h^{-1}, k k^{-1})
          =   (h h' h^{-1}, 1) \,,
        \]
        was die Normalität von $\induced{H}$ in $H \times K$ zeigt.
        Da $\varphi$ ein Isomorphismus ist, folgt daraus, dass $H = \varphi(\induced{H})$ und $K = \varphi(\induced{K})$ normal in $\varphi(H \times K) = G$ sind.
      \item
        Alternativ lassen sich auch die Normalisatoren $\normalizer{H}{G}$ und $\normalizer{K}{G}$ betrachten:
        Es gilt stets $H \subgroup \normalizer{H}{G}$, und da $H$ und $K$ miteinander kommutieren, gilt auch $K \subgroup \normalizer{H}{G}$.
        Aus $G = HK$ folgt damit, dass bereits $G = \normalizer{H}{G}$ gilt, dass also $H$ normal in $G$ ist.
        Analog ergibt sich die auch Normalität von $K$ in $G$.
    \end{itemize}
    
    Unabhängig von der Vorgehensweise erhalten wir das folgende wichtige Resultat:
    \begin{center}
      \emph{
      Ist $G$ das innere direkte Produkt zweier Untergruppen $H, K \subgroup G$,\\
      so sind $H$ und $K$ bereits beide normal in $G$.
      }
    \end{center}
  \item
    Wir nehmen nun umgekehrt an, dass die Untergruppen $H, K \subgroup G$ beide normal sind, und dass $H \cap K = 1$ gilt.
    Dann kommutieren $H$ und $K$ auch schon miteinander.
    Für alle $h \in H$ und $k \in K$ gilt nämlich, dass
    \[
            \text{$h$ und $k$ kommutieren miteinander}
      \iff  hk = kh
      \iff  h k h^{-1} k^{-1} = 1 \,.
    \]
    Da $K$ normal ist, gilt dabei $h k h^{-1} \in K$, und somit auch $(h k h^{-1}) k^{-1} \in K$.
    Analog gilt aber auch $h (k h^{-1} k^{-1}) \in H$.
    Somit gilt bereits $h k h^{-1} k^{-1} \in H \cap K = 1$.
    Also gilt $h k h^{-1} k^{-1} = 1$, weshalb $h$ und $k$ miteinander kommutieren.
\end{itemize}

Damit erhalten wir ingesamt, dass wir in Proposition~\ref{proposition: first characterization} die Bedingung, dass $H$ und $K$ miteinander kommutieren, durch die Bedingung ersetzen können, dass $H$ und $K$ beide normal in $G$ sind.

\begin{proposition}
  \label{proposition: second characterization}
  Eine Gruppe $G$ ist genau dann das innere direkte Produkt zweier Untergruppen $H, K \subgroup G$, falls $H$ und $K$ beide normal in $G$ sind, sowie $HK = G$ und $H \cap K = 1$ gelten.
\end{proposition}





\subsection*{Für endliche Gruppen}

Ist $G$ endlich, so lässt sich dies Ausnutzen, um Proposition~\ref{proposition: first characterization} und Proposition~\ref{proposition: second characterization} umzuformulieren:

\begin{itemize}
  \item
    Falls $G$ das innere direkte Produkt von $H$ und $K$ ist, so ist $\varphi \colon H \times K \to G$ ein Isomorphismus, und somit inbesondere
    \[
        \card{G}
      = \card{H \times K}
      = \card{H} \cdot \card{K} \,.
    \]
  \item
    Es gelte nun andererseits $\card{G} = \card{H} \cdot \card{K}$.
    Dann sind die Injektvität und Surjektivität von $\varphi \colon H \times K \to G$ äquivalent.
    In Proposition~\ref{proposition: first characterization} und Proposition~\ref{proposition: second characterization} genügt es deshalb jeweils, eine der beiden Bedingungen $HK = G$ und $H \cap K = 1$ zu fordern.
\end{itemize}

Damit erhalten wir für endliches $G$ die folgenden (vier) Charakterisierungen innerer direkter Produkte:

\begin{proposition}
  Eine endliche Gruppe $G$ ist genau dann das innere direkte Produkt zweier Gruppen $H, K \subgroup G$, falls $\card{G} = \card{H} \cdot \card{K}$ gilt, und
  \begin{enumerate}
    \item
      die Untergruppen $H$ und $K$ kommutieren, oder
    \item
      die Untergruppen $H$ und $K$ beide normal sind,
  \end{enumerate}
  und 
  \begin{enumerate}[label=\arabic*'.]
    \item
      es gilt $H \cap K = 1$, oder
    \item
      es gilt $H K = G$.
  \end{enumerate}
\end{proposition}

Besonders hervorheben möchten wir an dieser Stelle die folgende der oberen Charakterisierungen:

\begin{corollary}
  \label{corollary: inner direct product for finite groups}
  Eine endliche Gruppe $G$ ist genau dann das innere direkte Produkt zweier Untergruppen $H, K \subgroup G$, falls $\card{G} = \card{H} \cdot \card{K}$ gilt, $H$ und $K$ beide normal in $G$ sind, und $H \cap K = 1$ gilt.
\end{corollary}

\begin{example}
  \leavevmode
  \begin{enumerate}
    \item
      Für teilerfremde $n, m \geq 1$ betrachten wir die abelsche Gruppe $G \coloneqq \Integer/(nm)$.
      Die Untergruppen $H \coloneqq n\Integer/(nm)$ und $K \coloneqq m\Integer/(nm)$ sind normal in $G$, da $G$ abelsch ist.
      Es gelten $\card{H} = nm/n = m$ und $\card{K} = nm/m = n$.
      Da $\card{H \cap K}$ ein Teiler von $\card{H} = m$ und $\card{K} = n$ ist, folgt damit aus der Teilerfremdheit von $n$ und $m$, dass $\card{H \cap K} = 1$ und somit $H \cap K = 0$.
      Außerdem gilt $\card{G} = mn = \card{H} \card{K}$.
      Somit erhalten wir nach Korollar~\ref{corollary: inner direct product for finite groups}, dass $G$ das innere Produkt der Untergruppen $H$ und $K$ ist.
      
      Inbesondere gilt $G \cong H \times K$.
      Die Gruppen $H$ und $K$ sind als Untergruppen der zyklischen Gruppe $G$ ebenfalls zyklisch.
      Also gelten $H \cong \Integer/m$ und $K \cong \Integer/n$.
      Damit erhalten wir, dass
      \[
              \Integer/(mn)
        \cong \Integer/m \times \Integer/n \,.
      \]
    \item
      Ist allgemeiner $A$ eine endliche abelsche Gruppe, und sind $B, C \subgroup A$ zwei Untergruppen mit $B \cap C = 0$ und $B + C = 0$, oder äquivalent $\card{B} \card{C} = \card{A}$, so gilt $A \cong B \times C$.
    \item
      Es sei $G$ eine endliche Gruppe der Ordnung $\card{G} = pq$ mit $p$ und $q$ prim, so dass $p < q$ und $p \ndivides (q-1)$ gilt.
      Es sei $n_p$ die Anzahl der $p$-Sylowgruppen in $G$, und $n_q$ die Anzahl der $q$-Sylowgruppen.
      \begin{itemize}
        \item
          Es gelten $n_p \equiv 1 \pmod{p}$ und $n_p \divides q$.
          Würde $n_p \neq 1$ gelten, so müsste $n_p = q$ gelten, und somit $q \equiv 1 \pmod{p}$, was aber im Widerspruch zu $p \ndivides (q-1)$ steht.
          Also gilt $n_p = 1$.
          Es gibt also eine eindeutige $p$-Sylowuntergruppe $H \subgroup G$;
          da es sich um die einzige $p$-Sylowuntergruppe von $G$ handelt, ist $H$ normal.
        \item
          Es gelten $n_q \equiv 1 \pmod{q}$ und $n_q \divides p$.
          Würde $n_q \neq 1$ gelten, so wäre $n_q \geq 1+q$, was wegen $p < q$ im Widerspruch zu $p \divides n_q$ stünde.
          Somit gilt $n_q = 1$.
          Es gibt also eine eindeutige $q$-Sylowuntergruppe $K \subgroup G$, die notwendigerweise normal in $G$ ist.
      \end{itemize}
      Die Ordnung $\card{H \cap K}$ teilt $\card{H} = p$ und $\card{K} = q$, weshalb $\card{H \cap K} = 1$ und somit $H \cap K = 1$ gilt.
      Außerdem gilt $\card{G} = pq = \card{H} \card{K}$.
      Somit ergibt sich nach Korollar~\ref{corollary: inner direct product for finite groups}, dass $G$ das innere direkte Produkt der Gruppen $H$ und $K$ ist.
      
      Inbesondere gilt $G \cong H \times K$.
      Da $\card{H} = p$ und $\card{K} = q$ prim sind, gelten $H \cong \Integer/p$ und $K \cong \Integer/q$.
      Damit erhalten wir aus der Teilerfremdheit von $p$ und $q$, dass
      \[
              G
        \cong H \times K
        \cong \Integer/p \times \Integer/q
        =     \Integer/(pq) \,.
      \]
      
    \item
      Es sei $n \geq 1$ ungerade und es sei $G \coloneqq D_{2n}$ die Diedergruppe der Ordnung $\card{D_{2n}} = 2 \cdot 2n = 4n$.
      Ist $\rho \in D_{2n}$ eine Rotation um den Winkel $2\pi/(2n)$ und $\tau \in D_{2n}$ eine Spiegelung, so gilt
      \[
          D_{2n}
        = \{ 1, \rho, \dotsc, \rho^{2n-1}, \tau, \tau \rho, \dotsc, \tau \rho^{2n-1} \} \,,
      \]
      und Gruppenstruktur ist eindeutig durch die beiden \emph{Relationen}
      \[
              \rho^{2n} = 1 \,,
        \quad \tau^2    = 1 \,,
        \quad \tau \rho \tau^{-1} = \rho^{-1}
      \]
      festgelegt:
      Für alle $i, j, k, l \in \Integer$ gilt
      \[
          \tau^i \rho^j \cdot \tau^k \rho^l
        = \tau^i  \tau^k \rho^{(-1)^k j} \rho^l
        = \tau^{i+k} \rho^{(-1)^k j + l} \,.
      \]
      Das Zentrum von $D_{2n}$ ist
      \[
          \groupcenter{D_{2n}}
        = \{1, \rho^n\} \,,
      \]
      wobei $\rho^n$ geometrisch gesehen die Rotation um $180^\circ$ ist, also die Multiplikation mit $-1$.
      Außerdem ist
      \[
                  D
        \coloneqq \{
                    \tau^i \rho^j
                  \suchthat
                    i \in \Integer,
                    j \in 2\Integer
                  \}
      \]
      eine Untergruppe von $D_{2n}$ vom Index $2$.
      Inbesondere sind $\groupcenter{D_{2n}}$ und $D$ normal in $D_{2n}$ mit $\card{D_{2n}} = 4n = 2 \cdot 2n = \card{\groupcenter{D_{2n}}} \cdot \card{D}$.
      Da $n$ ungerade ist, gilt $\rho^n \notin D$, und somit $\groupcenter{D_{2n}} \cap D = 1$.
      Somit gilt nach Korollar~\ref{corollary: inner direct product for finite groups}, dass $D_{2n}$ das innere direkte Produkt der beiden Untergruppen $\groupcenter{D_{2n}}$ und $D$ ist.
      
      Inbesondere gilt $D_{2n} \cong \groupcenter{D_{2n}} \times D$.
      Dabei gelten $\groupcenter{D_{2n}} \cong \Integer/2$ und $D \cong D_n$, und somit
      \[
        D_{2n} \cong D_n \times \Integer/2 \,.
      \]
  \end{enumerate}
\end{example}


