\section{}

Wir betrachten im Folgenden nur die Fälle $n \geq 3$, da die auf dem Übungszettel gegebene Defition für $D_1$ und $D_2$ nicht (ohne weiteres) funktoniert.





\subsection{}

Es gibt verschiedene Möglichkeiten, die (Anzahl der) Elemente von $D_n$ zu bestimmen:

\begin{itemize}
  \item
    Es gibt $n$ Rotation, jeweils um Vielfache von $360^\circ/n$, bzw.\ um $2\pi/n$.
    Zudem gibt es noch $n$ Spiegelungen:
    \begin{itemize}
      \item
        Ist $n$ ungerade, so gehen die Spiegelungsachsen durch einen der Eckpunkte, sowie den Mittelpunkt der gegebenüberliegenden Kante.
        % TODO: Adding images for n = 3, 5$.
      \item
        Ist $n$ gerade, so gibt es zwei Arten von Spiegelungen:
        \begin{itemize}
          \item
            Es gibt $n/2$ Spiegelungen, deren Spiegelungsachse durch einen Eckspunkt sowie den gegenüberliegenden Eckpunkt gehen.
          \item
            Es gibt $n/2$ Spiegelungen, deren Spiegelungsachse durch den Mittelpunkte einer Kante sowie den Mittelpunkt der gegenüberliegenden Kante gehen.
        \end{itemize}
        % TODO: Adding images for n = 6.
    \end{itemize}
    Damit ergeben sich insgesamt $2n$ Isometrien.
    
  \item
    Es sei $x$ einer der Eckpunkte und $x'$ einer der zu $x$ benachbarten Eckpunkte.
    Dann ist jede Isometrie des $n$-Ecks durch die Wirkung auf den benachbarten Eckpunkten $x$ und $x'$ bereits eindeutig bestimmt.
    
    Der Eckpunkt $x$ kann auf jeden der anderen Eckpunkte abgebildet werden, wofür es $n$ Möglichkeiten gibt.
    Wird der Eckpunkt $x$ auf einen Eckpunkt $y$ abgebildet, so kann $x'$ auf jeden der beiden zu $y$ benachbarten Eckpunkt geschickt werden.
    
    Somit ergeben sich $2n$ Isometrien
\end{itemize}

Um zu zeigen, dass $D_n$ nicht abelsch ist, nummerieren wir die Eckpunkte des $n$-Ecks mit den Elementent von $\Integer/n$, so dass der Eckpunkt $\class{k}$ mit den Eckpunkten $\class{k-1}$ und $\class{k+1}$ benachbart sind.
% TODO: Adding images for n = 3, 4, 5.

Die Rotation um $360^\circ/n$ ist dann durch
\[
          r
  \colon  \Integer/n
  \to     \Integer/n,
  \quad   \class{k}
  \mapsto \class{k+1}
\]
gegeben.
Die Spiegelung, deren Achse durch den Eckpunkt $\class{0}$ geht, ist dann durch
\[
          r
  \colon  \Integer/n
  \to     \Integer/n,
  \quad   \class{k}
  \mapsto \class{-k}
\]
gegeben.
% TODO: Adding images.
Es gilt
\begin{gather*}
    (r \circ s)(\class{0})
  = r(s(\class{0}))
  = r(\class{0})
  = \class{1}
\shortintertext{aber}
    (s \circ r)(\class{0})
  = s(r(\class{0}))
  = s(\class{0})
  = \class{-1},
\end{gather*}
wobei $\class{1} \neq \class{-1}$ da $n \geq 3$.
% TODO: Adding images.





\addtocounter{subsection}{1}





\addtocounter{subsection}{1}




