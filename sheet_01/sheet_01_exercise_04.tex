\section{}

Wir betrachten im Folgenden nur die Fälle $n \geq 3$, da die auf dem Übungszettel gegebene Defition für $D_1$ und $D_2$ nicht (ohne weiteres) funktoniert.





\subsection{}

Es gibt verschiedene Möglichkeiten, die (Anzahl der) Elemente von $D_n$ zu bestimmen:

\begin{itemize}
  \item
    Es gibt $n$ Rotation, jeweils um Vielfache von $360^\circ/n$, bzw.\ um $2\pi/n$.
    Zudem gibt es noch $n$ Spiegelungen:
    \begin{itemize}
      \item
        Ist $n$ ungerade, so gehen die Spiegelungsachsen durch einen der Eckpunkte, sowie den Mittelpunkt der gegebenüberliegenden Kante.
        % TODO: Adding images for n = 3, 5$.
      \item
        Ist $n$ gerade, so gibt es zwei Arten von Spiegelungen:
        \begin{itemize}
          \item
            Es gibt $n/2$ Spiegelungen, deren Spiegelungsachse durch einen Eckspunkt sowie den gegenüberliegenden Eckpunkt gehen.
          \item
            Es gibt $n/2$ Spiegelungen, deren Spiegelungsachse durch den Mittelpunkte einer Kante sowie den Mittelpunkt der gegenüberliegenden Kante gehen.
        \end{itemize}
        % TODO: Adding images for n = 6.
    \end{itemize}
    Damit ergeben sich insgesamt $2n$ Isometrien.
    
  \item
    Es sei $x$ einer der Eckpunkte und $x'$ einer der zu $x$ benachbarten Eckpunkte.
    Dann ist jede Isometrie des $n$-Ecks durch die Wirkung auf den benachbarten Eckpunkten $x$ und $x'$ bereits eindeutig bestimmt.
    
    Der Eckpunkt $x$ kann auf jeden der anderen Eckpunkte abgebildet werden, wofür es $n$ Möglichkeiten gibt.
    Wird der Eckpunkt $x$ auf einen Eckpunkt $y$ abgebildet, so kann $x'$ auf jeden der beiden zu $y$ benachbarten Eckpunkt geschickt werden.
    
    Somit ergeben sich $2n$ Isometrien
\end{itemize}

Um zu zeigen, dass $D_n$ nicht abelsch ist, nummerieren wir die Eckpunkte des $n$-Ecks mit den Elementent von $\Integer/n$, so dass der Eckpunkt $\class{k}$ mit den Eckpunkten $\class{k-1}$ und $\class{k+1}$ benachbart sind.
% TODO: Adding images for n = 3, 4, 5.

Die Rotation um $360^\circ/n$ ist dann durch
\[
          r
  \colon  \Integer/n
  \to     \Integer/n,
  \quad   \class{k}
  \mapsto \class{k+1}
\]
gegeben.
Die Spiegelung, deren Achse durch den Eckpunkt $\class{0}$ geht, ist dann durch
\[
          r
  \colon  \Integer/n
  \to     \Integer/n,
  \quad   \class{k}
  \mapsto \class{-k}
\]
gegeben.
% TODO: Adding images.
Es gilt
\begin{gather*}
    (r \circ s)(\class{0})
  = r(s(\class{0}))
  = r(\class{0})
  = \class{1}
\shortintertext{aber}
    (s \circ r)(\class{0})
  = s(r(\class{0}))
  = s(\class{0})
  = \class{-1},
\end{gather*}
wobei $\class{1} \neq \class{-1}$ da $n \geq 3$.
% TODO: Adding images.





\subsection{}

Das regelmäßige $n$-Eck lässt sich in die Ebene $\Real^2$ einbetten, so dass der Nullpunkt $(0,0)$ der Schwerpunkt des $n$-Ecks ist, und einer der Eckpunkte der $n$-Ecks auf der $x$-Achse liegt.
Dann lassen sich die Elemente von $D_n$ als Rotationen und Spiegelungen der Ebene auffassen, und somit als Rotations- und Spiegelungsmatrizen.
Für $\alpha \in \Real$ ist die Rotation um den Winkel $\alpha$ durch die Matrix
\[
            R_\alpha
  \coloneqq \begin{pmatrix*}[r]
              \cos \alpha & -\sin \alpha  \\
              \sin \alpha &  \cos \alpha
            \end{pmatrix*}
\]
gegeben, und die Spiegelung an der Gerade mit Winkel $\alpha$ (zur $x$-Achse) ist durch die Matrix
\[
            S_\alpha
  \coloneqq \begin{pmatrix*}[r]
              \cos 2\alpha  &  \sin 2\alpha \\
              \sin 2\alpha  & -\cos 2\alpha
            \end{pmatrix*}
\]
gegeben.
Die Gruppe $D_n$ ist dann durch die Matrizen
\[
            \hat{D}_n
  \coloneqq \{
              R_{k 2\pi/n}
            \suchthat
              k = 0, \dotsc, n-1
            \}
            \cup
            \{
              S_{k \pi/n}
            \suchthat
              k = 0, \dotsc, n-1
            \}
\]
gegeben.
Es ist $\hat{D}_n \subgroup \orthogonal{2}$ eine Untergruppe, weshalb wir den surjektiven Gruppenhomomorphismus $\restrict{\det}{\orthogonal{2}} \colon \orthogonal{2} \to \{1, -1\}$ zu einem Gruppenhomomorphismus $\restrict{\det}{\hat{D}_n} \colon \hat{D}_n \to \{1,-1\}$ einschränken.
Es gilt $\det R_\alpha = 1$ und $\det S_\alpha = -1$ für alle $\alpha \in \Real$, weshalb auch $\restrict{\det}{\hat{D}_n}$ noch surjektiv ist.
Damit erhalten wir einen surjektiven Gruppenhomomorphismus
\[
          \hat{g}
  \colon  D_n
  \to     \{1, -1\},
  \quad   x
  \mapsto \begin{cases}
            \phantom{-}1  & \text{falls $x$ eine Rotation ist}, \\
                      -1  & \text{falls $x$ eine Spiegelung ist}.
          \end{cases}
\]
Da $\{1, -1\} \cong \Integer/2$ lässt sich $\hat{g}$ auch als ein Gruppenhomomorphismus
\[
          g
  \colon  D_n
  \to     \Integer/2,
  \quad   x
  \mapsto \begin{cases}
            \class{0} & \text{falls $x$ eine Rotation ist}, \\
            \class{1} & \text{falls $x$ eine Spiegelung ist},
          \end{cases}
\]
auffassen.





\subsection{}

Der Kern von $g$ besteht genau aus den Rotationen.

Jedes Element $x \in D_n$ liefert einen Gruppenhomomorphismus
\[
          \tilde{s}
  \colon  \Integer
  \to     D_n,
  \quad   n
  \mapsto x^n.
\]
Ist $x$ eine Spiegelung, so gilt $x \neq 1$ aber $x^2 = 1$, und somit $\ker \tilde{s} = 2\Integer$.
Somit induziert $\tilde{s}$ einen wohldefinierten Gruppenhomomorphismus
\[
          s
  \colon  \Integer/2
  \to     D_n,
  \quad   \class{n}
  \mapsto x^n.
\]
Dann gilt
\[
    (g \circ s)(\class{1})
  = g(s(\class{1}))
  = g(x^1)
  = g(x)
  = \class{1}.,
\]
sowie $(g \circ s)(\class{0}) = \class{0}$ da $g \circ s$ ein Gruppenhomomorphismus ist.
Somit gilt $g \circ s = \id_{\Integer/2}$.





\subsection*{Bemerkungen}

\begin{itemize}
  \item
    Manche Autoren schreiben $D_{2n}$ statt $D_n$, d.h.\ $D_{2n}$ ist die Diedergruppe von Ordnung $2n$.
    
  \item
    Ist $r \in D_n$ eine Rotation um den Winkel $2\pi/n$ und $s \in D_n$ eine Spiegelung, so gilt
    \begin{itemize}
      \item
        $\ord{r} = n$,
      \item
        $\ord{s} = 2$,
      \item
        $s r = r^{-1} s$.
    \end{itemize}
    Durch diese Bedingunegn sind die Elemente und die Gruppenstruktur von $D_n$ bereits eindeutig bestimmt:
    \begin{itemize}
      \item
        Es gilt $\generated{r} = \{1, r, \dotsc, r^{n-1}\}$ mit $\card{ \generated{r} } = n$.
        Für $\generated{r} s = \{s, rs, \dotsc, r^{n-1} s\}$ gilt dann auch $\card{ \generated{r} s } = n$.
      \item
        Es gilt $s \notin \generated{r}$, da $s$ keine Rotation ist.
        Deshalb sind $\generated{r}$ und $\generated{r} s$ disjunkt.
      \item
        Da $\card{D_n} = 2n = n + n = \card{\generated{r}} + \card{\generated{r} s}$ gilt, ist $D_n$ bereits die disjunkte Vereinigung von $\generated{r}$ und $\generated{r} s$.
        Es lässt sich also jedes Element $x \in D_n$ als $x = r^i s^j$ mit eindeutigen $0 \leq i \leq n-1$ und $0 \leq j \leq 1$ darstellen.
    \end{itemize}
    Das zeigt, dass die Elemente von $D_n$ eindeutig bestimmt sind.
    \begin{itemize}[resume]
      \item
        Aus $s r = r^{-1} s$ folgt, dass $s r s^{-1} = r$.
        Da die Abbildung $c \colon G \to G$, $x \mapsto s x s^{-1}$ ein Gruppenhomomorphismus ist, folgt ferner, dass bereits $s r^i s^{-1} = r^{-i}$ für alle $i \in \Integer$ gilt.
        Für alle $i \in \Integer$ und $j \in \Integer$ gilt also $s^j r^i = r^{(-1)^j i} s^j$ (da $s^2 = 1$ gilt, genügt es, die Fälle $j = 0$ und $j = 1$ zu betrachten).
      \item
        Für alle $i, j, k , l \in \Integer$ gilt somit
        \[
            r^i s^j r^k s^l
          = r^i r^{(-1)^j k} s^j s^l
          = r^{i + (-1)^j } s^{j + l}
          = r^{(i + (-1)^j) \bmod 2} s^{(j + l) \bmod 2}
        \]
    \end{itemize}
    Also ist auch die Gruppenstruktur auf $D_n$ bereits bestimmt.
    
  \item
    Inbesondere ließe sich die Diedergruppe $D_n$ auch auf rein algebraische Weise als die Menge $(\Integer/n) \times (\Integer/2)$ zusammen mit der Verknüpfung
    \begin{equation}
      \label{equation: algebraic construction of the dihedral group}
                (\class{i}, \class{j}) \cdot (\class{k}, \class{l})
      \coloneqq (\class{i} + (-1)^j \class{k}, \class{j} + \class{l})
    \end{equation}
    definieren.
    
  \item
    Auf diese Weise lassen sich auch die Gruppen $D_1$ und $D_2$ definieren:
    \begin{itemize}
      \item
        Die Gruppe $D_1$ lässt sich als die Menge $(\Integer/1) \times (\Integer/2)$ mit Verknüpfung~\eqref{equation: algebraic construction of the dihedral group} definieren.
        Dann ist die Bijektion $(\Integer/1) \times (\Integer/2) \to \Integer/2$, $(x,y) \mapsto y$ ein Gruppenhomomoprhismus, und somit $D_1 \cong \Integer/2$.
      \item
        Die Gruppe $D_1$ lässt sich als die Menge $(\Integer/2) \times (\Integer/2)$ mit Verknüpfung~\eqref{equation: algebraic construction of the dihedral group} definieren.
        Für die Gruppe $P \coloneqq (\Integer/2) \times (\Integer/2)$ mit der \enquote{üblichen} Produkt-Gruppenstruktur ist dann die Abbildung
        \[
                  P
          \to     D_2,
          \quad   (x, y)
          \mapsto (x+y,y)
        \]
        ein Gruppenhomomorphismus, und somit $D_2 \cong \Integer/2 \times \Integer/2$ als Gruppen.
    \end{itemize}
    
    Ähnlich wie die Diedergruppen $D_n$ mit $n \geq 3$ lassen sich die Diedergruppen $D_1$ und $D_2$ ebenfalls geometrisch definieren:
    
    \begin{itemize}[resume]
      \item
        Das regelmäßige $1$-Eck und $2$-Eck lassen sich wie folgt in $\Real^2$ einbetten:
        \begin{center}
          \begin{tikzpicture}
            \draw[very thin, ->] (-2,0) -- (2,0) node[anchor = west, black] {$x$};
            \draw[very thin, ->] (0,-2) -- (0,2) node[anchor = south, black] {$y$};
            \fill (1,0) circle (0.1);
          \end{tikzpicture}
          \hspace{2em}
          \begin{tikzpicture}
            \draw[very thin, ->] (-2,0) -- (2,0) node[anchor = west, black] {$x$};;
            \draw[very thin, ->] (0,-2) -- (0,2) node[anchor = south, black] {$y$};
            \fill (-1,0) circle (0.1);
            \fill (1,0) circle (0.1);
            \draw[very thick] (-1,0) -- (1,0);
          \end{tikzpicture}
        \end{center}
      Die Diedergruppe $D_n$ für $n = 1, 2$ besteht dann aus all jeden Isometrien der umgebenen Ebene $\Real^2$, welche diese eingebettete Version des $n$-Ecks unverändert lassen:
      \begin{itemize}
        \item
          Die Diedergruppe $D_1$ besteht also aus all jenen Isometrien der Ebene, welche den Punkt $(1,0)$ unverändert lassen.
          Hierfür kommen nur die Identität und die Spiegelung an der $x$-Achse in Frage.
          Somit gilt $D_1 \cong \Integer/2$.
        \item
          Die Diedergruppe $D_2$ besteht aus all jenen Isometrien der Ebene, welche das eingezeichnete Interval $[-1, 1] \times \{0\}$ unverändert lassen.
          Dies sind die Identität, die Spiegelung an den beiden Achsen, sowie die Komposition dieser beiden Spiegelungen (welche die Spiegelung am Koordinatenursprung ist).
          Da die Spiegelungen an den Achsen miteinander kommutieren, ergibt sich, dass $D_2 \cong \Integer/2 \times \Integer/2$.
      \end{itemize}
    \end{itemize}
\end{itemize}





