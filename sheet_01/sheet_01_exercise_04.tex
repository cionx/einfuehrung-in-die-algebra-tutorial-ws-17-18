\section{}

Wir betrachten im Folgenden nur die Fälle $n \geq 3$, da die auf dem Übungszettel gegebene Defition für $D_1$ und $D_2$ nicht (ohne weiteres) funktoniert.





\subsection{}

Es gibt verschiedene Möglichkeiten, die (Anzahl der) Elemente von $D_n$ zu bestimmen:

\begin{itemize}
  \item
    Es gibt $n$ Rotation, jeweils um Vielfache von $360^\circ/n$, bzw.\ um $2\pi/n$.
    Zudem gibt es noch $n$ Spiegelungen:
    \begin{itemize}
      \item
        Ist $n$ ungerade, so gehen die Spiegelungsachsen durch einen der Eckpunkte, sowie den Mittelpunkt der gegebenüberliegenden Kante.
        % TODO: Adding images for n = 3, 5$.
      \item
        Ist $n$ gerade, so gibt es zwei Arten von Spiegelungen:
        \begin{itemize}
          \item
            Es gibt $n/2$ Spiegelungen, deren Spiegelungsachse durch einen Eckspunkt sowie den gegenüberliegenden Eckpunkt gehen.
          \item
            Es gibt $n/2$ Spiegelungen, deren Spiegelungsachse durch den Mittelpunkte einer Kante sowie den Mittelpunkt der gegenüberliegenden Kante gehen.
        \end{itemize}
        % TODO: Adding images for n = 6.
    \end{itemize}
    Damit ergeben sich insgesamt $2n$ Isometrien.
    
  \item
    Es sei $x$ einer der Eckpunkte und $x'$ einer der zu $x$ benachbarten Eckpunkte.
    Dann ist jede Isometrie des $n$-Ecks durch die Wirkung auf den benachbarten Eckpunkten $x$ und $x'$ bereits eindeutig bestimmt.
    
    Der Eckpunkt $x$ kann auf jeden der anderen Eckpunkte abgebildet werden, wofür es $n$ Möglichkeiten gibt.
    Wird der Eckpunkt $x$ auf einen Eckpunkt $y$ abgebildet, so kann $x'$ auf jeden der beiden zu $y$ benachbarten Eckpunkt geschickt werden.
    
    Somit ergeben sich $2n$ Isometrien
\end{itemize}

Um zu zeigen, dass $D_n$ nicht abelsch ist, nummerieren wir die Eckpunkte des $n$-Ecks mit den Elementent von $\Integer/n$, so dass der Eckpunkt $\class{k}$ mit den Eckpunkten $\class{k-1}$ und $\class{k+1}$ benachbart sind.
% TODO: Adding images for n = 3, 4, 5.

Die Rotation um $360^\circ/n$ ist dann durch
\[
          r
  \colon  \Integer/n
  \to     \Integer/n,
  \quad   \class{k}
  \mapsto \class{k+1}
\]
gegeben.
Die Spiegelung, deren Achse durch den Eckpunkt $\class{0}$ geht, ist dann durch
\[
          r
  \colon  \Integer/n
  \to     \Integer/n,
  \quad   \class{k}
  \mapsto \class{-k}
\]
gegeben.
% TODO: Adding images.
Es gilt
\begin{gather*}
    (r \circ s)(\class{0})
  = r(s(\class{0}))
  = r(\class{0})
  = \class{1}
\shortintertext{aber}
    (s \circ r)(\class{0})
  = s(r(\class{0}))
  = s(\class{0})
  = \class{-1},
\end{gather*}
wobei $\class{1} \neq \class{-1}$ da $n \geq 3$.
% TODO: Adding images.





\subsection{}

Das regelmäßige $n$-Eck lässt sich in die Ebene $\Real^2$ einbetten, so dass der Nullpunkt $(0,0)$ der Schwerpunkt des $n$-Ecks ist, und einer der Eckpunkte der $n$-Ecks auf der $x$-Achse liegt.
Dann lassen sich die Elemente von $D_n$ als Rotationen und Spiegelungen der Ebene auffassen, und somit als Rotations- und Spiegelungsmatrizen.
Für $\alpha \in \Real$ ist die Rotation um den Winkel $\alpha$ durch die Matrix
\[
            R_\alpha
  \coloneqq \begin{pmatrix*}[r]
              \cos \alpha & -\sin \alpha  \\
              \sin \alpha &  \cos \alpha
            \end{pmatrix*}
\]
gegeben, und die Spiegelung an der Gerade mit Winkel $\alpha$ (zur $x$-Achse) ist durch die Matrix
\[
            S_\alpha
  \coloneqq \begin{pmatrix*}[r]
              \cos 2\alpha  &  \sin 2\alpha \\
              \sin 2\alpha  & -\cos 2\alpha
            \end{pmatrix*}
\]
gegeben.
Die Gruppe $D_n$ ist dann durch die Matrizen
\[
            \hat{D}_n
  \coloneqq \{
              R_{k 2\pi/n}
            \suchthat
              k = 0, \dotsc, n-1
            \}
            \cup
            \{
              S_{k \pi/n}
            \suchthat
              k = 0, \dotsc, n-1
            \}
\]
gegeben.
Es ist $\hat{D}_n \subgroup \orthogonal{2}$ eine Untergruppe, weshalb wir den surjektiven Gruppenhomomorphismus $\restrict{\det}{\orthogonal{2}} \colon \orthogonal{2} \to \{1, -1\}$ zu einem Gruppenhomomorphismus $\restrict{\det}{\hat{D}_n} \colon \hat{D}_n \to \{1,-1\}$ einschränken.
Es gilt $\det R_\alpha = 1$ und $\det S_\alpha = -1$ für alle $\alpha \in \Real$, weshalb auch $\restrict{\det}{\hat{D}_n}$ noch surjektiv ist.
Damit erhalten wir einen surjektiven Gruppenhomomorphismus
\[
          \hat{g}
  \colon  D_n
  \to     \{1, -1\},
  \quad   x
  \mapsto \begin{cases}
            \phantom{-}1  & \text{falls $x$ eine Rotation ist}, \\
                      -1  & \text{falls $x$ eine Spiegelung ist}.
          \end{cases}
\]
Da $\{1, -1\} \cong \Integer/2$ lässt sich $\hat{g}$ auch als ein Gruppenhomomorphismus
\[
          g
  \colon  D_n
  \to     \Integer/2,
  \quad   x
  \mapsto \begin{cases}
            \class{0} & \text{falls $x$ eine Rotation ist}, \\
            \class{1} & \text{falls $x$ eine Spiegelung ist},
          \end{cases}
\]
auffassen.





\subsection{}

Jedes Element $x \in D_n$ liefert einen Gruppenhomomorphismus
\[
          \tilde{s}
  \colon  \Integer
  \to     D_n,
  \quad   n
  \mapsto x^n.
\]
Ist $x$ eine Spiegelung, so gilt $x \neq 1$ aber $x^2 = 1$, und somit $\ker \tilde{s} = 2\Integer$.
Somit induziert $\tilde{s}$ einen wohldefinierten Gruppenhomomorphismus
\[
          s
  \colon  \Integer/2
  \to     D_n,
  \quad   \class{n}
  \mapsto x^n.
\]
Dann gilt
\[
    (g \circ s)(\class{1})
  = g(s(\class{1}))
  = g(x^1)
  = g(x)
  = \class{1}.,
\]
sowie $(g \circ s)(\class{0}) = \class{0}$ da $g \circ s$ ein Gruppenhomomorphismus ist.
Somit gilt $g \circ s = \id_{\Integer/2}$.





\addtocounter{subsection}{1}
